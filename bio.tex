% Options for packages loaded elsewhere
\PassOptionsToPackage{unicode}{hyperref}
\PassOptionsToPackage{hyphens}{url}
%
\documentclass[
]{article}
\usepackage{lmodern}
\usepackage{amssymb,amsmath}
\usepackage{ifxetex,ifluatex}
\ifnum 0\ifxetex 1\fi\ifluatex 1\fi=0 % if pdftex
  \usepackage[T1]{fontenc}
  \usepackage[utf8]{inputenc}
  \usepackage{textcomp} % provide euro and other symbols
\else % if luatex or xetex
  \usepackage{unicode-math}
  \defaultfontfeatures{Scale=MatchLowercase}
  \defaultfontfeatures[\rmfamily]{Ligatures=TeX,Scale=1}
\fi
% Use upquote if available, for straight quotes in verbatim environments
\IfFileExists{upquote.sty}{\usepackage{upquote}}{}
\IfFileExists{microtype.sty}{% use microtype if available
  \usepackage[]{microtype}
  \UseMicrotypeSet[protrusion]{basicmath} % disable protrusion for tt fonts
}{}
\makeatletter
\@ifundefined{KOMAClassName}{% if non-KOMA class
  \IfFileExists{parskip.sty}{%
    \usepackage{parskip}
  }{% else
    \setlength{\parindent}{0pt}
    \setlength{\parskip}{6pt plus 2pt minus 1pt}}
}{% if KOMA class
  \KOMAoptions{parskip=half}}
\makeatother
\usepackage{xcolor}
\IfFileExists{xurl.sty}{\usepackage{xurl}}{} % add URL line breaks if available
\IfFileExists{bookmark.sty}{\usepackage{bookmark}}{\usepackage{hyperref}}
\hypersetup{
  hidelinks,
  pdfcreator={LaTeX via pandoc}}
\urlstyle{same} % disable monospaced font for URLs
\setlength{\emergencystretch}{3em} % prevent overfull lines
\providecommand{\tightlist}{%
  \setlength{\itemsep}{0pt}\setlength{\parskip}{0pt}}
\setcounter{secnumdepth}{-\maxdimen} % remove section numbering
\ifluatex
  \usepackage{selnolig}  % disable illegal ligatures
\fi

\author{}
\date{}

\begin{document}

\hypertarget{biology-the-science-of-life}{%
\section{Biology: The Science of
Life}\label{biology-the-science-of-life}}

\href{https://en.wikipedia.org/wiki/Biology}{Biology} is the natural
science that studies life and living organisms, including their physical
structure, chemical processes, molecular interactions, physiological
mechanisms, development and evolution. Despite the complexity of the
science, certain unifying concepts consolidate it into a single,
coherent field. Biology recognizes the cell as the basic unit of life,
genes as the basic unit of heredity, and evolution as the engine that
propels the creation and extinction of species. Living
\href{https://en.wikipedia.org/wiki/Organism}{organisms} (from Greek
ὀργανισμός, organismos, from ὄργανον, organon, i.e.~``instrument,
implement, tool, organ of sense or apprehension'') are open systems that
survive by transforming energy and decreasing their local entropy to
maintain a stable and vital condition defined as homeostasis.

(ref:bumble) A Common Eastern Bumble Bee (\emph{Bombus impatiens})
picking up nectar and pollen from a spear cistle (\emph{Cirsium
vulgare}) on Winter Island in Salem, Massachusetts.

\texttt{\{r\ bumblebee,\ fig.cap=\textquotesingle{}(ref:bumble)\textquotesingle{},\ echo=FALSE,\ message=FALSE,\ warning=FALSE\}\ knitr::include\_graphics("./figures/life/IMG\_1024.jpg")}

Sub-disciplines of biology are defined by the research methods employed
and the kind of system studied: theoretical biology uses mathematical
methods to formulate quantitative models while experimental biology
performs empirical experiments to test the validity of proposed theories
and understand the mechanisms underlying life and how it appeared and
evolved from non-living matter about 4 billion years ago through a
gradual increase in the complexity of the system.

Biology derives from the Ancient Greek words of βίος; romanized bíos
meaning ``life'' and -λογία; romanized logía (-logy) meaning ``branch of
study'' or ``to speak''. Those combined make the Greek word βιολογία;
romanized biología meaning biology. Despite this, the term βιολογία as a
whole didn't exist in Ancient Greek. The first to borrow it was the
English and French (biologie). Since the advent of the scientific era,
reanalyzable as a compound using the combining forms bio + logy.

The Latin-language form of the term first appeared in 1736 when Swedish
scientist \href{https://en.wikipedia.org/wiki/Carl_Linnaeus}{Carl
Linnaeus} (Carl von Linné) used biologi in his Bibliotheca Botanica. It
was used again in 1766 in a work entitled Philosophiae naturalis sive
physicae: tomus III, continens geologian, biologian, phytologian
generalis, by
\href{https://en.wikipedia.org/wiki/Michael_Christoph_Hanow}{Michael
Christoph Hanov}, a disciple of
\href{https://en.wikipedia.org/wiki/Christian_Wolff_(philosopher)}{Christian
Wolff}. The first German use, Biologie, was in a 1771 translation of
Linnaeus' work. In 1797, Theodor Georg August Roose used the term in the
preface of a book, Grundzüge der Lehre van der Lebenskraft.
\href{https://en.wikipedia.org/wiki/Karl_Friedrich_Burdach}{Karl
Friedrich Burdach} used the term in 1800 in a more restricted sense of
the study of human beings from a morphological, physiological and
psychological perspective (Propädeutik zum Studien der gesammten
Heilkunst). The term came into its modern usage with the six-volume
treatise Biologie, oder Philosophie der lebenden Natur (1802--22) by
\href{https://en.wikipedia.org/wiki/Gottfried_Reinhold_Treviranus}{Gottfried
Reinhold Treviranus}, who announced:

\begin{quote}
The objects of our research will be the different forms and
manifestations of life, the conditions and laws under which these
phenomena occur, and the causes through which they have been effected.
The science that concerns itself with these objects we will indicate by
the name biology {[}Biologie{]} or the doctrine of life
{[}Lebenslehre{]}.
\end{quote}

Although modern biology is a relatively recent development, sciences
related to and included within it have been studied since ancient times.
Natural philosophy was studied as early as the ancient civilizations of
Mesopotamia, Egypt, the Indian subcontinent, and China. However, the
origins of modern biology and its approach to the study of nature are
most often traced back to ancient Greece. While the formal study of
medicine dates back to Pharaonic Egypt, it was
\href{https://en.wikipedia.org/wiki/Aristotle}{Aristotle} (384--322 BC)
who contributed most extensively to the development of biology.
Especially important are his History of Animals and other works where he
showed naturalist leanings, and later more empirical works that focused
on biological causation and the diversity of life. Aristotle's successor
at the Lyceum,
\href{https://en.wikipedia.org/wiki/Theophrastus}{Theophrastus}, wrote a
series of books on botany that survived as the most important
contribution of antiquity to the plant sciences, even into the Middle
Ages.

Scholars of the medieval Islamic world who wrote on biology included
\href{https://en.wikipedia.org/wiki/Al-Jahiz}{Al-Jahiz} (781--869),
\href{https://en.wikipedia.org/wiki/Abu_Hanifa_Dinawari}{Al-Dīnawarī}
(828--896), who wrote on botany, and Rhazes (865--925) who wrote on
anatomy and physiology. Medicine was especially well studied by Islamic
scholars working in Greek philosopher traditions, while natural history
drew heavily on Aristotelian thought, especially in upholding a fixed
hierarchy of life.

Biology began to quickly develop and grow with
\href{https://en.wikipedia.org/wiki/Antonie_van_Leeuwenhoek}{Anton van
Leeuwenhoek's} dramatic improvement of the microscope. It was then that
scholars discovered spermatozoa, bacteria, infusoria and the diversity
of microscopic life. Investigations by Jan Swammerdam led to new
interest in entomology and helped to develop the basic techniques of
microscopic dissection and staining.

Advances in microscopy also had a profound impact on biological
thinking. In the early 19th century, a number of biologists pointed to
the central importance of the cell. Then, in 1838, Schleiden and Schwann
began promoting the now universal ideas that (1) the basic unit of
organisms is the cell and (2) that individual cells have all the
characteristics of life, although they opposed the idea that (3) all
cells come from the division of other cells. Thanks to the work of
Robert Remak and Rudolf Virchow, however, by the 1860s most biologists
accepted all three tenets of what came to be known as cell theory.

Meanwhile, taxonomy and classification became the focus of natural
historians. Carl Linnaeus published a basic taxonomy for the natural
world in 1735 (variations of which have been in use ever since), and in
the 1750s introduced scientific names for all his species.
\href{https://en.wikipedia.org/wiki/Georges-Louis_Leclerc,_Comte_de_Buffon}{Georges-Louis
Leclerc}, Comte de Buffon, treated species as artificial categories and
living forms as malleable---even suggesting the possibility of common
descent. Although he was opposed to evolution, Buffon is a key figure in
the history of evolutionary thought; his work influenced the
evolutionary theories of both Lamarck and Darwin.

Serious evolutionary thinking originated with the works of
\href{https://en.wikipedia.org/wiki/Jean-Baptiste_Lamarck}{Jean-Baptiste
Lamarck}, who was the first to present a coherent theory of evolution.
He posited that evolution was the result of environmental stress on
properties of animals, meaning that the more frequently and rigorously
an organ was used, the more complex and efficient it would become, thus
adapting the animal to its environment. Lamarck believed that these
acquired traits could then be passed on to the animal's offspring, who
would further develop and perfect them. However, it was the British
naturalist \href{https://en.wikipedia.org/wiki/Charles_Darwin}{Charles
Darwin}, combining the biogeographical approach of Humboldt, the
uniformitarian geology of Lyell, Malthus's writings on population
growth, and his own morphological expertise and extensive natural
observations, who forged a more successful evolutionary theory based on
natural selection; similar reasoning and evidence led
\href{https://en.wikipedia.org/wiki/Alfred_Russel_Wallace}{Alfred Russel
Wallace} to independently reach the same conclusions. Although it was
the subject of controversy (which continues to this day), Darwin's
theory quickly spread through the scientific community and soon became a
central axiom of the rapidly developing science of biology.

The discovery of the physical representation of heredity came along with
evolutionary principles and population genetics. In the 1940s and early
1950s, experiments pointed to DNA as the component of chromosomes that
held the trait-carrying units that had become known as genes. A focus on
new kinds of model organisms such as viruses and bacteria, along with
the discovery of the double-helical structure of DNA in 1953, marked the
transition to the era of molecular genetics. From the 1950s to the
present times, biology has been vastly extended in the molecular domain.
The genetic code was cracked by
\href{https://en.wikipedia.org/wiki/Har_Gobind_Khorana}{Har Gobind
Khorana}, \href{https://en.wikipedia.org/wiki/Robert_W._Holley}{Robert
W. Holley} and
\href{https://en.wikipedia.org/wiki/Marshall_Warren_Nirenberg}{Marshall
Warren Nirenberg} after DNA was understood to contain codons. Finally,
the Human Genome Project was launched in 1990 with the goal of mapping
the general human genome. This project was essentially completed in
2003, with further analysis still being published. The Human Genome
Project was the first step in a globalized effort to incorporate
accumulated knowledge of biology into a functional, molecular definition
of the human body and the bodies of other organisms.

\hypertarget{characteristics-of-life}{%
\subsection{Characteristics of Life}\label{characteristics-of-life}}

In the past, there have been many attempts to define what is meant by
``life'' through obsolete concepts such as odic force, hylomorphism,
spontaneous generation and vitalism, that have now been disproved by
biological discoveries. Aristotle is considered to be the first person
to classify organisms. Later, Carl Linnaeus introduced his system of
binomial nomenclature for the classification of species. Eventually new
groups and categories of life were discovered, such as cells and
microorganisms, forcing dramatic revisions of the structure of
relationships between living organisms. Though currently only known on
Earth, life need not be restricted to it, and many scientists speculate
in the existence of extraterrestrial life. Artificial life is a computer
simulation or human-made reconstruction of any aspect of life, which is
often used to examine systems related to natural life.

Death is the permanent termination of all biological functions which
sustain an organism, and as such, is the end of its life. Extinction is
the term describing the dying out of a group or taxon, usually a
species. Fossils are the preserved remains or traces of organisms.

The definition of life has long been a challenge for scientists and
philosophers, with many varied definitions put forward. This is
partially because life is a process, not a substance. This is
complicated by a lack of knowledge of the characteristics of living
entities, if any, that may have developed outside of Earth.
Philosophical definitions of life have also been put forward, with
similar difficulties on how to distinguish living things from the
non-living. Legal definitions of life have also been described and
debated, though these generally focus on the decision to declare a human
dead, and the legal ramifications of this decision.

Since there is no unequivocal definition of life, most current
definitions in biology are descriptive. Life is considered a
characteristic of something that preserves, furthers or reinforces its
existence in the given environment. This characteristic exhibits all or
most of the following traits:

\begin{itemize}
\tightlist
\item
  Homeostasis: regulation of the internal environment to maintain a
  constant state; for example, sweating to reduce temperature
\item
  Organization: being structurally composed of one or more cells -- the
  basic units of life
\item
  Metabolism: transformation of energy by converting chemicals and
  energy into cellular components (anabolism) and decomposing organic
  matter (catabolism). Living things require energy to maintain internal
  organization (homeostasis) and to produce the other phenomena
  associated with life.
\item
  Growth: maintenance of a higher rate of anabolism than catabolism. A
  growing organism increases in size in all of its parts, rather than
  simply accumulating matter.
\item
  Adaptation: the ability to change over time in response to the
  environment. This ability is fundamental to the process of evolution
  and is determined by the organism's heredity, diet, and external
  factors.
\item
  Response to stimuli: a response can take many forms, from the
  contraction of a unicellular organism to external chemicals, to
  complex reactions involving all the senses of multicellular organisms.
  A response is often expressed by motion; for example, the leaves of a
  plant turning toward the sun (phototropism), and chemotaxis.
\item
  Reproduction: the ability to produce new individual organisms, either
  asexually from a single parent organism or sexually from two parent
  organisms.
\end{itemize}

These complex processes, called physiological functions, have underlying
physical and chemical bases, as well as signaling and control mechanisms
that are essential to maintaining life.

More than 99\% of all species of life forms, amounting to over five
billion species, that ever lived on Earth are estimated to be extinct.

Although the number of Earth's catalogued species of lifeforms is
between 1.2 million and 2 million, the total number of species in the
planet is uncertain. Estimates range from 8 million to 100 million, with
a more narrow range between 10 and 14 million, but it may be as high as
1 trillion (with only one-thousandth of one percent of the species
described) according to studies realized in May 2016. The total number
of related DNA base pairs on Earth is estimated at 5.0 x
10\textsuperscript{37} and weighs 50 billion tonnes. In comparison, the
total mass of the biosphere has been estimated to be as much as 4 TtC
(trillion tons of carbon). In July 2016, scientists reported identifying
a set of 355 genes from the Last Universal Common Ancestor (LUCA) of all
organisms living on Earth.

\hypertarget{origin-of-life}{%
\subsection{Origin of Life}\label{origin-of-life}}

The Ancient Greeks believed that living things could spontaneously come
into being from nonliving matter, and that the goddess Gaia could make
life arise spontaneously from stones -- a process known as Generatio
spontanea. \href{https://en.wikipedia.org/wiki/Aristotle}{Aristotle}
disagreed, but he still believed that creatures could arise from
dissimilar organisms or from soil. Variations of this concept of
spontaneous generation still existed as late as the 17th century, but
towards the end of the 17th century, a series of observations and
arguments began that eventually discredited such ideas. This advance in
scientific understanding was met with much opposition, with personal
beliefs and individual prejudices often obscuring the facts.

\href{https://en.wikipedia.org/wiki/William_Harvey}{William Harvey}
(1578--1657) was an early proponent of all life beginning from an egg,
omne vivum ex ovo.
\href{https://en.wikipedia.org/wiki/Francesco_Redi}{Francesco Redi}, an
Italian physician, proved as early as 1668 that higher forms of life did
not originate spontaneously by demonstrating that maggots come from eggs
of flies. But proponents of spontaneous generation claimed that this did
not apply to microbes and continued to hold that these could arise
spontaneously. Attempts to disprove the spontaneous generation of life
from non-life continued in the early 19th century with observations and
experiments by Franz Schulze and
\href{https://en.wikipedia.org/wiki/Theodor_Schwann}{Theodor Schwann}.
In 1745, \href{https://en.wikipedia.org/wiki/John_Needham}{John Needham}
added chicken broth to a flask and boiled it. He then let it cool and
waited. Microbes grew, and he proposed it as an example of spontaneous
generation. In 1768,
\href{https://en.wikipedia.org/wiki/Lazzaro_Spallanzani}{Lazzaro
Spallanzani} repeated Needham's experiment but removed all the air from
the flask. No growth occurred. In 1854, Heinrich G. F. Schröder
(1810--1885) and Theodor von Dusch, and in 1859, Schröder alone,
repeated the Helmholtz filtration experiment and showed that living
particles can be removed from air by filtering it through cotton-wool.

In 1864, \href{https://en.wikipedia.org/wiki/Louis_Pasteur}{Louis
Pasteur} finally announced the results of his scientific experiments. In
a series of experiments similar to those performed earlier by Needham
and Spallanzani, Pasteur demonstrated that life does not arise in areas
that have not been contaminated by existing life. Pasteur's empirical
results were summarized in the phrase Omne vivum ex vivo, Latin for
``all life {[}is{]} from life''.

All known life forms share fundamental molecular mechanisms, reflecting
their common descent; based on these observations, hypotheses on the
origin of life attempt to find a mechanism explaining the formation of a
universal common ancestor, from simple organic molecules via
pre-cellular life to protocells and metabolism. Models have been divided
into ``genes-first'' and ``metabolism-first'' categories, but a recent
trend is the emergence of hybrid models that combine both categories.

Life on Earth is based on carbon and water. Carbon provides stable
frameworks for complex chemicals and can be easily extracted from the
environment, especially from carbon dioxide. There is no other chemical
element whose properties are similar enough to carbon's to be called an
analogue; silicon, the element directly below carbon on the periodic
table, does not form very many complex stable molecules, and because
most of its compounds are water-insoluble and because silicon dioxide is
a hard and abrasive solid in contrast to carbon dioxide at temperatures
associated with living things, it would be more difficult for organisms
to extract. The elements boron and phosphorus have more complex
chemistries, but suffer from other limitations relative to carbon. Water
is an excellent solvent and has two other useful properties: the fact
that ice floats enables aquatic organisms to survive beneath it in
winter; and its molecules have electrically negative and positive ends,
which enables it to form a wider range of compounds than other solvents
can. Other good solvents, such as ammonia, are liquid only at such low
temperatures that chemical reactions may be too slow to sustain life,
and lack water's other advantages. Organisms based on alternative
biochemistry may, however, be possible on other planets.

\href{https://en.wikipedia.org/wiki/Abiogenesis}{Abiogenesis}, or
informally the origin of life, is the natural process of life arising
from non-living matter, such as simple organic compounds. The prevailing
scientific hypothesis is that the transition from non-living to living
entities was not a single event, but a gradual process of increasing
complexity. Although the occurrence of abiogenesis is uncontroversial
among scientists, its possible mechanisms are poorly understood. There
are several principles and hypotheses for how abiogenesis could have
occurred. Life on Earth first appeared as early as 4.28 billion years
ago, soon after ocean formation 4.41 billion years ago, and not long
after the formation of the Earth 4.54 billion years ago. The earliest
known life forms are microfossils of bacteria.

There is no current scientific consensus as to how life originated.
However, many accepted scientific models build on the
\href{https://en.wikipedia.org/wiki/Miller–Urey_experiment}{Miller--Urey
experiment} and the work of
\href{https://en.wikipedia.org/wiki/Sidney_W._Fox}{Sidney Fox}, which
show that conditions on the primitive Earth favored chemical reactions
that synthesize amino acids and other organic compounds from inorganic
precursors, and phospholipids spontaneously form lipid bilayers, the
basic structure of a cell membrane.

The classic 1952 Miller--Urey experiment (Figure @ref(fig:millerurey))
demonstrated that most amino acids, the chemical constituents of the
proteins used in all living organisms, can be synthesized from inorganic
compounds under conditions intended to replicate those of the early
Earth. The experiment used water (H\textsubscript{2}O), methane
(CH\textsubscript{4}), ammonia (NH\textsubscript{3}), and hydrogen
(H\textsubscript{2}). The chemicals were all sealed inside a sterile
5-liter glass flask connected to a 500 ml flask half-full of water. The
water in the smaller flask was heated to induce evaporation, and the
water vapour was allowed to enter the larger flask. Continuous
electrical sparks were fired between the electrodes to simulate
lightning in the water vapour and gaseous mixture, and then the
simulated atmosphere was cooled again so that the water condensed and
trickled into a U-shaped trap at the bottom of the apparatus.

After a day, the solution collected at the trap had turned pink in
colour, and after a week of continuous operation the solution was deep
red and turbid. The boiling flask was then removed, and mercuric
chloride was added to prevent microbial contamination. The reaction was
stopped by adding barium hydroxide and sulfuric acid, and evaporated to
remove impurities. Using paper chromatography, Miller identified five
amino acids present in the solution: glycine, α-alanine and β-alanine
were positively identified, while aspartic acid and α-aminobutyric acid
(AABA) were less certain, due to the spots being faint.Complex organic
molecules occur in the Solar System and in interstellar space, and these
molecules may have provided starting material for the development of
life on Earth.

(ref:muex)
\href{https://commons.wikimedia.org/wiki/File:MUexperiment.png}{The
Miller--Urey experiment} was a chemical experiment that simulated the
conditions thought at the time (1952) to be present on the early Earth
and tested the chemical origin of life under those conditions. The
experiment at the time supported Alexander Oparin's and J. B. S.
Haldane's hypothesis that putative conditions on the primitive Earth
favoured chemical reactions that synthesized more complex organic
compounds from simpler inorganic precursors. Considered to be the
classic experiment investigating abiogenesis, it was conducted in 1952
by Stanley Miller, with assistance from Harold Urey, at the University
of Chicago and later the University of California, San Diego and
published the following year.

\texttt{\{r\ millerurey,\ fig.cap=\textquotesingle{}(ref:muex)\textquotesingle{},\ echo=FALSE,\ message=FALSE,\ warning=FALSE\}\ knitr::include\_graphics("./figures/life/Miller-Urey\_experiment-en.svg")}

Living organisms synthesize proteins, which are polymers of amino acids
using instructions encoded by deoxyribonucleic acid (DNA). Protein
synthesis entails intermediary ribonucleic acid (RNA) polymers. One
possibility for how life began is that genes originated first, followed
by proteins; the alternative being that proteins came first and then
genes.

However, because genes and proteins are both required to produce the
other, the problem of considering which came first is like that of the
chicken or the egg. Most scientists have adopted the hypothesis that
because of this, it is unlikely that genes and proteins arose
independently.

Therefore, a possibility, first suggested by
\href{https://en.wikipedia.org/wiki/Francis_Crick}{Francis Crick}, is
that the first life was based on RNA, which has the DNA-like properties
of information storage and the catalytic properties of some proteins.
This is called the RNA world hypothesis, and it is supported by the
observation that many of the most critical components of cells (those
that evolve the slowest) are composed mostly or entirely of RNA. Also,
many critical cofactors (ATP, Acetyl-CoA, NADH, etc.) are either
nucleotides or substances clearly related to them. The catalytic
properties of RNA had not yet been demonstrated when the hypothesis was
first proposed, but they were confirmed by
\href{https://en.wikipedia.org/wiki/Thomas_Cech}{Thomas Cech} in 1986.

One issue with the RNA world hypothesis is that synthesis of RNA from
simple inorganic precursors is more difficult than for other organic
molecules. One reason for this is that RNA precursors are very stable
and react with each other very slowly under ambient conditions, and it
has also been proposed that living organisms consisted of other
molecules before RNA. However, the successful synthesis of certain RNA
molecules under the conditions that existed prior to life on Earth has
been achieved by adding alternative precursors in a specified order with
the precursor phosphate present throughout the reaction. This study
makes the RNA world hypothesis more plausible.

Geological findings in 2013 showed that reactive phosphorus species
(like phosphite) were in abundance in the ocean before 3.5 Ga, and that
Schreibersite easily reacts with aqueous glycerol to generate phosphite
and glycerol 3-phosphate. It is hypothesized that
Schreibersite-containing meteorites from the Late Heavy Bombardment
could have provided early reduced phosphorus, which could react with
prebiotic organic molecules to form phosphorylated biomolecules, like
RNA.

In 2009, experiments demonstrated Darwinian evolution of a two-component
system of RNA enzymes (ribozymes) in vitro. The work was performed in
the laboratory of Gerald Joyce, who stated ``This is the first example,
outside of biology, of evolutionary adaptation in a molecular genetic
system.''

Prebiotic compounds may have originated extraterrestrially. NASA
findings in 2011, based on studies with meteorites found on Earth,
suggest DNA and RNA components (adenine, guanine and related organic
molecules) may be formed in outer space.

In March 2015, NASA scientists reported that, for the first time,
complex DNA and RNA organic compounds of life, including uracil,
cytosine and thymine, have been formed in the laboratory under outer
space conditions, using starting chemicals, such as pyrimidine, found in
meteorites. Pyrimidine, like polycyclic aromatic hydrocarbons (PAHs),
the most carbon-rich chemical found in the universe, may have been
formed in red giants or in interstellar dust and gas clouds, according
to the scientists.

According to the panspermia hypothesis, microscopic life---distributed
by meteoroids, asteroids and other small Solar System bodies---may exist
throughout the universe.

Since its primordial beginnings, life on Earth has changed its
environment on a geologic time scale, but it has also adapted to survive
in most ecosystems and conditions. Some microorganisms, called
extremophiles, thrive in physically or geochemically extreme
environments that are detrimental to most other life on Earth. The cell
is considered the structural and functional unit of life. There are two
kinds of cells, prokaryotic and eukaryotic, both of which consist of
cytoplasm enclosed within a membrane and contain many biomolecules such
as proteins and nucleic acids. Cells reproduce through a process of cell
division, in which the parent cell divides into two or more daughter
cells.

(ref:celltypes)
\href{https://commons.wikimedia.org/wiki/File:Celltypes.svg}{Cartoons of
a eukaryotic and prokaryotic cell.}

\texttt{\{r\ celltypecartoon,\ fig.cap=\textquotesingle{}(ref:celltypes)\textquotesingle{},\ echo=FALSE,\ message=FALSE,\ warning=FALSE\}\ knitr::include\_graphics("./figures/life/Celltypes.svg")}

\hypertarget{foundations-of-modern-biology}{%
\subsection{Foundations of Modern
Biology}\label{foundations-of-modern-biology}}

\hypertarget{cell-theory}{%
\subsubsection{Cell theory}\label{cell-theory}}

\href{https://en.wikipedia.org/wiki/Cell_theory}{Cell theory} states
that the cell is the fundamental unit of life, that all living things
are composed of one or more cells, and that all cells arise from
pre-existing cells through cell division. In multicellular organisms,
every cell in the organism's body derives ultimately from a single cell
in a fertilized egg. The cell is also considered to be the basic unit in
many pathological processes. In addition, the phenomenon of energy flow
occurs in cells in processes that are part of the function known as
metabolism. Finally, cells contain hereditary information (DNA), which
is passed from cell to cell during cell division. Research into the
origin of life, abiogenesis, amounts to an attempt to discover the
origin of the first cells.

Cells are the basic unit of structure in every living thing, and all
cells arise from pre-existing cells by division. Cell theory was
formulated by \href{https://en.wikipedia.org/wiki/Henri_Dutrochet}{Henri
Dutrochet}, \href{https://en.wikipedia.org/wiki/Theodor_Schwann}{Theodor
Schwann},
\href{https://en.wikipedia.org/wiki/Matthias_Jakob_Schleiden}{Matthias
Jakob Schleiden},
\href{https://en.wikipedia.org/wiki/Rudolf_Virchow}{Rudolf Virchow} and
others during the early nineteenth century, and subsequently became
widely accepted. The activity of an organism depends on the total
activity of its cells, with energy flow occurring within and between
them. Cells contain hereditary information that is carried forward as a
genetic code during cell division.

There are two primary types of cells.
\href{https://en.wikipedia.org/wiki/Prokaryote}{Prokaryotes} lack a
nucleus and other membrane-bound organelles, although they have circular
DNA and ribosomes.
\href{https://en.wikipedia.org/wiki/Bacteria}{Bacteria} and
\href{https://en.wikipedia.org/wiki/Archaea}{Archaea} are two domains of
prokaryotes. The other primary type of cells are the
\href{https://en.wikipedia.org/wiki/Eukaryote}{eukaryotes}, which have
distinct nuclei bound by a nuclear membrane and membrane-bound
organelles, including mitochondria, chloroplasts, lysosomes, rough and
smooth endoplasmic reticulum, and vacuoles. In addition, they possess
organized chromosomes that store genetic material. All species of large
complex organisms are eukaryotes, including animals, plants and fungi,
though most species of eukaryote are protist microorganisms. The
conventional model is that eukaryotes evolved from prokaryotes, with the
main organelles of the eukaryotes forming through endosymbiosis between
bacteria and the progenitor eukaryotic cell.

(ref:livorg)
\href{https://commons.wikimedia.org/wiki/File:Tree_of_Living_Organisms_2.png}{Tree
diagram} illustrating the evolutionary relationship of living
organisms{]}

\texttt{\{r\ livorgtree,\ fig.cap=\textquotesingle{}(ref:livorg)\textquotesingle{},\ echo=FALSE,\ message=FALSE,\ warning=FALSE\}\ knitr::include\_graphics("./figures/life/Tree\_of\_Living\_Organisms\_2.png")}

A \href{https://en.wikipedia.org/wiki/Virus}{virus} is a submicroscopic
infectious agent that replicates only inside the living cells of an
organism. Scientific opinions differ on whether viruses are a form of
life, or organic structures that interact with living organisms. They
have been described as ``organisms at the edge of life'', since they
resemble organisms in that they possess genes, evolve by natural
selection, and reproduce by creating multiple copies of themselves
through self-assembly. Although they have genes, they do not have a
cellular structure, which is seen as the basic unit of life. Viruses do
not have their own metabolism, and require a host cell to make new
products. They therefore cannot naturally reproduce outside a host
cell---although bacterial species such as rickettsia and chlamydia are
considered living organisms despite the same limitation. Accepted forms
of life use cell division to reproduce, whereas viruses spontaneously
assemble within cells. They differ from autonomous growth of crystals as
they inherit genetic mutations while being subject to natural selection.
Virus self-assembly within host cells has implications for the study of
the origin of life, as it lends further credence to the hypothesis that
life could have started as self-assembling organic molecules.

The molecular mechanisms of cell biology are based on proteins which are
synthesized by the ribosomes through an enzyme-catalyzed process called
protein biosynthesis. A sequence of amino acids is assembled and joined
together based upon gene expression of the cell's nucleic acid. In
eukaryotic cells, these proteins may then be transported and processed
through the Golgi apparatus in preparation for dispatch to their
destination.

Cells reproduce through a process of cell division in which the parent
cell divides into two or more daughter cells. For prokaryotes, cell
division occurs through a process of fission in which the DNA is
replicated, then the two copies are attached to parts of the cell
membrane. In eukaryotes, a more complex process of mitosis is followed.
However, the end result is the same; the resulting cell copies are
identical to each other and to the original cell (except for mutations),
and both are capable of further division following an interphase period.

Multicellular organisms may have first evolved through the formation of
colonies of identical cells. These cells can form group organisms
through cell adhesion. The individual members of a colony are capable of
surviving on their own, whereas the members of a true multi-cellular
organism have developed specializations, making them dependent on the
remainder of the organism for survival. Such organisms are formed
clonally or from a single germ cell that is capable of forming the
various specialized cells that form the adult organism. This
specialization allows multicellular organisms to exploit resources more
efficiently than single cells. In January 2016, scientists reported
that, about 800 million years ago, a minor genetic change in a single
molecule, called GK-PID, may have allowed organisms to go from a single
cell organism to one of many cells.

Cells have evolved methods to perceive and respond to their
microenvironment, thereby enhancing their adaptability. Cell signaling
coordinates cellular activities, and hence governs the basic functions
of multicellular organisms. Signaling between cells can occur through
direct cell contact using juxtacrine signalling, or indirectly through
the exchange of agents as in the endocrine system. In more complex
organisms, coordination of activities can occur through a dedicated
nervous system.

\hypertarget{the-theory-of-evolution}{%
\subsubsection{The Theory of Evolution}\label{the-theory-of-evolution}}

A central organizing concept in biology is that life changes and
develops through
\href{https://en.wikipedia.org/wiki/Evolution}{evolution}, and that all
life-forms known have a common origin. The theory of evolution
postulates that all organisms on the Earth, both living and extinct,
have descended from a common ancestor or an ancestral gene pool. This
universal common ancestor of all organisms is believed to have appeared
about 3.5 billion years ago. Biologists regard the ubiquity of the
genetic code as definitive evidence in favor of the theory of universal
common descent for all bacteria, archaea, and eukaryotes.

The term ``evolution'' was introduced into the scientific lexicon by
\href{https://en.wikipedia.org/wiki/Jean-Baptiste_Lamarck}{Jean-Baptiste
de Lamarck} in 1809, and fifty years later
\href{https://en.wikipedia.org/wiki/Charles_Darwin}{Charles Darwin}
posited a scientific model of natural selection as evolution's driving
force. \href{https://en.wikipedia.org/wiki/Alfred_Russel_Wallace}{Alfred
Russel Wallace} independently reached the same conclusions and is
recognized as the co-discoverer of this concept. Evolution is now used
to explain the great variations of life found on Earth.

Darwin theorized that species flourish or die when subjected to the
processes of natural selection or selective breeding. Genetic drift was
embraced as an additional mechanism of evolutionary development in the
modern synthesis of the theory.

The evolutionary history of the species---which describes the
characteristics of the various species from which it
descended---together with its genealogical relationship to every other
species is known as its phylogeny. Widely varied approaches to biology
generate information about phylogeny. These include the comparisons of
DNA sequences, a product of molecular biology (more particularly
genomics), and comparisons of fossils or other records of ancient
organisms, a product of paleontology. Biologists organize and analyze
evolutionary relationships through various methods, including
phylogenetics, phenetics, and cladistics

Evolution is relevant to the understanding of the natural history of
life forms and to the understanding of the organization of current life
forms. But, those organizations can only be understood in light of how
they came to be by way of the process of evolution. Consequently,
evolution is central to all fields of biology.

The evolutionary history of life on Earth traces the processes by which
living and fossil organisms evolved, from the earliest emergence of life
to the present. Earth formed about 4.5 billion years (Ga) ago and
evidence suggests life emerged prior to 3.7 Ga. (Although there is some
evidence of life as early as 4.1 to 4.28 Ga, it remains controversial
due to the possible non-biological formation of the purported fossils.)
The similarities among all known present-day species indicate that they
have diverged through the process of evolution from a common ancestor.
Approximately 1 trillion species currently live on Earth of which only
1.75--1.8 million have been named and 1.6 million documented in a
central database. These currently living species represent less than one
percent of all species that have ever lived on earth.

The earliest evidence of life comes from biogenic carbon signatures and
stromatolite fossils discovered in 3.7 billion-year-old metasedimentary
rocks from western Greenland. In 2015, possible ``remains of biotic
life'' were found in 4.1 billion-year-old rocks in Western Australia. In
March 2017, putative evidence of possibly the oldest forms of life on
Earth was reported in the form of fossilized microorganisms discovered
in hydrothermal vent precipitates in the Nuvvuagittuq Belt of Quebec,
Canada, that may have lived as early as 4.28 billion years ago, not long
after the oceans formed 4.4 billion years ago, and not long after the
formation of the Earth 4.54 billion years ago.

(ref:stromato)
\href{https://commons.wikimedia.org/wiki/File:Stromatolites_in_Sharkbay.jpg}{Modern
stromatolites in Shark Bay, Western Australia}

\texttt{\{r\ stromatolites,\ fig.cap=\textquotesingle{}(ref:stromato)\textquotesingle{},\ echo=FALSE,\ message=FALSE,\ warning=FALSE\}\ knitr::include\_graphics("./figures/life/Stromatolites\_in\_Sharkbay.jpg")}

Microbial mats of coexisting bacteria and archaea were the dominant form
of life in the early Archean Epoch and many of the major steps in early
evolution are thought to have taken place in this environment. The
evolution of photosynthesis, around 3.5 Ga, eventually led to a buildup
of its waste product, oxygen, in the atmosphere, leading to the great
oxygenation event, beginning around 2.4 Ga. The earliest evidence of
eukaryotes (complex cells with organelles) dates from 1.85 Ga, and while
they may have been present earlier, their diversification accelerated
when they started using oxygen in their metabolism. Later, around 1.7
Ga, multicellular organisms began to appear, with differentiated cells
performing specialised functions. Sexual reproduction, which involves
the fusion of male and female reproductive cells (gametes) to create a
zygote in a process called fertilization is, in contrast to asexual
reproduction, the primary method of reproduction for the vast majority
of macroscopic organisms, including almost all eukaryotes (which
includes animals and plants). However the origin and evolution of sexual
reproduction remain a puzzle for biologists though it did evolve from a
common ancestor that was a single celled eukaryotic species. Bilateria,
animals having a left and a right side that are mirror images of each
other, appeared by 555 Ma (million years ago).

The earliest plants on land date back to around 850 million years ago
(Ma), from carbon isotopes in Precambrian rocks, while algae-like
multicellular land plants are dated back even to about 1 billion years
ago, although evidence suggests that microorganisms formed the earliest
terrestrial ecosystems, at least 2.7 billion years ago (Ga).
Microorganisms are thought to have paved the way for the inception of
land plants in the Ordovician. Land plants were so successful that they
are thought to have contributed to the Late Devonian extinction event.
(The long causal chain implied seems to involve the success of early
tree archaeopteris (1) drew down CO\textsubscript{2} levels, leading to
global cooling and lowered sea levels, (2) roots of archeopteris
fostered soil development which increased rock weathering, and the
subsequent nutrient run-off may have triggered algal blooms resulting in
anoxic events which caused marine-life die-offs. Marine species were the
primary victims of the Late Devonian extinction.)

Ediacara biota appear during the Ediacaran period, while vertebrates,
along with most other modern phyla originated about 525 Ma during the
Cambrian explosion. During the Permian period, synapsids, including the
ancestors of mammals, dominated the land, but most of this group became
extinct in the Permian--Triassic extinction event 252 Ma. During the
recovery from this catastrophe, archosaurs became the most abundant land
vertebrates; one archosaur group, the dinosaurs, dominated the Jurassic
and Cretaceous periods. After the Cretaceous--Paleogene extinction event
65 Ma killed off the non-avian dinosaurs, mammals increased rapidly in
size and diversity. Such mass extinctions may have accelerated evolution
by providing opportunities for new groups of organisms to diversify.

\hypertarget{genetics}{%
\subsubsection{Genetics}\label{genetics}}

\href{https://en.wikipedia.org/wiki/Gene}{Genes} are the primary units
of inheritance in all organisms. A gene is a unit of heredity and
corresponds to a region of DNA that influences the form or function of
an organism in specific ways. All organisms, from bacteria to animals,
share the same basic machinery that copies and translates DNA into
proteins. Cells transcribe a DNA gene into an RNA version of the gene,
and a ribosome then translates the RNA into a sequence of amino acids
known as a protein. The translation code from RNA codon to amino acid is
the same for most organisms. For example, a sequence of DNA that codes
for insulin in humans also codes for insulin when inserted into other
organisms, such as plants.

\href{https://en.wikipedia.org/wiki/DNA}{DNA} is found as linear
chromosomes in eukaryotes, and circular chromosomes in prokaryotes. A
chromosome is an organized structure consisting of DNA and histones. The
set of chromosomes in a cell and any other hereditary information found
in the mitochondria, chloroplasts, or other locations is collectively
known as a cell's genome. In eukaryotes, genomic DNA is localized in the
cell nucleus, or with small amounts in mitochondria and chloroplasts. In
prokaryotes, the DNA is held within an irregularly shaped body in the
cytoplasm called the nucleoid. The genetic information in a genome is
held within genes, and the complete assemblage of this information in an
organism is called its genotype.

\hypertarget{homeostasis}{%
\subsubsection{Homeostasis}\label{homeostasis}}

\href{https://en.wikipedia.org/wiki/Homeostasis}{Homeostasis} is the
ability of an open system to regulate its internal environment to
maintain stable conditions by means of multiple dynamic equilibrium
adjustments that are controlled by interrelated regulation mechanisms.
All living organisms, whether unicellular or multicellular, exhibit
homeostasis.

To maintain dynamic equilibrium and effectively carry out certain
functions, a system must detect and respond to perturbations. After the
detection of a perturbation, a biological system normally responds
through negative feedback that stabilize conditions by reducing or
increasing the activity of an organ or system. One example is the
release of glucagon when sugar levels are too low.

\hypertarget{energy}{%
\subsubsection{Energy}\label{energy}}

The survival of a living organism depends on the continuous input of
\href{https://en.wikipedia.org/wiki/Energy}{energy}. Chemical reactions
that are responsible for its structure and function are tuned to extract
energy from substances that act as its food and transform them to help
form new cells and sustain them. In this process, molecules of chemical
substances that constitute food play two roles; first, they contain
energy that can be transformed and reused in that organism's biological,
chemical reactions; second, food can be transformed into new molecular
structures (biomolecules) that are of use to that organism.

The organisms responsible for the introduction of energy into an
ecosystem are known as producers or autotrophs. Nearly all such
organisms originally draw their energy from the sun. Plants and other
phototrophs use solar energy via a process known as photosynthesis to
convert raw materials into organic molecules, such as ATP, whose bonds
can be broken to release energy. A few ecosystems, however, depend
entirely on energy extracted by chemotrophs from methane, sulfides, or
other non-luminal energy sources.

Some of the energy thus captured produces biomass and energy that is
available for growth and development of other life forms. The majority
of the rest of this biomass and energy are lost as waste molecules and
heat. The most important processes for converting the energy trapped in
chemical substances into energy useful to sustain life are metabolism
and cellular respiration.

\hypertarget{areas-and-levels-of-study-and-research-in-biology}{%
\subsection{Areas And Levels of Study And Research in
Biology}\label{areas-and-levels-of-study-and-research-in-biology}}

\href{https://en.wikipedia.org/wiki/Molecular_biology}{Molecular
biology} is the study of biology at the molecular level. This field
overlaps with other areas of biology, particularly those of genetics and
biochemistry. Molecular biology is a study of the interactions of the
various systems within a cell, including the interrelationships of DNA,
RNA, and protein synthesis and how those interactions are regulated.

The next larger scale,
\href{https://en.wikipedia.org/wiki/Cell_biology}{cell biology}, studies
the structural and physiological properties of cells, including their
internal behavior, interactions with other cells, and with their
environment. This is done on both the microscopic and molecular levels,
for unicellular organisms such as bacteria, as well as the specialized
cells of multicellular organisms such as humans. Understanding the
structure and function of cells is fundamental to all of the biological
sciences. The similarities and differences between cell types are
particularly relevant to molecular biology.

\href{https://en.wikipedia.org/wiki/Anatomy}{Anatomy} is a treatment of
the macroscopic forms of such structures organs and organ systems.

\href{https://en.wikipedia.org/wiki/Genetics}{Genetics} is the science
of genes, heredity, and the variation of organisms. Genes encode the
information needed by cells for the synthesis of proteins, which in turn
play a central role in influencing the final phenotype of the organism.
Genetics provides research tools used in the investigation of the
function of a particular gene, or the analysis of genetic interactions.
Within organisms, genetic information is physically represented as
chromosomes, within which it is represented by a particular sequence of
amino acids in particular DNA molecules.

\href{https://en.wikipedia.org/wiki/Developmental_biology}{Developmental
biology} studies the process by which organisms grow and develop.
Developmental biology, originated from embryology, studies the genetic
control of cell growth, cellular differentiation, and ``cellular
morphogenesis,'' which is the process that progressively gives rise to
tissues, organs, and anatomy. Model organisms for developmental biology
include the round worm Caenorhabditis elegans, the fruit fly Drosophila
melanogaster, the zebrafish Danio rerio, the mouse Mus musculus, and the
weed Arabidopsis thaliana. (A model organism is a species that is
extensively studied to understand particular biological phenomena, with
the expectation that discoveries made in that organism provide insight
into the workings of other organisms.)

\href{https://en.wikipedia.org/wiki/Physiology}{Physiology} is the study
of the mechanical, physical, and biochemical processes of living
organisms function as a whole. The theme of ``structure to function'' is
central to biology. Physiological studies have traditionally been
divided into plant physiology and animal physiology, but some principles
of physiology are universal, no matter what particular organism is being
studied. For example, what is learned about the physiology of yeast
cells can also apply to human cells. The field of animal physiology
extends the tools and methods of human physiology to non-human species.
Plant physiology borrows techniques from both research fields.

\href{https://en.wikipedia.org/wiki/Evolutionary_biology}{Evolutionary
biology} is concerned with the origin and descent of species, and their
change over time. It employs scientists from many taxonomically oriented
disciplines, for example, those with special training in particular
organisms such as mammalogy, ornithology, botany, or herpetology, but
are of use in answering more general questions about evolution.

Evolutionary biology is partly based on
\href{https://en.wikipedia.org/wiki/Paleontology}{paleontology}, which
uses the fossil record to answer questions about the mode and tempo of
evolution, and partly on the developments in areas such as population
genetics. In the 1980s, developmental biology re-entered evolutionary
biology after its initial exclusion from the modern synthesis through
the study of evolutionary developmental biology. Phylogenetics,
systematics, and taxonomy are related fields often considered part of
evolutionary biology.

Multiple speciation events create a tree structured system of
relationships between species. The role of
\href{https://en.wikipedia.org/wiki/Systematics}{systematics} is to
study these relationships and thus the differences and similarities
between species and groups of species. However, systematics was an
active field of research long before evolutionary thinking was common.

Traditionally, living things have been divided into five kingdoms:
Monera; Protista; Fungi; Plantae; Animalia. However, many scientists now
consider this five-kingdom system outdated. Modern alternative
classification systems generally begin with the three-domain system:
Archaea (originally Archaebacteria); Bacteria (originally Eubacteria)
and Eukaryota (including protists, fungi, plants, and animals). These
domains reflect whether the cells have nuclei or not, as well as
differences in the chemical composition of key biomolecules such as
ribosomes.

(ref:eukardiv)
\href{https://commons.wikimedia.org/wiki/File:Eukaryota_diversity_2.jpg}{Eukaryotes
and some examples of their diversity} -- clockwise from top left: Red
mason bee, Boletus edulis, chimpanzee, Isotricha intestinalis,
Ranunculus asiaticus, and Volvox carteri.

\texttt{\{r\ eukaryotediversity,\ fig.cap=\textquotesingle{}(ref:eukardiv)\textquotesingle{},\ echo=FALSE,\ message=FALSE,\ warning=FALSE\}\ knitr::include\_graphics("./figures/life/Eukaryota\_diversity\_2.jpg")}

Further, each kingdom is broken down recursively until each species is
separately classified. The order is: Domain; Kingdom; Phylum; Class;
Order; Family; Genus; Species.

(ref:hierarchy)
\href{https://commons.wikimedia.org/wiki/File:Biological_classification_L_Pengo.svg}{The
hierarchy of biological classification's eight major taxonomic ranks
from the most specific (top) to the most general (bottom). Intermediate
minor rankings are not shown. This diagram uses a 3 Domains / 6 Kingdoms
format}

\texttt{\{r\ hierarchycartoon,\ fig.cap=\textquotesingle{}(ref:hierarchy)\textquotesingle{},\ echo=FALSE,\ message=FALSE,\ warning=FALSE\}\ knitr::include\_graphics("./figures/life/Biological\_classification\_L\_Pengo.svg")}

Outside of these categories, there are obligate intracellular parasites
that are ``on the edge of life'' in terms of metabolic activity, meaning
that many scientists do not actually classify such structures as alive,
due to their lack of at least one or more of the fundamental functions
or characteristics that define life. They are classified as viruses,
viroids, prions, or satellites.

The scientific name of an organism is generated from its genus and
species. For example, humans are listed as Homo sapiens. Homo is the
genus, and sapiens the species. When writing the scientific name of an
organism, it is proper to capitalize the first letter in the genus and
put all of the species in lowercase. Additionally, the entire term may
be italicized or underlined.

The dominant classification system is called the
\href{https://en.wikipedia.org/wiki/Linnaean_taxonomy}{Linnaean
taxonomy}. It includes ranks and binomial nomenclature. How organisms
are named is governed by international agreements such as the
International Code of Nomenclature for algae, fungi, and plants (ICN),
the International Code of Zoological Nomenclature (ICZN), and the
International Code of Nomenclature of Bacteria (ICNB). The
classification of viruses, viroids, prions, and all other sub-viral
agents that demonstrate biological characteristics is conducted by the
International Committee on Taxonomy of Viruses (ICTV) and is known as
the International Code of Viral Classification and Nomenclature (ICVCN).

\href{https://en.wikipedia.org/wiki/Ecology}{Ecology} is the study of
the distribution and abundance of living organisms, the interaction
between them and their environment. An organism shares an environment
that includes other organisms and biotic factors as well as local
abiotic factors (non-living) such as climate and ecology. One reason
that biological systems can be difficult to study is that so many
different interactions with other organisms and the environment are
possible, even on small scales. A microscopic bacterium responding to a
local sugar gradient is responding to its environment as much as a lion
searching for food in the African savanna. For any species, behaviors
can be co-operative, competitive, parasitic, or symbiotic. Matters
become more complex when two or more species interact in an ecosystem.

Ecological systems are studied at several different levels, from the
scale of the ecology of individual organisms, to those of populations,
to the ecosystems and finally the biosphere. The term population biology
is often used interchangeably with population ecology, although
population biology is more frequently used in the case of diseases,
viruses, and microbes, while the term population ecology is more
commonly applied to the study of plants and animals. Ecology draws on
many subdisciplines.

\href{https://en.wikipedia.org/wiki/Ethology}{Ethology} is the study of
animal behavior (particularly that of social animals such as primates
and canids), and is sometimes considered a branch of zoology.
Ethologists have been particularly concerned with the evolution of
behavior and the understanding of behavior in terms of the theory of
natural selection. In one sense, the first modern ethologist was Charles
Darwin, whose book, The Expression of the Emotions in Man and Animals,
influenced many ethologists to come.

(ref:ethol)
\href{https://commons.wikimedia.org/wiki/File:Ethology_diversity_2.jpg}{A
range of animal behaviours.}

\texttt{\{r\ ethology,\ fig.cap=\textquotesingle{}(ref:ethol)\textquotesingle{},\ echo=FALSE,\ message=FALSE,\ warning=FALSE\}\ knitr::include\_graphics("./figures/life/Ethology\_diversity\_2.jpg")}

Biogeography studies the spatial distribution of organisms on the Earth,
focusing on such topics as plate tectonics, climate change, dispersal
and migration, and cladistics.

\hypertarget{science}{%
\subsection{Science}\label{science}}

\href{https://en.wikipedia.org/wiki/Science}{Science} (from the Latin
word scientia, meaning ``knowledge'') is a systematic enterprise that
builds and organizes knowledge in the form of testable explanations and
predictions about the universe.

Scientists are individuals who conduct scientific research to advance
knowledge in an area of interest. In classical antiquity, there was no
real ancient analog of a modern scientist. Instead, philosophers engaged
in the philosophical study of nature called natural philosophy, a
precursor of natural science. It was not until the 19th century that the
term scientist came into regular use after it was coined by the
theologian, philosopher, and historian of science William Whewell in
1833. In modern times, many professional scientists are trained in an
academic setting and upon completion, attain an academic degree, with
the highest degree being a doctorate such as a Doctor of Philosophy
(PhD). Many scientists pursue careers in various sectors of the economy
such as academia, industry, government, and nonprofit organizations.

(ref:scientist)
\href{https://commons.wikimedia.org/wiki/File:Researcher_looking_through_microscope.jpg}{Two
scientists working at the National Cancer Institute which is part of the
National Institutes of Health of the United States of America.}

\texttt{\{r\ twoscientists,\ fig.cap=\textquotesingle{}(ref:scientist)\textquotesingle{},\ echo=FALSE,\ message=FALSE,\ warning=FALSE\}\ knitr::include\_graphics("./figures/life/Researcher\_looking\_through\_microscope.jpg")}

The roles of ``scientists'', and their predecessors before the emergence
of modern scientific disciplines, have evolved considerably over time.
Scientists of different eras (and before them, natural philosophers and
others who contributed to the development of science) have had widely
different places in society, and the social norms, ethical values, and
epistemic virtues associated with scientists---and expected of
them---have changed over time as well.

Some historians point to the Scientific Revolution that began in 16th
century as the period when science in a recognizably modern form
developed. It wasn't until the 19th century that sufficient
socioeconomic changes occurred for scientists to emerge as a major
profession.

The presence of women in science spans the earliest times of the history
of science wherein they have made significant contributions. Historians
with an interest in gender and science have researched the scientific
endeavors and accomplishments of women, the barriers they have faced,
and the strategies implemented to have their work peer-reviewed and
accepted in major scientific journals and other publications.

The involvement of women in the field of medicine occurred in several
early civilizations, and the study of natural philosophy in ancient
Greece was open to women. Women contributed to the proto-science of
alchemy in the first or second centuries AD. During the Middle Ages,
religious convents were an important place of education for women, and
some of these communities provided opportunities for women to contribute
to scholarly research. The 11th century saw the emergence of the first
universities; women were, for the most part, excluded from university
education. Outside academia, botany was the science that benefitted most
from contributions of women in early modern times. Gender roles were
largely deterministic in the eighteenth century and women made
substantial advances in science. During the nineteenth century, women
were excluded from most formal scientific education, but they began to
be admitted into learned societies during this period. In the later
nineteenth century, the rise of the women's college provided jobs for
women scientists and opportunities for education.
\href{https://en.wikipedia.org/wiki/Marie_Curie}{Marie Curie}, a
physicist and chemist who conducted pioneering research on radioactive
decay, was the first woman to receive a Nobel Prize in Physics and
became the first person to receive a second Nobel Prize in Chemistry.
Forty women have been awarded the Nobel Prize between 1901 and 2010.
Seventeen women have been awarded the Nobel Prize in physics, chemistry,
physiology or medicine.

(ref:mc)
\href{https://commons.wikimedia.org/wiki/File:1911_Solvay_conference.jpg}{At
First Solvay Conference (1911), Marie Curie (seated, second from right)
confers with Henri Poincaré; standing, fourth from right, is Rutherford;
second from right, Einstein; far right, Paul Langevin}

\texttt{\{r\ mariecuriesolvay,\ fig.cap=\textquotesingle{}(ref:mc)\textquotesingle{},\ echo=FALSE,\ message=FALSE,\ warning=FALSE\}\ knitr::include\_graphics("./figures/life/1911\_Solvay\_conference.jpg")}

\href{https://en.wikipedia.org/wiki/List_of_African-American_inventors_and_scientists}{A
list of African Americans inventors and scientists} documents many of
the African-Americans who have invented a multitude of items or made
discoveries in the course of their lives. These have ranged from
practical everyday devices to applications and scientific discoveries in
diverse fields, including physics, biology, math, plus the medical,
nuclear and space science.

(ref:gwc)
\href{https://en.wikipedia.org/wiki/George_Washington_Carver}{George
Washington Carver} (1860s -- January 5, 1943) was an American
agricultural scientist and inventor who promoted alternative crops to
cotton and methods to prevent soil depletion. He was the most prominent
black scientist of the early 20th century. Apart from his work to
improve the lives of farmers, Carver was also a leader in promoting
environmentalism. He received numerous honors for his work, including
the Spingarn Medal of the NAACP. In an era of high racial polarization,
his fame reached beyond the black community. He was widely recognized
and praised in the white community for his many achievements and
talents. In 1941, Time magazine dubbed Carver a ``Black Leonardo''.

\texttt{\{r\ gwcarter,\ fig.cap=\textquotesingle{}(ref:gwc)\textquotesingle{},\ echo=FALSE,\ message=FALSE,\ warning=FALSE\}\ knitr::include\_graphics("./figures/life/George\_Washington\_Carver\_c1910\_-\_Restoration.jpg")}

Scientists exhibit a strong curiosity about reality, with some
scientists having a desire to apply scientific knowledge for the benefit
of health, nations, environment, or industries. Other motivations
include recognition by their peers and prestige. The Nobel Prize, a
widely regarded prestigious award, is awarded annually to those who have
achieved scientific advances in the fields of medicine, physics,
chemistry, and economics.

The earliest roots of science can be traced to Ancient Egypt and
Mesopotamia in around 3500 to 3000 BCE. Their contributions to
mathematics, astronomy, and medicine entered and shaped Greek natural
philosophy of classical antiquity, whereby formal attempts were made to
provide explanations of events in the physical world based on natural
causes. After the fall of the Western Roman Empire, knowledge of Greek
conceptions of the world deteriorated in Western Europe during the early
centuries (400 to 1000 CE) of the Middle Ages but was preserved in the
Muslim world during the Islamic Golden Age. The recovery and
assimilation of Greek works and Islamic inquiries into Western Europe
from the 10th to 13th century revived ``natural philosophy'', which was
later transformed by the Scientific Revolution that began in the 16th
century as new ideas and discoveries departed from previous Greek
conceptions and traditions. The scientific method soon played a greater
role in knowledge creation and it was not until the 19th century that
many of the institutional and professional features of science began to
take shape; along with the changing of ``natural philosophy'' to
``natural science.''

Modern science is typically divided into three major branches that
consist of the natural sciences (e.g., biology, chemistry, and physics),
which study nature in the broadest sense; the social sciences (e.g.,
economics, psychology, and sociology), which study individuals and
societies; and the formal sciences (e.g., logic, mathematics, and
theoretical computer science), which study abstract concepts. There is
disagreement, however, on whether the formal sciences actually
constitute a science as they do not rely on empirical evidence.
Disciplines that use existing scientific knowledge for practical
purposes, such as engineering and medicine, are described as applied
sciences.

Science is based on research, which is commonly conducted in academic
and research institutions as well as in government agencies and
companies. The practical impact of scientific research has led to the
emergence of science policies that seek to influence the scientific
enterprise by prioritizing the development of commercial products,
armaments, health care, and environmental protection.

Natural science is concerned with the description, prediction, and
understanding of natural phenomena based on empirical evidence from
observation and experimentation.

\hypertarget{scientific-terminology}{%
\subsubsection{Scientific Terminology}\label{scientific-terminology}}

\href{https://en.wikipedia.org/wiki/Scientific_terminology}{Scientific
terminology} is the part of the language that is used by scientists in
the context of their professional activities. While studying nature,
scientists often encounter or create new material or immaterial objects
and concepts and are compelled to name them.

In modern science and its applied fields such as technology and
medicine, a knowledge of classical languages is not as rigid a
prerequisite as it used to be. However, traces of their influence
remain. Firstly, languages such as Greek, Latin and Arabic -- either
directly or via more recently derived languages such as French -- have
provided not only most of the technical terms used in Western science,
but also a de facto vocabulary of roots, prefixes and suffixes for the
construction of new terms as required. Echoes of the consequences sound
in remarks such as "Television? The word is half Latin and half Greek.

Branches of science that are based, however tenuously, on fields of
study known to the ancients, or that were established by more recent
workers familiar with Greek and Latin, often use terminology that is
fairly correct descriptive Latin, or occasionally Greek. Descriptive
human anatomy or works on biological morphology often use such terms,
for example, musculus gluteus maximus simply means the ``largest rump
muscle'', where musculus was the Latin for ``little mouse'' and the name
applied to muscles. During the last two centuries there has been an
increasing tendency to modernise the terminology, though how beneficial
that might be is subject to discussion. In other descriptive anatomical
terms, whether in vertebrates or invertebrates, a frenum (a structure
for keeping something in place) is simply the Latin for a bridle; and a
foramen (a passage or perforation) also is the actual Latin word.

A special class of terminology that overwhelmingly is derived from
classical sources, is biological classification, in which binomial
nomenclature still is most often based on classical origins.
\href{https://en.wikipedia.org/wiki/Binomial_nomenclature}{Binomial
nomenclature} (``two-term naming system'') or binary nomenclature, is a
formal system of naming species of living things by giving each a name
composed of two parts, both of which use Latin grammatical forms,
although they can be based on words from other languages. Such a name is
called a binomial name (which may be shortened to just ``binomial''), a
binomen, binominal name or a scientific name; more informally it is also
called a Latin name.

The first part of the name -- the generic name -- identifies the genus
to which the species belongs, while the second part -- the specific name
or specific epithet -- identifies the species within the genus. For
example, humans belong to the genus Homo and within this genus to the
species \emph{Homo sapiens}. \emph{Tyrannosaurus rex} is probably the
most widely known binomial. The formal introduction of this system of
naming species is credited to
\href{https://en.wikipedia.org/wiki/Carl_Linnaeus}{Carl Linnaeus},
effectively beginning with his work Species Plantarum in 1753. But
\href{https://en.wikipedia.org/wiki/Gaspard_Bauhin}{Gaspard Bauhin}, in
as early as 1622, had introduced in his book Pinax theatri botanici
(English, Illustrated exposition of plants) many names of genera that
were later adopted by Linnaeus.

The application of binomial nomenclature is now governed by various
internationally agreed codes of rules, of which the two most important
are the International Code of Zoological Nomenclature (ICZN) for animals
and the International Code of Nomenclature for algae, fungi, and plants
(ICNafp).

In modern usage, the first letter of the first part of the name, the
genus, is always capitalized in writing, while that of the second part
is not, even when derived from a proper noun such as the name of a
person or place. Similarly, both parts are italicized when a binomial
name occurs in normal text (or underlined in handwriting). Thus the
binomial name of the annual phlox (named after botanist Thomas Drummond)
is now written as
\href{https://en.wikipedia.org/wiki/Phlox_drummondii}{\emph{Phlox
drummondii}}.

\hypertarget{the-scientific-method}{%
\subsubsection{The Scientific Method}\label{the-scientific-method}}

The \href{https://en.wikipedia.org/wiki/Scientific_method}{scientific
method} is the process by which science is carried out. Scientific
research involves using the scientific method, which seeks to
objectively explain the events of nature in a reproducible way. An
explanatory thought experiment or hypothesis is put forward as
explanation using principles such as parsimony (also known as ``Occam's
Razor'') and are generally expected to seek consilience -- fitting well
with other accepted facts related to the phenomena.

A \href{https://en.wikipedia.org/wiki/Hypothesis}{hypothesis} (plural
hypotheses) is a proposed explanation for a phenomenon. For a hypothesis
to be a scientific hypothesis, the scientific method requires that one
can test it. An experiment is a procedure carried out to support,
refute, or validate a hypothesis. Experiments provide insight into
cause-and-effect by demonstrating what outcome occurs when a particular
factor is manipulated. Experiments vary greatly in goal and scale, but
always rely on repeatable procedure and logical analysis of the results.
Scientists generally base scientific hypotheses on previous observations
that cannot satisfactorily be explained with the available scientific
theories. Even though the words ``hypothesis'' and ``theory'' are often
used synonymously, a scientific hypothesis is not the same as a
scientific theory. A working hypothesis is a provisionally accepted
hypothesis proposed for further research, in a process beginning with an
educated guess or thought.

\href{https://en.wikipedia.org/wiki/Occam\%27s_razor}{Occam's razor} or
law of parsimony (Latin: lex parsimoniae) is the problem-solving
principle that ``entities should not be multiplied without necessity.''
The idea is attributed to English Franciscan friar William of Ockham
(c.~1287--1347), a scholastic philosopher and theologian who used a
preference for simplicity to defend the idea of divine miracles. It is
variously paraphrased by statements like ``the simplest explanation is
most likely the right one''. This philosophical razor advocates that
when presented with competing hypotheses about the same prediction, one
should select the solution with the fewest assumptions, and that this is
not meant to be a way of choosing between hypotheses that make different
predictions.

A hypothesis is used to make falsifiable predictions that are testable
by experiment or observation. The predictions are to be posted before a
confirming experiment or observation is sought, as proof that no
tampering has occurred. Disproof of a prediction is evidence of
progress. This is done partly through observation of natural phenomena,
but also through experimentation that tries to simulate natural events
under controlled conditions as appropriate to the discipline (in the
observational sciences, such as astronomy or geology, a predicted
observation might take the place of a controlled experiment).
Experimentation is especially important in science to help establish
causal relationships (to avoid the correlation fallacy).

\href{https://en.wikipedia.org/wiki/Falsifiability}{Falsifiability} or
refutability is the capacity for a statement, theory or hypothesis to be
contradicted by evidence. For example, the statement ``All swans are
white'' is falsifiable because one can observe that black swans exist

When a hypothesis proves unsatisfactory, it is either modified or
discarded. If the hypothesis survived testing, it may become adopted
into the framework of a scientific theory, a logically reasoned,
self-consistent model or framework for describing the behavior of
certain natural phenomena. A theory typically describes the behavior of
much broader sets of phenomena than a hypothesis; commonly, a large
number of hypotheses can be logically bound together by a single theory.
Thus a theory is a hypothesis explaining various other hypotheses. In
that vein, theories are formulated according to most of the same
scientific principles as hypotheses. In addition to testing hypotheses,
scientists may also generate a model, an attempt to describe or depict
the phenomenon in terms of a logical, physical or mathematical
representation and to generate new hypotheses that can be tested, based
on observable phenomena.

While performing experiments to test hypotheses, scientists may have a
preference for one outcome over another, and so it is important to
ensure that science as a whole can eliminate this bias. This can be
achieved by careful experimental design, transparency, and a thorough
peer review process of the experimental results as well as any
conclusions. After the results of an experiment are announced or
published, it is normal practice for independent researchers to
double-check how the research was performed, and to follow up by
performing similar experiments to determine how dependable the results
might be. Taken in its entirety, the scientific method allows for highly
creative problem solving while minimizing any effects of subjective bias
on the part of its users (especially the confirmation bias).

\hypertarget{dissemination-of-scientific-knowledge}{%
\subsubsection{Dissemination of Scientific
Knowledge}\label{dissemination-of-scientific-knowledge}}

Scientific research is published in an enormous range of scientific
literature. Scientific journals communicate and document the results of
research carried out in universities and various other research
institutions, serving as an archival record of science. The first
scientific journals, Journal des Sçavans followed by the Philosophical
Transactions, began publication in 1665. Since that time the total
number of active periodicals has steadily increased. In 1981, one
estimate for the number of scientific and technical journals in
publication was 11,500. The United States National Library of Medicine
currently indexes 5,516 journals that contain articles on topics related
to the life sciences. Although the journals are in 39 languages, 91
percent of the indexed articles are published in English.

Most scientific journals cover a single scientific field and publish the
research within that field; the research is normally expressed in the
form of a scientific paper. Science has become so pervasive in modern
societies that it is generally considered necessary to communicate the
achievements, news, and ambitions of scientists to a wider populace.

An important aspect of the dissemination of scientific results is
scholarly \href{https://en.wikipedia.org/wiki/Peer_review}{peer review}
(also known as refereeing). This is the process of subjecting an
author's scholarly work, research, or ideas to the scrutiny of others
who are experts in the same field, before a paper describing this work
is published in a journal, conference proceedings or as a book. The peer
review helps the publisher (that is, the editor-in-chief, the editorial
board or the program committee) decide whether the work should be
accepted, considered acceptable with revisions, or rejected.

Researchers have peer reviewed manuscripts prior to publishing them in a
variety of ways since the 18th century. The main goal of this practice
is to improve the relevance and accuracy of scientific discussions. Even
though experts often criticize peer review for a number of reasons, the
process is still often considered the ``gold standard'' of science.
Occasionally however, peer review approves studies that are later found
to be wrong and rarely deceptive or fraudulent results are discovered
prior to publication. Thus, there seems to be an element of discord
between the ideology behind and the practice of peer review. By failing
to effectively communicate that peer review is imperfect, the message
conveyed to the wider public is that studies published in peer-reviewed
journals are ``true'' and that peer review protects the literature from
flawed science.

Open access (OA) is a set of principles and a range of practices through
which research outputs are distributed online, free of cost or other
access barriers. With open access strictly defined (according to the
2001 definition), or libre open access, barriers to copying or reuse are
also reduced or removed by applying an open license for copyright.

The main focus of the open access movement is ``peer reviewed research
literature.'' Historically, this has centered mainly on print-based
academic journals. Whereas conventional (non-open access) journals cover
publishing costs through access tolls such as subscriptions, site
licenses or pay-per-view charges, open-access journals are characterised
by funding models which do not require the reader to pay to read the
journal's contents but the authors may be charged for the cost of
publishing .

\hypertarget{philosophy-of-science}{%
\subsection{Philosophy of Science}\label{philosophy-of-science}}

\href{https://en.wikipedia.org/wiki/Philosophy_of_science}{Philosophy of
science} is a branch of philosophy concerned with the foundations,
methods, and implications of science. The central questions of this
study concern what qualifies as science, the reliability of scientific
theories, and the ultimate purpose of science.

There is no consensus among philosophers about many of the central
problems concerned with the philosophy of science, including whether
science can reveal the truth about unobservable things and whether
scientific reasoning can be justified at all.

Problems of philosophy of science include

\begin{itemize}
\tightlist
\item
  defining science
\item
  scientific explanation
\item
  justifying science
\item
  separability of theory and obsrvation
\item
  purpose of science
\item
  values and science
\end{itemize}

Scientists usually take for granted a set of basic assumptions that are
needed to justify the scientific method: (1) that there is an objective
reality shared by all rational observers; (2) that this objective
reality is governed by natural laws; (3) that these laws can be
discovered by means of systematic observation and experimentation.
Philosophy of science seeks a deep understanding of what these
underlying assumptions mean and whether they are valid.

The belief that scientific theories should and do represent metaphysical
reality is known as realism. It can be contrasted with anti-realism, the
view that the success of science does not depend on it being accurate
about unobservable entities such as electrons. One form of anti-realism
is idealism, the belief that the mind or consciousness is the most basic
essence, and that each mind generates its own reality. In an idealistic
world view, what is true for one mind need not be true for other minds.

There are different schools of thought in philosophy of science. The
most popular position is empiricism, which holds that knowledge is
created by a process involving observation and that scientific theories
are the result of generalizations from such observations. Empiricism
generally encompasses inductivism, a position that tries to explain the
way general theories can be justified by the finite number of
observations humans can make and hence the finite amount of empirical
evidence available to confirm scientific theories. This is necessary
because the number of predictions those theories make is infinite, which
means that they cannot be known from the finite amount of evidence using
deductive logic only. Many versions of empiricism exist, with the
predominant ones being Bayesianism and the hypothetico-deductive method.

Logical positivism, later called logical empiricism, and both of which
together are also known as neopositivism, is a branch in Western
philosophy whose central thesis is the verification principle (also
known as the verifiability criterion of meaning). This theory of
knowledge asserts that only statements verifiable through direct
observation or logical proof are meaningful. Starting in the late 1920s,
groups of philosophers, scientists, and mathematicians formed the Berlin
Circle and the Vienna Circle, which, in these two cities, would propound
the ideas of logical positivism.

Flourishing in several European centres through the 1930s, the movement
sought to prevent confusion rooted in unclear language and unverifiable
claims by converting philosophy into ``scientific philosophy'', which,
according to the logical positivists, ought to share the bases and
structures of empirical sciences' best examples, such as Albert
Einstein's general theory of relativity. Despite its ambition to
overhaul philosophy by studying and mimicking the extant conduct of
empirical science, logical positivism became erroneously stereotyped as
a movement to regulate the scientific process and to place strict
standards on it.

After World War II, the movement shifted to a milder variant, logical
empiricism, led mainly by
\href{https://en.wikipedia.org/wiki/Carl_Gustav_Hempel}{Carl Hempel},
who, during the rise of Nazism, had immigrated to the United States. In
the ensuing years, the movement's central premises, still unresolved,
were criticised by other philosophers, particularly
\href{https://en.wikipedia.org/wiki/Willard_Van_Orman_Quine}{Willard van
Orman Quine} and by Austrian-British philosopher
\href{https://en.wikipedia.org/wiki/Karl_Popper}{Karl Popper}.

Empiricism has stood in contrast to rationalism, the position originally
associated with Descartes, which holds that knowledge is created by the
human intellect, not by observation. Critical rationalism is a
contrasting 20th-century approach to science, first defined Karl Popper.
Popper rejected the way that empiricism describes the connection between
theory and observation. He claimed that theories are not generated by
observation, but that observation is made in the light of theories and
that the only way a theory can be affected by observation is when it
comes in conflict with it. Popper proposed replacing verifiability with
falsifiability as the landmark of scientific theories and replacing
induction with falsification as the empirical method. Popper further
claimed that there is actually only one universal method, not specific
to science: the negative method of criticism, trial and error. It covers
all products of the human mind, including science, mathematics,
philosophy, and art.

Another approach, instrumentalism, colloquially termed
``\href{https://physicstoday.scitation.org/doi/10.1063/1.1768652}{shut
up and calculate},'' emphasizes the utility of theories as instruments
for explaining and predicting phenomena. It views scientific theories as
black boxes with only their input (initial conditions) and output
(predictions) being relevant. Consequences, theoretical entities, and
logical structure are claimed to be something that should simply be
ignored and that scientists should not make a fuss about (see
interpretations of quantum mechanics). Close to instrumentalism is
constructive empiricism, according to which the main criterion for the
success of a scientific theory is whether what it says about observable
entities is true.

In his book The
\href{https://en.wikipedia.org/wiki/The_Structure_of_Scientific_Revolutions}{Structure
of Scientific Revolutions} the American philosopher of science
\href{https://en.wikipedia.org/wiki/Thomas_Kuhn}{Thomas Samuel Kuhn}
made several claims concerning the progress of scientific knowledge:
that scientific fields undergo periodic ``paradigm shifts'' rather than
solely progressing in a linear and continuous way, and that these
paradigm shifts open up new approaches to understanding what scientists
would never have considered valid before; and that the notion of
scientific truth, at any given moment, cannot be established solely by
objective criteria but is defined by a consensus of a scientific
community. Competing paradigms are frequently incommensurable; that is,
they are competing and irreconcilable accounts of reality. Thus, our
comprehension of science can never rely wholly upon ``objectivity''
alone. Science must account for subjective perspectives as well, since
all objective conclusions are ultimately founded upon the subjective
conditioning/worldview of its researchers and participants.

A scientific theory is empirical and is always open to falsification if
new evidence is presented. That is, no theory is ever considered
strictly certain as science accepts the concept of fallibilism. The
philosopher of science Karl Popper sharply distinguished truth from
certainty. He wrote that scientific knowledge ``consists in the search
for truth,'' but it "is not the search for certainty \ldots{} All human
knowledge is fallible and therefore uncertain.

New scientific knowledge rarely results in vast changes in our
understanding. Knowledge in science is gained by a gradual synthesis of
information from different experiments by various researchers across
different branches of science; it is more like a climb than a leap.
Theories vary in the extent to which they have been tested and verified,
as well as their acceptance in the scientific community. For example,
heliocentric theory, the theory of evolution, relativity theory, and
germ theory still bear the name ``theory'' even though, in practice,
they are considered factual. Philosopher Barry Stroud adds that,
although the best definition for ``knowledge'' is contested, being
skeptical and entertaining the possibility that one is incorrect is
compatible with being correct. Therefore, scientists adhering to proper
scientific approaches will doubt themselves even once they possess the
truth.

The Polish and Israeli physician, biologist and philosopher of science
\href{https://en.wikipedia.org/wiki/Ludwik_Fleck}{Ludwik Fleck} wrote in
his 1935 book ``Entstehung und Entwicklung einer wissenschaftlichen
Tatsache; Einführung in die Lehre vom Denkstil und Denkkollektiv'' that
the development of truth in scientific research was an unattainable
ideal as different researchers were locked into thought collectives (or
thought-styles). This means ``that a pure and direct observation cannot
exist: in the act of perceiving objects the observer, i.e.~the
epistemological subject, is always influenced by the epoch and the
environment to which he belongs, that is by what Fleck calls the thought
style.'' A ``truth'' was a relative value, expressed in the language or
symbolism of the thought collective in which it belonged, and subject to
the social and temporal structure of this collective. To state therefore
that a specific truth is true or false is impossible. It is true in its
own collective, but incomprehensible or unverifiable in most others. He
felt that the development of scientific insights was not unidirectional
and does not consist of just accumulating new pieces of information, but
also in overthrowing the old ones. This overthrowing of old insights is
difficult because a collective attains over time a specific way of
investigating, bringing with it a blindness to alternative ways of
observing and conceptualization. Change was especially possible when
members of two thought collectives met and cooperated in observing,
formulating hypothesis and ideas. He strongly advocated comparative
epistemology. This approach anticipated later developments in social
constructionism, and especially the development of critical science and
technology studies.

In his book Against Method and Science in a Free Society the philosopher
\href{https://en.wikipedia.org/wiki/Paul_Feyerabend}{Paul Feyerabend}
defended the idea that there are no methodological rules which are
always used by scientists. He objected to any single prescriptive
scientific method on the grounds that any such method would limit the
activities of scientists, and hence restrict scientific progress. In his
view, science would benefit most from a ``dose'' of theoretical
anarchism. He also thought that theoretical anarchism was desirable
because it was more humanitarian than other systems of organization, by
not imposing rigid rules on scientists.

\begin{quote}
For is it not possible that science as we know it today, or a ``search
for the truth'' in the style of traditional philosophy, will create a
monster? Is it not possible that an objective approach that frowns upon
personal connections between the entities examined will harm people,
turn them into miserable, unfriendly, self-righteous mechanisms without
charm or humour? ``Is it not possible,'' asks Kierkegaard, ``that my
activity as an objective {[}or critico-rational{]} observer of nature
will weaken my strength as a human being?'' I suspect the answer to many
of these questions is affirmative and I believe that a reform of the
sciences that makes them more anarchic and more subjective (in
Kierkegaard's sense) is urgently needed.
\end{quote}

\begin{quote}
Against Method: Outline of an Anarchistic Theory of Knowledge (1975)
\end{quote}

Feyerabend's position was seen as radical in the philosophy of science,
because it implies that philosophy can neither succeed in providing a
general description of science, nor in devising a method for
differentiating products of science from non-scientific entities like
myths. (Feyerabend's position also implies that philosophical guidelines
should be ignored by scientists, if they are to aim for progress.)

To support his position that methodological rules generally do not
contribute to scientific success, Feyerabend provides counterexamples to
the claim that (good) science operates according to a certain fixed
method. He took some examples of episodes in science that are generally
regarded as indisputable instances of progress (e.g.~the Copernican
revolution), and argued that these episodes violated all common
prescriptive rules of science. Moreover, he claimed that applying such
rules in these historical situations would actually have prevented
scientific revolution.

According to Feyerabend, new theories came to be accepted not because of
their accord with scientific method, but because their supporters made
use of any trick -- rational, rhetorical or ribald -- in order to
advance their cause. Without a fixed ideology, or the introduction of
religious tendencies, the only approach which does not inhibit progress
(using whichever definition one sees fit) is ``anything goes'':
``\,`anything goes' is not a `principle' I hold\ldots{} but the
terrified exclamation of a rationalist who takes a closer look at
history.''

\hypertarget{science-and-religion}{%
\subsection{Science And Religion}\label{science-and-religion}}

Historians of
\href{https://en.wikipedia.org/wiki/Relationship_between_religion_and_science}{science
and of religion}, philosophers, theologians, scientists, and others from
various geographical regions and cultures have addressed numerous
aspects of the relationship between religion and science. Critical
questions in this debate include whether religion and science are
compatible, whether religious beliefs can be conducive to science (or
necessarily inhibit it), and what the nature of religious beliefs is.
Another question is whether or not science has for some people become a
new religion.

Even though the ancient and medieval worlds did not have conceptions
resembling the modern understandings of ``science'' or of ``religion'',
certain elements of modern ideas on the subject recur throughout
history. The pair-structured phrases ``religion and science'' and
``science and religion'' first emerged in the literature in the 19th
century. This coincided with the refining of ``science'' (from the
studies of ``natural philosophy'') and of ``religion'' as distinct
concepts in the preceding few centuries---partly due to
professionalization of the sciences, the Protestant Reformation,
colonization, and globalization. Since then the relationship between
science and religion has been characterized in terms of `conflict',
`harmony', `complexity', and `mutual independence', among others.

Both science and religion are complex social and cultural endeavors that
vary across cultures and change over time. Most scientific (and
technical) innovations prior to the scientific revolution were achieved
by societies organized by religious traditions. Ancient pagan, Islamic,
and Christian scholars pioneered individual elements of the scientific
method. Roger Bacon, often credited with formalizing the scientific
method, was a Franciscan friar. Confucian thought, whether religious or
non-religious in nature, has held different views of science over time.
Many 21st-century Buddhists view science as complementary to their
beliefs. While the classification of the material world by the ancient
Indians and Greeks into air, earth, fire and water was more
metaphysical, and figures like Anaxagoras questioned certain popular
views of Greek divinities, medieval Middle Eastern scholars empirically
classified materials.

Events in Europe such as the Galileo affair of the early 17th century,
associated with the scientific revolution and the Age of Enlightenment,
led scholars such as John William Draper to postulate (c.  1874) a
conflict thesis, suggesting that religion and science have been in
conflict methodologically, factually and politically throughout history.
Some contemporary scientists (such as Richard Dawkins, Lawrence Krauss,
Peter Atkins, and Donald Prothero) subscribe to this thesis. However,
the conflict thesis has lost favor among most contemporary historians of
science.

Many scientists, philosophers, and theologians throughout history, such
as Francisco Ayala, Kenneth R. Miller and Francis Collins, have seen
compatibility or interdependence between religion and science. Biologist
Stephen Jay Gould, other scientists, and some contemporary theologians
regard religion and science as non-overlapping magisteria, addressing
fundamentally separate forms of knowledge and aspects of life. Some
theologians or historians of science, including John Lennox, Thomas
Berry, Brian Swimme and Ken Wilber propose an interconnection between
science and religion, while others such as Ian Barbour believe there are
even parallels.

Public acceptance of scientific facts may sometimes be influenced by
religious beliefs such as in the United States, where some reject the
concept of evolution by natural selection, especially regarding human
beings. Nevertheless, the American National Academy of Sciences has
written that ``the evidence for evolution can be fully compatible with
religious faith'', a view endorsed by many religious denominations.

The concepts of ``science'' and ``religion'' are a recent invention:
``religion'' emerged in the 17th century in the midst of colonization
and globalization and the Protestant Reformation, ``science'' emerged in
the 19th century in the midst of attempts to narrowly define those who
studied nature. Originally what is now known as ``science'' was
pioneered as ``natural philosophy''.

It was in the 19th century that the terms ``Buddhism'', ``Hinduism'',
``Taoism'', ``Confucianism'' and ``World Religions'' first emerged. In
the ancient and medieval world, the etymological Latin roots of both
science (scientia) and religion (religio) were understood as inner
qualities of the individual or virtues, never as doctrines, practices,
or actual sources of knowledge.

It was in the 19th century that the concept of ``science'' received its
modern shape with new titles emerging such as ``biology'' and
``biologist'', ``physics'', and ``physicist'', among other technical
fields and titles; institutions and communities were founded, and
unprecedented applications to and interactions with other aspects of
society and culture occurred. Even in the 19th century, a treatise by
Lord Kelvin and Peter Guthrie Tait's, which helped define much of modern
physics, was titled Treatise on Natural Philosophy (1867).

It was in the 17th century that the concept of ``religion'' received its
modern shape despite the fact that ancient texts like the Bible, the
Quran, and other texts did not have a concept of religion in the
original languages and neither did the people or the cultures in which
these texts were written. In the 19th century, Max Müller noted that
what is called ancient religion today, would have been called ``law'' in
antiquity.

The development of sciences (especially natural philosophy) in Western
Europe during the Middle Ages, has considerable foundation in the works
of the Arabs who translated Greek and Latin compositions. The works of
Aristotle played a major role in the institutionalization,
systematization, and expansion of reason. Christianity accepted reason
within the ambit of faith. In Christendom, reason was considered
subordinate to revelation, which contained the ultimate truth and this
truth could not be challenged. In medieval universities, the faculty for
natural philosophy and theology were separate, and discussions
pertaining to theological issues were often not allowed to be undertaken
by the faculty of philosophy.

Natural philosophy, as taught in the arts faculties of the universities,
was seen as an essential area of study in its own right and was
considered necessary for almost every area of study. It was an
independent field, separated from theology, and enjoyed a good deal of
intellectual freedom as long as it was restricted to the natural world.
In general, there was religious support for natural science by the late
Middle Ages and a recognition that it was an important element of
learning.

The extent to which medieval science led directly to the new philosophy
of the scientific revolution remains a subject for debate, but it
certainly had a significant influence.

The Middle Ages laid ground for the developments that took place in
science, during the Renaissance which immediately succeeded it. By 1630,
ancient authority from classical literature and philosophy, as well as
their necessity, started eroding, although scientists were still
expected to be fluent in Latin, the international language of Europe's
intellectuals. With the sheer success of science and the steady advance
of rationalism, the individual scientist gained prestige. Along with the
inventions of this period, especially the printing press by Johannes
Gutenberg, allowed for the dissemination of the Bible in languages of
the common people (languages other than Latin). This allowed more people
to read and learn from the scripture, leading to the Evangelical
movement. The people who spread this message, concentrated more on
individual agency rather than the structures of the Church.

In the 17th century, founders of the Royal Society largely held
conventional and orthodox religious views, and a number of them were
prominent Churchmen. While theological issues that had the potential to
be divisive were typically excluded from formal discussions of the early
Society, many of its fellows nonetheless believed that their scientific
activities provided support for traditional religious belief. Clerical
involvement in the Royal Society remained high until the mid-nineteenth
century, when science became more professionalised.

Albert Einstein supported the compatibility of some interpretations of
religion with science. In ``Science, Philosophy and Religion, A
Symposium'' published by the Conference on Science, Philosophy and
Religion in Their Relation to the Democratic Way of Life, Inc., New York
in 1941, Einstein stated:

\begin{quote}
Accordingly, a religious person is devout in the sense that he has no
doubt of the significance and loftiness of those superpersonal objects
and goals which neither require nor are capable of rational foundation.
They exist with the same necessity and matter-of-factness as he himself.
In this sense religion is the age-old endeavor of mankind to become
clearly and completely conscious of these values and goals and
constantly to strengthen and extend their effect. If one conceives of
religion and science according to these definitions then a conflict
between them appears impossible. For science can only ascertain what is,
but not what should be, and outside of its domain value judgments of all
kinds remain necessary. Religion, on the other hand, deals only with
evaluations of human thought and action: it cannot justifiably speak of
facts and relationships between facts. According to this interpretation
the well-known conflicts between religion and science in the past must
all be ascribed to a misapprehension of the situation which has been
described.
\end{quote}

Prominent modern scientists who are atheists include evolutionary
biologist Richard Dawkins and Nobel Prize--winning physicist Steven
Weinberg. Prominent scientists advocating religious belief include Nobel
Prize--winning physicist and United Church of Christ member Charles
Townes, evangelical Christian and past head of the Human Genome Project
Francis Collins, and climatologist John T. Houghton.

A modern view, described by Stephen Jay Gould as ``non-overlapping
magisteria'' (NOMA), is that science and religion deal with
fundamentally separate aspects of human experience and so, when each
stays within its own domain, they co-exist peacefully. While Gould spoke
of independence from the perspective of science, W. T. Stace viewed
independence from the perspective of the philosophy of religion. Stace
felt that science and religion, when each is viewed in its own domain,
are both consistent and complete. They originate from different
perceptions of reality, as Arnold O. Benz points out, but meet each
other, for example, in the feeling of amazement and in ethics.

The USA's National Academy of Science supports the view that science and
religion are independent.

\begin{quote}
Science and religion are based on different aspects of human experience.
In science, explanations must be based on evidence drawn from examining
the natural world. Scientifically based observations or experiments that
conflict with an explanation eventually must lead to modification or
even abandonment of that explanation. Religious faith, in contrast, does
not depend on empirical evidence, is not necessarily modified in the
face of conflicting evidence, and typically involves supernatural forces
or entities. Because they are not a part of nature, supernatural
entities cannot be investigated by science. In this sense, science and
religion are separate and address aspects of human understanding in
different ways. Attempts to put science and religion against each other
create controversy where none needs to exist.
\end{quote}

\end{document}
