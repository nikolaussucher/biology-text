\hypertarget{sensation-receptors-organs-and-systems}{%
\section{Sensation: Receptors, Organs And
Systems}\label{sensation-receptors-organs-and-systems}}

Sensation is the physical process during which sensory systems respond
to stimuli and provide data for perception. A sense is any of the
systems involved in sensation. During sensation, sense organs engage in
stimulus collection and transduction. Sensation is often differentiated
from the related and dependent concept of perception, which processes
and integrates sensory information in order to give meaning to and
understand detected stimuli, giving rise to subjective perceptual
experience, or qualia. Sensation and perception are central to and
precede almost all aspects of cognition, behavior and thought.

The sensory nervous system is a part of the nervous system responsible
for processing sensory information. A sensory system consists of sensory
neurons (including the sensory receptor cells), neural pathways, and
parts of the brain involved in sensory perception. Commonly recognized
sensory systems are those for vision, hearing, touch, taste, smell, and
balance. In short, senses are transducers from the physical world to the
realm of the mind where we interpret the information, creating our
perception of the world around us.

Organisms need information to solve at least three kinds of problems:
(a) to maintain an appropriate environment, i.e., homeostasis; (b) to
time activities (e.g., seasonal changes in behavior) or synchronize
activities with those of conspecifics; and (c) to locate and respond to
resources or threats (e.g., by moving towards resources or evading or
attacking threats). Organisms also need to transmit information in order
to influence another's behavior: to identify themselves, warn
conspecifics of danger, coordinate activities, or deceive.

The receptive field is the area of the body or environment to which a
receptor organ and receptor cells respond. For instance, the part of the
world an eye can see, is its receptive field; the light that each rod or
cone can see, is its receptive field. Receptive fields have been
identified for the visual system, auditory system and somatosensory
system.

Sensory systems code for four aspects of a stimulus; type (modality),
intensity, location, and duration. Arrival time of a sound pulse and
phase differences of continuous sound are used for sound localization.
Certain receptors are sensitive to certain types of stimuli (for
example, different mechanoreceptors respond best to different kinds of
touch stimuli, like sharp or blunt objects). Receptors send impulses in
certain patterns to send information about the intensity of a stimulus
(for example, how loud a sound is). The location of the receptor that is
stimulated gives the brain information about the location of the
stimulus (for example, stimulating a mechanoreceptor in a finger will
send information to the brain about that finger). The duration of the
stimulus (how long it lasts) is conveyed by firing patterns of
receptors. These impulses are transmitted to the brain through afferent
neurons.

While debate exists among neurologists as to the specific number of
senses due to differing definitions of what constitutes a sense, Gautama
Buddha and Aristotle classified five `traditional' human senses which
have become universally accepted: touch, taste, smell, sight, and
hearing. Other senses that have been well-accepted in most mammals,
including humans, include nociception, equilibrioception, kinaesthesia,
and thermoception. Furthermore, some nonhuman animals have been shown to
possess alternate senses, including magnetoception and electroreception.

In organisms, a sensory organ consists of a group of related sensory
cells that respond to a specific type of physical stimulus. Via cranial
and spinal nerves, the different types of sensory receptor cells
(mechanoreceptors, photoreceptors, chemoreceptors, thermoreceptors) in
sensory organs transduct sensory information from sensory organs towards
the central nervous system, to the sensory cortices in the brain, where
sensory signals are further processed and interpreted (perceived).
Sensory systems, or senses, are often divided into external
(exteroception) and internal (interoception) sensory systems. Sensory
modalities or submodalities refer to the way sensory information is
encoded or transduced. Multimodality integrates different senses into
one unified perceptual experience. For example, information from one
sense has the potential to influence how information from another is
perceived. Sensation and perception are studied by a variety of related
fields, most notably psychophysics, neurobiology, cognitive psychology,
and cognitive science.

Humans have a multitude of sensory systems. Human external sensation is
based on the sensory organs of the eyes, ears, skin, inner ear, nose,
and mouth. The corresponding sensory systems of the visual system (sense
of vision), auditory system (sense of hearing), somatosensory system
(sense of touch), vestibular system (sense of balance), olfactory system
(sense of smell), and gustatory system (sense of taste) contribute,
respectively, to the perceptions of vision, hearing, touch, spatial
orientation, smell, and taste (flavor). Internal sensation, or
interoception, detects stimuli from internal organs and tissues. Many
internal sensory and perceptual systems exist in humans, including
proprioception (body position) and nociception (pain). Further internal
chemoreception and osmoreception based sensory systems lead to various
perceptions, such as hunger, thirst, suffocation, and nausea, or
different involuntary behaviors, such as vomiting.

Nonhuman animals experience sensation and perception, with varying
levels of similarity to and difference from humans and other animal
species. For example, mammals, in general, have a stronger sense of
smell than humans. Some animal species lack one or more human sensory
system analogues, some have sensory systems that are not found in
humans, while others process and interpret the same sensory information
in very different ways. For example, some animals are able to detect
electrical and magnetic fields, air moisture, or polarized light, while
others sense and perceive through alternative systems, such as
echolocation. Recently, it has been suggested that plants and artificial
agents may be able to detect and interpret environmental information in
an analogous manner to animals.

\hypertarget{sensory-receptors}{%
\subsection{Sensory Receptors}\label{sensory-receptors}}

Sensory receptors are the cells or structures that detect sensations.
Stimuli in the environment activate specialized receptor cells in the
peripheral nervous system. During transduction, physical stimulus is
converted into action potential by receptors and transmitted towards the
central nervous system for processing. Different types of stimuli are
sensed by different types of receptor cells. Receptor cells can be
classified into types on the basis of three different criteria: cell
type, position, and function. Receptors can be classified structurally
on the basis of cell type and their position in relation to stimuli they
sense. Receptors can further be classified functionally on the basis of
the transduction of stimuli, or how the mechanical stimulus, light, or
chemical changed the cell membrane potential.

One way to classify receptors is based on their location relative to the
stimuli. An exteroceptor is a receptor that is located near a stimulus
of the external environment, such as the somatosensory receptors that
are located in the skin. An interoceptor is one that interprets stimuli
from internal organs and tissues, such as the receptors that sense the
increase in blood pressure in the aorta or carotid sinus.

The cells that interpret information about the environment can be either
(1) a neuron that has a free nerve ending, with dendrites embedded in
tissue that would receive a sensation; (2) a neuron that has an
encapsulated ending in which the sensory nerve endings are encapsulated
in connective tissue that enhances their sensitivity; or (3) a
specialized receptor cell, which has distinct structural components that
interpret a specific type of stimulus. The pain and temperature
receptors in the dermis of the skin are examples of neurons that have
free nerve endings (1). Also located in the dermis of the skin are
lamellated corpuscles, neurons with encapsulated nerve endings that
respond to pressure and touch (2). The cells in the retina that respond
to light stimuli are an example of a specialized receptor (3), a
photoreceptor.

A transmembrane protein receptor is a protein in the cell membrane that
mediates a physiological change in a neuron, most often through the
opening of ion channels or changes in the cell signaling processes.
Transmembrane receptors are activated by chemicals called ligands. For
example, a molecule in food can serve as a ligand for taste receptors.
Other transmembrane proteins, which are not accurately called receptors,
are sensitive to mechanical or thermal changes. Physical changes in
these proteins increase ion flow across the membrane, and can generate
an action potential or a graded potential in the sensory neurons.

A third classification of receptors is by how the receptor transduces
stimuli into membrane potential changes. Stimuli are of three general
types. Some stimuli are ions and macromolecules that affect
transmembrane receptor proteins when these chemicals diffuse across the
cell membrane. Some stimuli are physical variations in the environment
that affect receptor cell membrane potentials. Other stimuli include the
electromagnetic radiation from visible light. For humans, the only
electromagnetic energy that is perceived by our eyes is visible light.
Some other organisms have receptors that humans lack, such as the heat
sensors of snakes, the ultraviolet light sensors of bees, or magnetic
receptors in migratory birds.

Receptor cells can be further categorized on the basis of the type of
stimuli they transduce. The different types of functional receptor cell
types are mechanoreceptors, photoreceptors, chemoreceptors
(osmoreceptor), thermoreceptors, and nociceptors. Physical stimuli, such
as pressure and vibration, as well as the sensation of sound and body
position (balance), are interpreted through a mechanoreceptor.
Photoreceptors convert light (visible electromagnetic radiation) into
signals. Chemical stimuli can be interpreted by a chemoreceptor that
interprets chemical stimuli, such as an object's taste or smell, while
osmoreceptors respond to a chemical solute concentrations of body
fluids. Nociception (pain) interprets the presence of tissue damage,
from sensory information from mechano-, chemo-, and thermoreceptors.
Another physical stimulus that has its own type of receptor is
temperature, which is sensed through a thermoreceptor that is either
sensitive to temperatures above (heat) or below (cold) normal body
temperature.

Each sense organ (eyes or nose, for instance) requires a minimal amount
of stimulation in order to detect a stimulus. This minimum amount of
stimulus is called the absolute threshold. The absolute threshold is
defined as the minimum amount of stimulation necessary for the detection
of a stimulus 50\% of the time. Absolute threshold is measured by using
a method called signal detection. This process involves presenting
stimuli of varying intensities to a subject in order to determine the
level at which the subject can reliably detect stimulation in a given
sense.

Differential threshold or just noticeable difference (JDS) is the
smallest detectable difference between two stimuli, or the smallest
difference in stimuli that can be judged to be different from each
other. Weber's Law is an empirical law that states that the difference
threshold is a constant fraction of the comparison stimulus. According
to Weber's Law, bigger stimuli require larger differences to be noticed.

Perception occurs when nerves that lead from the sensory organs
(e.g.~eye) to the brain are stimulated, even if that stimulation is
unrelated to the target signal of the sensory organ. For example, in the
case of the eye, it does not matter whether light or something else
stimulates the optic nerve, that stimulation will results in visual
perception, even if there was no visual stimulus to begin with. (To
prove this point to yourself (and if you are a human), close your eyes
(preferably in a dark room) and press gently on the outside corner of
one eye through the eyelid. You will see a visual spot toward the inside
of your visual field, near your nose.

The initialization of sensation stems from the response of a specific
receptor to a physical stimulus. The receptors which react to the
stimulus and initiate the process of sensation are commonly
characterized in four distinct categories: chemoreceptors,
photoreceptors, mechanoreceptors, and thermoreceptors. All receptors
receive distinct physical stimuli and transduce the signal into an
electrical action potential. This action potential then travels along
afferent neurons to specific brain regions where it is processed and
interpreted.

\hypertarget{chemoreceptors}{%
\subsubsection{Chemoreceptors}\label{chemoreceptors}}

Chemoreceptors, or chemosensors, detect certain chemical stimuli and
transduce that signal into an electrical action potential. The two
primary types of chemoreceptors are:

Distance chemoreceptors are integral to receiving stimuli in gases in
the olfactory system through both olfactory receptor neurons and neurons
in the vomeronasal organ. Direct chemoreceptors that detect stimuli in
liquids include the taste buds in the gustatory system as well as
receptors in the aortic bodies which detect changes in oxygen
concentration.

\hypertarget{photoreceptors}{%
\subsubsection{Photoreceptors}\label{photoreceptors}}

Photoreceptors are capable of phototransduction, a process which
converts light (electromagnetic radiation) into, among other types of
energy, a membrane potential. The three primary types of photoreceptors
are: Cones are photoreceptors which respond significantly to color. In
humans the three different types of cones correspond with a primary
response to short wavelength (blue), medium wavelength (green), and long
wavelength (yellow/red). Rods are photoreceptors which are very
sensitive to the intensity of light, allowing for vision in dim
lighting. The concentrations and ratio of rods to cones is strongly
correlated with whether an animal is diurnal or nocturnal. In humans
rods outnumber cones by approximately 20:1, while in nocturnal animals,
such as the tawny owl, the ratio is closer to 1000:1. Ganglion Cells
reside in the adrenal medulla and retina where they are involved in the
sympathetic response. Of the \textasciitilde1.3 million ganglion cells
present in the retina, 1-2\% are believed to be photosensitive ganglia.
These photosensitive ganglia play a role in conscious vision for some
animals, and are believed to do the same in humans.

\hypertarget{mechanoreceptors}{%
\subsubsection{Mechanoreceptors}\label{mechanoreceptors}}

Mechanoreceptors are sensory receptors which respond to mechanical
forces, such as pressure or distortion. While mechanoreceptors are
present in hair cells and play an integral role in the vestibular and
auditory systems, the majority of mechanoreceptors are cutaneous and are
grouped into four categories:

\begin{itemize}
\tightlist
\item
  Slowly adapting type 1 receptors have small receptive fields and
  respond to static stimulation. These receptors are primarily used in
  the sensations of form and roughness.
\item
  Slowly adapting type 2 receptors have large receptive fields and
  respond to stretch. Similarly to type 1, they produce sustained
  responses to a continued stimuli.
\item
  Rapidly adapting receptors have small receptive fields and underlie
  the perception of slip. Pacinian receptors have large receptive fields
  and are the predominant receptors for high-frequency vibration.
\end{itemize}

\hypertarget{thermoreceptors}{%
\subsubsection{Thermoreceptors}\label{thermoreceptors}}

Thermoreceptors are sensory receptors which respond to varying
temperatures. While the mechanisms through which these receptors operate
is unclear, recent discoveries have shown that mammals have at least two
distinct types of thermoreceptors:{[}permanent dead link{]}‹See
TfM›{[}failed verification{]}

\begin{itemize}
\tightlist
\item
  The end-bulb of Krause, or bulboid corpuscle, detects temperatures
  above body temperature.
\item
  Ruffini's end organ detects temperatures below body temperature.
\end{itemize}

TRPV1 is a heat-activated channel that acts as a small heat detecting
thermometer in the membrane which begins the polarization of the neural
fiber when exposed to changes in temperature. Ultimately, this allows us
to detect ambient temperature in the warm/hot range. Similarly, the
molecular cousin to TRPV1, TRPM8, is a cold-activated ion channel that
responds to cold. Both cold and hot receptors are segregated by distinct
subpopulations of sensory nerve fibers, which shows us that the
information coming into the spinal cord is originally separate. Each
sensory receptor has its own ``labeled line'' to convey a simple
sensation experienced by the recipient. Ultimately, TRP channels act as
thermosensors, channels that help us to detect changes in ambient
temperatures.

\hypertarget{nociceptors}{%
\subsubsection{Nociceptors}\label{nociceptors}}

Nociceptors respond to potentially damaging stimuli by sending signals
to the spinal cord and brain. This process, called nociception, usually
causes the perception of pain. They are found in internal organs, as
well as on the surface of the body. Nociceptors detect different kinds
of damaging stimuli or actual damage. Those that only respond when
tissues are damaged are known as ``sleeping'' or ``silent'' nociceptors.

Thermal nociceptors are activated by noxious heat or cold at various
temperatures. Mechanical nociceptors respond to excess pressure or
mechanical deformation. Chemical nociceptors respond to a wide variety
of chemicals, some of which are signs of tissue damage. They are
involved in the detection of some spices in food.

\hypertarget{the-visual-system}{%
\subsection{The Visual System}\label{the-visual-system}}

Visual perception is the ability to interpret the surrounding
environment using light in the visible spectrum reflected by the objects
in the environment. This is different from visual acuity, which refers
to how clearly a person sees (for example ``20/20 vision''). A person
can have problems with visual perceptual processing even if they have
20/20 vision.

The resulting perception is also known as visual perception, eyesight,
sight, or vision. The various physiological components involved in
vision are referred to collectively as the visual system.

Different species are able to see different parts of the light spectrum;
for example, bees can see into the ultraviolet, while pit vipers can
accurately target prey with their pit organs, which are sensitive to
infrared radiation. The mantis shrimp possesses arguably the most
complex visual system in any species. The eye of the mantis shrimp holds
16 color receptive cones, whereas humans only have three. The variety of
cones enables them to perceive an enhanced array of colors as a
mechanism for mate selection, avoidance of predators, and detection of
prey. Swordfish also possess an impressive visual system. The eye of a
swordfish can generate heat to better cope with detecting their prey at
depths of 2000 feet. Certain one-celled micro-organisms, the warnowiid
dinoflagellates have eye-like ocelloids, with analogous structures for
the lens and retina of the multi-cellular eye. The armored shell of the
chiton \emph{Acanthopleura granulata} is also covered with hundreds of
aragonite crystalline eyes, named ocelli, which can form images.

Many fan worms, such as \emph{Acromegalomma interruptum} which live in
tubes on the sea floor of the Great Barrier Reef, have evolved compound
eyes on their tentacles, which they use to detect encroaching movement.
If movement is detected the fan worms will rapidly withdraw their
tentacles.

Only higher primate Old World (African) monkeys and apes have the same
kind of three-cone photoreceptor color vision humans have, while lower
primate New World (South American) monkeys have a two-cone photoreceptor
kind of color vision.

\hypertarget{the-eye}{%
\subsubsection{The Eye}\label{the-eye}}

Light entering the eye is refracted as it passes through the cornea. It
then passes through the pupil (controlled by the iris) and is further
refracted by the lens. The cornea and lens act together as a compound
lens to project an inverted image onto the retina.

(ref:ey) Horizontal section of the human eyeball. From
\href{https://archive.org/details/anatomyofhumanbo1918gray/page/n6/mode/2up}{Gray
Henry, Anatomy of the Human Body. 20\textsuperscript{th} Edition, Lea \&
Febiger, Philadelphia \& New York, 1918}

\texttt{\{r\ eye,\ fig.cap=\textquotesingle{}(ref:ey)\textquotesingle{},\ echo=FALSE,\ message=FALSE,\ warning=FALSE\}\ knitr::include\_graphics("./figures/sensation/anatomyofhumanbo1918gray\_1008.jpg")}

\hypertarget{the-retina}{%
\subsubsection{The Retina}\label{the-retina}}

The retina is the light-sensitive layer of tissue of the eye of most
vertebrates and some molluscs. The optics of the eye create a focused
two-dimensional image of the visual world on the retina, which
translates that image into electrical neural impulses to the brain to
create visual perception.

The retina translates an optical image into neural impulses starting
with the patterned excitation of the light-sensitive pigments of its
rods and cones, the retina's photoreceptor cells. The excitation is
processed by the neural system and various parts of the brain working in
parallel to form a representation of the external environment in the
brain.

Light striking the retina initiates a cascade of chemical and electrical
events that ultimately trigger nerve impulses that are sent to various
visual centres of the brain through the fibres of the optic nerve.
Neural signals from the rods and cones undergo processing by other
neurons, whose output takes the form of action potentials in retinal
ganglion cells whose axons form the optic nerve. Several important
features of visual perception can be traced to the retinal encoding and
processing of light.

The cones respond to bright light and mediate high-resolution colour
vision during daylight illumination (also called photopic vision). The
rod responses are saturated at daylight levels and don't contribute to
pattern vision. However, rods do respond to dim light and mediate
lower-resolution, monochromatic vision under very low levels of
illumination (called scotopic vision). The illumination in most office
settings falls between these two levels and is called mesopic vision. At
mesopic light levels, both the rods and cones are actively contributing
pattern information. What contribution the rod information makes to
pattern vision under these circumstances is unclear.

The response of cones to various wavelengths of light is called their
spectral sensitivity. In normal human vision, the spectral sensitivity
of a cone falls into one of three subtypes, often called blue, green,
and red, but more accurately known as short, medium, and long
wavelength-sensitive cone subtypes. It is a lack of one or more of the
cone subtypes that causes individuals to have deficiencies in colour
vision or various kinds of colour blindness. These individuals are not
blind to objects of a particular colour, but are unable to distinguish
between colours that can be distinguished by people with normal vision.
Humans have this trichromatic vision, while most other mammals lack
cones with red sensitive pigment and therefore have poorer dichromatic
colour vision. However, some animals have four spectral subtypes,
e.g.~the trout adds an ultraviolet subgroup to short, medium, and long
subtypes that are similar to humans. Some fish are sensitive to the
polarization of light as well.

When thus excited by light, the photoceptor sends a proportional
response synaptically to bipolar cells which in turn signal the retinal
ganglion cells. The photoreceptors are also cross-linked by horizontal
cells and amacrine cells, which modify the synaptic signal before it
reaches the ganglion cells, the neural signals being intermixed and
combined. Of the retina's nerve cells, only the retinal ganglion cells
and few amacrine cells create action potentials.

In the retinal ganglion cells there are two types of response, depending
on the receptive field of the cell. The receptive fields of retinal
ganglion cells comprise a central, approximately circular area, where
light has one effect on the firing of the cell, and an annular surround,
where light has the opposite effect. In ON cells, an increment in light
intensity in the centre of the receptive field causes the firing rate to
increase. In OFF cells, it makes it decrease. Beyond this simple
difference, ganglion cells are also differentiated by chromatic
sensitivity and the type of spatial summation. Cells showing linear
spatial summation are termed X cells (also called parvocellular, P, or
midget ganglion cells), and those showing non-linear summation are Y
cells (also called magnocellular, M, or parasol retinal ganglion cells),
although the correspondence between X and Y cells (in the cat retina)
and P and M cells (in the primate retina) is not as simple as it once
seemed.

In the transfer of visual signals to the brain, the visual pathway, the
retina is vertically divided in two, a temporal (nearer to the temple)
half and a nasal (nearer to the nose) half. The axons from the nasal
half cross the brain at the optic chiasma to join with axons from the
temporal half of the other eye before passing into the lateral
geniculate body.

(ref:visual)
\href{https://commons.m.wikimedia.org/wiki/File:Human_visual_pathway.svg}{A
simplified schema} of the human visual pathway.

\texttt{\{r\ visualpathways,\ fig.cap=\textquotesingle{}(ref:visual)\textquotesingle{},\ echo=FALSE,\ message=FALSE,\ warning=FALSE\}\ knitr::include\_graphics("./figures/sensation/Human\_visual\_pathway.svg")}

Although there are more than 130 million retinal receptors, there are
only approximately 1.2 million fibres (axons) in the optic nerve. So, a
large amount of pre-processing is performed within the retina. The fovea
produces the most accurate information. Despite occupying about 0.01\%
of the visual field (less than 2° of visual angle), about 10\% of axons
in the optic nerve are devoted to the fovea.

The final result of all this processing is five different populations of
ganglion cells that send visual (image-forming and non-image-forming)
information to the brain:

\begin{itemize}
\tightlist
\item
  M cells, with large center-surround receptive fields that are
  sensitive to depth, indifferent to color, and rapidly adapt to a
  stimulus
\item
  P cells, with smaller center-surround receptive fields that are
  sensitive to color and shape
\item
  K cells, with very large center-only receptive fields that are
  sensitive to color and indifferent to shape or depth
\item
  another population that is intrinsically photosensitive
\item
  a final population that is involved in the control of eye movements.
  The neural retina consists of several layers of neurons interconnected
  by synapses, and is supported by an outer layer of pigmented
  epithelial cells. The primary light-sensing cells in the retina are
  the photoreceptor cells, which are of two types: rods and cones. Rods
  function mainly in dim light and provide black-and-white vision. Cones
  function in well-lit conditions and are responsible for the perception
  of colour, as well as high-acuity vision used for tasks such as
  reading. A third type of light-sensing cell, the photosensitive
  ganglion cell, is important for entrainment of circadian rhythms and
  reflexive responses such as the pupillary light reflex.
\end{itemize}

In vertebrate embryonic development, the retina and the optic nerve
originate as outgrowths of the developing brain, specifically the
embryonic diencephalon; thus, the retina is considered part of the
central nervous system (CNS) and is actually brain tissue. It is the
only part of the CNS that can be visualized non-invasively.

The vertebrate retina has ten distinct layers. From closest to farthest
from the vitreous body:

\begin{itemize}
\tightlist
\item
  Inner limiting membrane -- basement membrane elaborated by Müller
  cells.
\item
  Nerve fibre layer -- axons of the ganglion cell bodies (note that a
  thin layer of Müller cell footplates exists between this layer and the
  inner limiting membrane).
\item
  Ganglion cell layer -- contains nuclei of ganglion cells, the axons of
  which become the optic nerve fibres, and some displaced amacrine
  cells.
\item
  Inner plexiform layer -- contains the synapse between the bipolar cell
  axons and the dendrites of the ganglion and amacrine cells.
\item
  Inner nuclear layer -- contains the nuclei and surrounding cell bodies
  (perikarya) of the amacrine cells, bipolar cells, and horizontal
  cells.
\item
  Outer plexiform layer -- projections of rods and cones ending in the
  rod spherule and cone pedicle, respectively. These make synapses with
  dendrites of bipolar cells and horizontal cells. In the macular
  region, this is known as the Fiber layer of Henle.
\item
  Outer nuclear layer -- cell bodies of rods and cones.
\item
  External limiting membrane -- layer that separates the inner segment
  portions of the photoreceptors from their cell nuclei.
\item
  Inner segment / outer segment layer -- inner segments and outer
  segments of rods and cones. The outer segments contain a highly
  specialized light-sensing apparatus.
\item
  Retinal pigment epithelium -- single layer of cuboidal epithelial
  cells. This layer is closest to the choroid, and provides nourishment
  and supportive functions to the neural retina, The black pigment
  melanin in the pigment layer prevents light reflection throughout the
  globe of the eyeball.
\end{itemize}

These layers can be grouped into 4 main processing stages:
photoreception; transmission to bipolar cells; transmission to ganglion
cells, which also contain photoreceptors, the photosensitive ganglion
cells; and transmission along the optic nerve. At each synaptic stage
there are also laterally connecting horizontal and amacrine cells.

(ref:retina) Vertical section of the adult human retina. Carmine and
Nissl stain. \emph{A}, Photoreceptor layer. \emph{B}, Cell bodies of the
photoreceptors. \emph{C}, Outer plexiform layer. \emph{D}, Internal
granule layer. \emph{E}, Internal plexiform layer. \emph{F}, Ganglion
cell layer. \emph{G}, Ganglion cell axons. \emph{a}, external limiting
membrane. \emph{b}, internal limiting membrane. \emph{c}, Spherical
endfeet of the rod photoreceptors. \emph{d}, endfeet of the cones. e.
\emph{a}, cone. \emph{f}, a rod \emph{g}, horizontal cells. \emph{h},
amacrine cells. Fig. 188 from
\href{https://wellcomelibrary.org/item/b2129592x\#?c=0\&m=0\&s=0\&cv=0\&z=-0.9137\%2C-0.0887\%2C2.8274\%2C1.7747}{Histologie
du système nerveux de l'homme \& des vertébrés} (1909) by Santiago Ramón
y Cajal translated from Spanish by Dr.~L. Azoulay.

\texttt{\{r\ retcajal,\ fig.cap=\textquotesingle{}(ref:retina)\textquotesingle{},\ echo=FALSE,\ message=FALSE,\ warning=FALSE\}\ knitr::include\_graphics("./figures/sensation/retina\_cajal.png")}

The optic nerve is a central tract of many axons of ganglion cells
connecting primarily to the lateral geniculate body, a visual relay
station in the diencephalon (the rear of the forebrain). It also
projects to the superior colliculus, the suprachiasmatic nucleus, and
the nucleus of the optic tract. It passes through the other layers,
creating the optic disc in primates.

Additional structures, not directly associated with vision, are found as
outgrowths of the retina in some vertebrate groups. In birds, the pecten
is a vascular structure of complex shape that projects from the retina
into the vitreous humour; it supplies oxygen and nutrients to the eye,
and may also aid in vision. Reptiles have a similar, but much simpler,
structure.

In adult humans, the entire retina is approximately 72\% of a sphere
about 22 mm in diameter. The entire retina contains about 7 million
cones and 75 to 150 million rods. The optic disc, a part of the retina
sometimes called ``the blind spot'' because it lacks photoreceptors, is
located at the optic papilla, where the optic-nerve fibres leave the
eye. It appears as an oval white area of 3 mm². Temporal (in the
direction of the temples) to this disc is the macula, at whose centre is
the fovea, a pit that is responsible for our sharp central vision but is
actually less sensitive to light because of its lack of rods. Human and
non-human primates possess one fovea, as opposed to certain bird
species, such as hawks, who are bifoviate, and dogs and cats, who
possess no fovea but a central band known as the visual streak. Around
the fovea extends the central retina for about 6 mm and then the
peripheral retina. The farthest edge of the retina is defined by the ora
serrata.

In section, the retina is no more than 0.5 mm thick. It has three layers
of nerve cells and two of synapses, including the unique ribbon synapse.
The optic nerve carries the ganglion cell axons to the brain, and the
blood vessels that supply the retina. The ganglion cells lie innermost
in the eye while the photoreceptive cells lie beyond. Because of this
counter-intuitive arrangement, light must first pass through and around
the ganglion cells and through the thickness of the retina, before
reaching the rods and cones. Light is absorbed by the retinal pigment
epithelium or the choroid.

The white blood cells in the capillaries in front of the photoreceptors
can be perceived as tiny bright moving dots when looking into blue
light. This is known as the blue field entoptic phenomenon (or
Scheerer's phenomenon).

Between the ganglion cell layer and the rods and cones there are two
layers of neuropils where synaptic contacts are made. The neuropil
layers are the outer plexiform layer and the inner plexiform layer. In
the outer neuropil layer, the rods and cones connect to the vertically
running bipolar cells, and the horizontally oriented horizontal cells
connect to ganglion cells.

(ref:frogret) A semischematic diagram of the frog retina. a) green rods;
b (left) red rods; c) cone; i) horizontal cell; h) bipolar cell;
n,m,r,s,t) amacrine cells; o,p) ganglion cells; q) displaced amacrine
cell. A) Pigment epithelial cell with extended process; B) Pigment
epithelial cell with retracted process.

\texttt{\{r\ frogretina,\ fig.cap=\textquotesingle{}(ref:frogret)\textquotesingle{},\ echo=FALSE,\ message=FALSE,\ warning=FALSE\}\ knitr::include\_graphics("./figures/sensation/FrogRetinaCajalManual.jpg")}

The central retina predominantly contains cones, while the peripheral
retina predominantly contains rods. At the centre of the macula is the
foveal pit where the cones are narrow and long, and, arranged in a
hexagonal mosaic, the most dense, in contradistinction to the much
fatter cones located more peripherally in the retina. At the foveal pit
the other retinal layers are displaced, before building up along the
foveal slope until the rim of the fovea, or parafovea, is reached, which
is the thickest portion of the retina. The macula has a yellow
pigmentation and is known as the macula lutea. The area directly
surrounding the fovea has the highest density of rods converging on
single bipolar cells. Since its cones have a much lesser convergence of
signals, the fovea allows for the sharpest vision the eye can attain.

Though the rod and cones are a mosaic of sorts, transmission from
receptors, to bipolars, to ganglion cells is not direct. Since there are
about 150 million receptors and only 1 million optic nerve fibres, there
must be convergence and thus mixing of signals. Moreover, the horizontal
action of the horizontal and amacrine cells can allow one area of the
retina to control another (e.g.~one stimulus inhibiting another). This
inhibition is key to lessening the sum of messages sent to the higher
regions of the brain. In some lower vertebrates (e.g.~the pigeon), there
is a ``centrifugal'' control of messages -- that is, one layer can
control another, or higher regions of the brain can drive the retinal
nerve cells, but in primates this does not occur.

\hypertarget{the-photoreceptors}{%
\subsubsection{The Photoreceptors}\label{the-photoreceptors}}

A photoreceptor cell is a specialized type of neuroepithelial cell found
in the retina that is capable of visual phototransduction. The great
biological importance of photoreceptors is that they convert light
(visible electromagnetic radiation) into signals that can stimulate
biological processes. To be more specific, photoreceptor proteins in the
cell absorb photons, triggering a change in the cell's membrane
potential.

There are currently three known types of photoreceptor cells in
mammalian eyes: rods, cones, and intrinsically photosensitive retinal
ganglion cells. The two classic photoreceptor cells are rods and cones,
each contributing information used by the visual system to form a
representation of the visual world, sight. The rods are narrower than
the cones and distributed differently across the retina, but the
chemical process in each that supports phototransduction is similar. A
third class of mammalian photoreceptor cell was discovered during the
1990s: the intrinsically photosensitive retinal ganglion cells. These
cells do not contribute to sight directly, but are thought to support
circadian rhythms and pupillary reflex.

(ref:photo) Rods and cones from the human retina. A) a rod from the
peripheral retina; B) a cone from the peripheral retina; C) cones from
the fovea.

\texttt{\{r\ photoreceptors,\ fig.cap=\textquotesingle{}(ref:photo)\textquotesingle{},\ echo=FALSE,\ message=FALSE,\ warning=FALSE\}\ knitr::include\_graphics("./figures/sensation/PhotorecptorsCajalManual.jpg")}

There are major functional differences between the rods and cones. Rods
are extremely sensitive, and can be triggered by a single photon. At
very low light levels, visual experience is based solely on the rod
signal.

Cones require significantly brighter light (that is, a larger number of
photons) to produce a signal. In humans, there are three different types
of cone cell, distinguished by their pattern of response to light of
different wavelengths. Color experience is calculated from these three
distinct signals. This explains why colors cannot be seen at low light
levels, when only the rod and not the cone photoreceptor cells are
active. The three types of cone cell respond (roughly) to light of
short, medium, and long wavelengths, so they may respectively be
referred to as S-cones, M-cones, and L-cones. The different responses of
the three types of cone cells are determined by the likelihoods that
their respective photoreceptor proteins will absorb photons of different
wavelengths. So, for example, an L cone cell contains a photoreceptor
protein that more readily absorbs long wavelengths of light (that is,
more ``red''). Light of a shorter wavelength can also produce the same
response, but it must be much brighter to do so.

The number and ratio of rods to cones varies among species, dependent on
whether an animal is primarily diurnal or nocturnal.

\hypertarget{visual-phototransduction}{%
\subsubsection{Visual
Phototransduction}\label{visual-phototransduction}}

Visual phototransduction is the sensory transduction of the visual
system. It is a process by which light is converted into electrical
signals in the rod cells, cone cells and photosensitive ganglion cells
of the retina of the eye. This cycle was elucidated by
\href{https://en.wikipedia.org/wiki/George_Wald}{George Wald}
(1906--1997) for which he received the Nobel Prize in 1967.

The visual cycle is the biological conversion of a photon into an
electrical signal in the retina. This process occurs via G-protein
coupled receptors called opsins which contain the chromophore 11-cis
retinal. 11-cis retinal is covalently linked to the opsin receptor via
Schiff base forming retinylidene protein. When struck by a photon,
11-cis retinal undergoes photoisomerization to all-trans retinal which
changes the conformation of the opsin GPCR leading to signal
transduction cascades which causes closure of cyclic GMP-gated cation
channel, and hyperpolarization of the photoreceptor cell.

Following isomerization and release from the opsin protein, all-trans
retinal is reduced to all-trans retinol and travels back to the retinal
pigment epithelium to be ``recharged''. It is first esterified by
lecithin retinol acyltransferase (LRAT) and then converted to 11-cis
retinol by the isomerohydrolase RPE65. Finally, it is oxidized to 11-cis
retinal before traveling back to the rod outer segment where it is again
conjugated to an opsin to form new, functional visual pigment
(rhodopsin).

To understand the photoreceptor's behaviour to light intensities, it is
necessary to understand the roles of different currents.

(ref:retinal)
\href{https://commons.wikimedia.org/wiki/File:Visual_cycle.svg}{The
chemical reactions involved in the photoreceptor visual cycle.}

\texttt{\{r\ absorption,\ fig.cap=\textquotesingle{}(ref:retinal)\textquotesingle{},\ echo=FALSE,\ message=FALSE,\ warning=FALSE\}\ knitr::include\_graphics("./figures/sensation/Visual\_cycle.svg")}

There is an ongoing outward potassium current through nongated
K\textsuperscript{+}-selective channels. This outward current tends to
hyperpolarize the photoreceptor at around -70 mV (the equilibrium
potential for K\textsuperscript{+}).

There is also an inward sodium current carried by cGMP-gated sodium
channels. This so-called `dark current' depolarizes the cell to around
-40 mV. Note that this is significantly more depolarized than most other
neurons.

A high density of Na\textsuperscript{+}-K\textsuperscript{+} pumps
enables the photoreceptor to maintain a steady intracellular
concentration of Na\textsuperscript{+} and K\textsuperscript{+}.

Photoreceptor cells are unusual cells in that they are depolarized under
scotopic conditions (darkness). In photopic conditions (light),
photoreceptors are hyperpolarized to a potential of -60mV.

In the dark, cGMP levels are high and keep cGMP-gated sodium channels
open allowing a steady inward current, called the dark current. This
dark current keeps the cell depolarized at about -40 mV, leading to
glutamate release.

The depolarization of the cell membrane in scotopic conditions opens
voltage-gated calcium channels. An increased intracellular concentration
of Ca\textsuperscript{2+} causes vesicles containing glutamate, the
photoreceptor neurotransmitter, to merge with the cell membrane,
therefore releasing glutamate.

In the cone pathway glutamate

\begin{itemize}
\tightlist
\item
  Hyperpolarizes on-center bipolar cells. Glutamate that is released
  from the photoreceptors in the dark binds to metabotropic glutamate
  receptors (mGluR6), which, through a G-protein coupling mechanism,
  causes non-specific cation channels in the cells to close, thus
  hyperpolarizing the bipolar cell.
\item
  Depolarizes off-center bipolar cells. Binding of glutamate to
  ionotropic glutamate receptors results in an inward cation current
  that depolarizes the bipolar cell.
\end{itemize}

Activation of the phototransduction cascade

\begin{enumerate}
\def\labelenumi{\arabic{enumi}.}
\tightlist
\item
  A light photon interacts with the retinal in a photoreceptor cell. The
  retinal undergoes isomerisation, changing from the 11-cis to all-trans
  configuration.
\item
  Opsin therefore undergoes a conformational change to metarhodopsin II.
\item
  Metarhodopsin II activates a G protein known as transducin. This
  causes transducin to dissociate from its bound GDP, and bind GTP, then
  the alpha subunit of transducin dissociates from the beta and gamma
  subunits, with the GTP still bound to the alpha subunit.
\item
  The alpha subunit-GTP complex activates phosphodiesterase, also known
  as PDE6. It binds to one of two regulatory subunits of PDE (which
  itself is a tetramer) and inhibits its activity.
\item
  PDE hydrolyzes cGMP, forming GMP. This lowers the intracellular
  concentration of cGMP and therefore the sodium channels close.
\item
  Closure of the sodium channels causes hyperpolarization of the cell
  due to the ongoing efflux of potassium ions.
\item
  Hyperpolarization of the cell causes voltage-gated calcium channels to
  close.
\item
  As the calcium level in the photoreceptor cell drops, the amount of
  the neurotransmitter glutamate that is released by the cell also
  drops. This is because calcium is required for the
  glutamate-containing vesicles to fuse with cell membrane and release
  their contents (see SNARE proteins).
\item
  A decrease in the amount of glutamate released by the photoreceptors
  causes depolarization of on-center bipolar cells (rod and cone On
  bipolar cells) and hyperpolarization of cone off-center bipolar cells.
\end{enumerate}

(ref:steps)
\href{https://commons.wikimedia.org/wiki/File:Phototransduction.png}{Representation}
of molecular steps in photoactivation (modified from Leskov et al.,
2000). Depicted is an outer membrane disk in a rod. Step 1: Incident
photon (hν) is absorbed and activates a rhodopsin by conformational
change in the disk membrane to R\emph{. Step 2: Next, R} makes repeated
contacts with transducin molecules, catalyzing its activation to G* by
the release of bound GDP in exchange for cytoplasmic GTP, which expels
its β and γ subunits. Step 3: G* binds inhibitory γ subunits of the
phosphodiesterase (PDE) activating its α and β subunits. Step 4:
Activated PDE hydrolyzes cGMP. Step 5: Guanylyl cyclase (GC) synthesizes
cGMP, the second messenger in the phototransduction cascade. Reduced
levels of cytosolic cGMP cause cyclic nucleotide gated channels to close
preventing further influx of Na\textsuperscript{+} and
Ca\textsuperscript{2+}.

\texttt{\{r\ molsteps,\ fig.cap=\textquotesingle{}(ref:steps)\textquotesingle{},\ echo=FALSE,\ message=FALSE,\ warning=FALSE\}\ knitr::include\_graphics("./figures/sensation/Phototransduction.png")}

Deactivation of the phototransduction cascade

In light, low cGMP levels close Na\textsuperscript{+} and
Ca\textsuperscript{2+} channels, reducing intracellular
Na\textsuperscript{+} and Ca\textsuperscript{2+}. During recovery (dark
adaptation), the low Ca\textsuperscript{2+} levels induce recovery
(termination of the phototransduction cascade), as follows:

\begin{enumerate}
\def\labelenumi{\arabic{enumi}.}
\tightlist
\item
  Low intracellular Ca\textsuperscript{2+} makes intracellular Ca-GCAP
  (Ca-Guanylate cyclase activating protein) dissociate into
  Ca\textsuperscript{2+} and GCAP. The liberated GCAP ultimately
  restores depleted cGMP levels, which re-opens the cGMP-gated cation
  channels (restoring dark current).
\item
  Low intracellular Ca\textsuperscript{2+} makes intracellular Ca-GAP
  (Ca-GTPase Accelerating Protein) dissociate into
  Ca\textsuperscript{2+} and GAP. The liberated GAP deactivates
  activated-transducin, terminating the phototransduction cascade
  (restoring dark current).
\item
  Low intracellular Ca\textsuperscript{2+} makes intracellular
  Ca-recoverin-RK dissociate into Ca\textsuperscript{2+} and recoverin
  and RK. The liberated RK then phosphorylates metarhodopsin II,
  reducing its binding affinity for transducin. Arrestin then completely
  deactivates the phosphorylated-metarhodopsin II, terminating the
  phototransduction cascade (restoring dark current).
\item
  Low intracellular Ca\textsuperscript{2+} make the
  Ca\textsuperscript{2+}/calmodulin complex within the cGMP-gated cation
  channels more sensitive to low cGMP levels (thereby, keeping the
  cGMP-gated cation channel open even at low cGMP levels, restoring dark
  current)
\end{enumerate}

All-trans retinal cannot be synthesised by humans and must be supplied
by vitamin A in the diet. Deficiency of all-trans retinal can lead to
night blindness. This is part of the bleach and recycle process of
retinoids in the photoreceptors and retinal pigment epithelium.

Photoreceptor cells are typically arranged in an irregular but
approximately hexagonal grid, known as the retinal mosaic.

The opsin found in the intrinsically photosensitive ganglion cells of
the retina is called melanopsin. These cells are involved in various
reflexive responses of the brain and body to the presence of (day)light,
such as the regulation of circadian rhythms, pupillary reflex and other
non-visual responses to light. Melanopsin functionally resembles
invertebrate opsins.

\hypertarget{the-visual-pathways}{%
\subsubsection{The Visual Pathways}\label{the-visual-pathways}}

\hypertarget{the-optic-nerve-and-optic-tract}{%
\subsubsection{The Optic Nerve And Optic
Tract}\label{the-optic-nerve-and-optic-tract}}

The optic nerve conducts the action potentials generated by the retinal
ganglion cells through the optic canal to the subsequent processing
centers in the brain. Upon reaching the optic chiasm the nerve fibers
from the nasal part of the retina in each eye cross over to the other
side (decussate). The fibers then branch and terminate in three places.

The optic nerve is composed of retinal ganglion cell axons and glial
cells. Each human optic nerve contains between 770,000 and 1.7 million
nerve fibers, which are axons of the retinal ganglion cells of one
retina.

In humans, the optic nerve is derived from optic stalks during the
seventh week of development. It extends from the optic disc to the optic
chiasma and continues as the optic tract to the lateral geniculate
nucleus, pretectal nuclei, and superior colliculus.

Most of the axons of the optic nerve terminate in the lateral geniculate
nucleus from where information is relayed to the visual cortex, while
other axons terminate in the pretectal nucleus and are involved in
reflexive eye movements. Other axons terminate in the suprachiasmatic
nucleus and are involved in regulating the sleep-wake cycle. Its
diameter increases from about 1.6 mm within the eye to 3.5 mm in the
orbit to 4.5 mm within the cranial space.

\hypertarget{the-superior-colliculus}{%
\subsubsection{The Superior Colliculus}\label{the-superior-colliculus}}

The superior colliculus (Latin, upper hill) is a structure lying on the
roof of the mammalian midbrain. In non-mammalian vertebrates the
homologous structure, is known as the optic tectum or optic lobe.

In mammals the superior colliculus forms a major component of the
midbrain. It is a paired structure and together with the paired inferior
colliculi form the corpora quadrigemina (from Latin quadruplet bodies).
The superior colliculus is a layered structure, with a number of layers
that varies by species. The layers can be grouped into the superficial
layers (stratum opticum and above) and the deeper remaining layers.
Neurons in the superficial layers receive direct input from the retina
and respond almost exclusively to visual stimuli. Many neurons in the
deeper layers also respond to other modalities, and some respond to
stimuli in multiple modalities. The deeper layers also contain a
population of motor-related neurons, capable of activating eye movements
as well as other responses.

The general function of the tectal system is to direct behavioral
responses toward specific points in egocentric (``body-centered'')
space. Each layer contains a topographic map of the surrounding world in
retinotopic coordinates, and activation of neurons at a particular point
in the map evokes a response directed toward the corresponding point in
space. In primates, the superior colliculus has been studied mainly with
respect to its role in directing eye movements. Visual input from the
retina, or ``command'' input from the cerebral cortex, create a ``bump''
of activity in the tectal map, which, if strong enough, induces a
saccadic eye movement. Even in primates, however, the superior
colliculus is also involved in generating spatially directed head turns,
arm-reaching movements, and shifts in attention that do not involve any
overt movements. In mammals, and especially primates, the massive
expansion of the cerebral cortex reduces the superior colliculus to a
much smaller fraction of the whole brain. It remains nonetheless
important in terms of function as the primary integrating center for eye
movements.

Behavioral studies have shown that the SC is not needed for object
recognition, but plays a critical role in the ability to direct
behaviors toward specific objects, and can support this ability even in
the absence of the cerebral cortex. Thus, cats with major damage to the
visual cortex cannot recognize objects, but may still be able to follow
and orient toward moving stimuli, although more slowly than usual. If
one half of the SC is removed, however, the cats will circle constantly
toward the side of the lesion, and orient compulsively toward objects
located there, but fail to orient at all toward objects located in the
opposite hemifield. These deficits diminish over time but never
disappear.

In primates, eye movements can be divided into several types: fixation,
in which the eyes are directed toward a motionless object, with eye
movements only to compensate for movements of the head; smooth pursuit,
in which the eyes move steadily to track a moving object; saccades, in
which the eyes move very rapidly from one location to another; and
vergence, in which the eyes move simultaneously in opposite directions
to obtain or maintain single binocular vision. The superior colliculus
is involved in all of these, but its role in saccades has been studied
most intensively.

The output from the motor sector of the SC goes to a set of midbrain and
brainstem nuclei, which transform the ``place'' code used by the SC into
the ``rate'' code used by oculomotor neurons. Eye movements are
generated by six muscles, arranged in three orthogonally-aligned pairs.
Thus, at the level of the final common path, eye movements are encoded
in essentially a Cartesian coordinate system.

Although the SC receives a strong input directly from the retina, in
primates it is largely under the control of the cerebral cortex, which
contains several areas that are involved in determining eye movements.
The frontal eye fields, a portion of the motor cortex, are involved in
triggering intentional saccades, and an adjoining area, the
supplementary eye fields, are involved in organizing groups of saccades
into sequences. The parietal eye fields, farther back in the brain, are
involved mainly in reflexive saccades, made in response to changes in
the view.

The SC only receives visual inputs in its superficial layers, whereas
the deeper layers of the colliculus receive also auditory and
somatosensory inputs and are connected to many sensorimotor areas of the
brain. The colliculus as a whole is thought to help orient the head and
eyes toward something seen and heard.

The superior colliculus also receives auditory information from the
inferior colliculus. This auditory information is integrated with the
visual information already present to produce the ventriloquist effect.

\hypertarget{the-lateral-geniculate-nucleus-lgn}{%
\subsubsection{The Lateral Geniculate Nucleus
(LGN)}\label{the-lateral-geniculate-nucleus-lgn}}

The lateral geniculate nucleus (LGN; also called the lateral geniculate
body or lateral geniculate complex; named after its resemblance to a
bent knee) is a relay center in the thalamus for the visual pathway. It
receives a major sensory input from the retina. The LGN is the main
central connection for the optic nerve to the occipital lobe,
particularly the primary visual cortex. In humans, each LGN has six
layers of neurons (grey matter) alternating with optic fibers (white
matter).

The LGN is a small, ovoid, ventral projection at the termination of the
optic tract on each side of the brain. The LGN and the medial geniculate
nucleus which deals with auditory information are both thalamic nuclei
and so are present in both hemispheres.

The LGN receives information directly from the ascending retinal
ganglion cells via the optic tract and from the reticular activating
system. Neurons of the LGN send their axons through the optic radiation,
a direct pathway to the primary visual cortex. In addition, the LGN
receives many strong feedback connections from the primary visual
cortex. In humans as well as other mammals, the two strongest pathways
linking the eye to the brain are those projecting to the dorsal part of
the LGN in the thalamus, and to the superior colliculus.

In humans as well as in many other primates, the LGN has layers of
magnocellular cells and parvocellular cells that are interleaved with
layers of koniocellular cells. In humans the LGN is normally described
as having six distinctive layers. The inner two layers, (1 and 2) are
magnocellular layers, while the outer four layers, (3,4,5 and 6), are
parvocellular layers. An additional set of neurons, known as the
koniocellular layers, are found ventral to each of the magnocellular and
parvocellular layers.

The magnocellular, parvocellular, and koniocellular layers of the LGN
correspond with the similarly named types of retinal ganglion cells.
Retinal P ganglion cells send axons to a parvocellular layer, M ganglion
cells send axons to a magnocellular layer, and K ganglion cells send
axons to a koniocellular layer.:269

Koniocellular cells are functionally and neurochemically distinct from M
and P cells and provide a third channel to the visual cortex. They
project their axons between the layers of the lateral geniculate nucleus
where M and P cells project. Their role in visual perception is
presently unclear; however, the koniocellular system has been linked
with the integration of somatosensory system-proprioceptive information
with visual perception, and it may also be involved in color perception.

The other major retino--cortical visual pathway is the tectopulvinar
pathway, routing primarily through the superior colliculus and thalamic
pulvinar nucleus onto posterior parietal cortex and visual area MT.

Ipsilateral and contralateral layers

Layer 1, 2

\begin{itemize}
\tightlist
\item
  Large cells, called magnocellular pathways
\item
  Input from M-ganglion cells
\item
  Very rapid conduction
\item
  Colour blind system
\end{itemize}

Layer 3--6

\begin{itemize}
\tightlist
\item
  Parvocellular
\item
  Input from P-ganglion cells
\item
  Colour vision
\item
  Moderate velocity.
\end{itemize}

Both the LGN in the right hemisphere and the LGN in the left hemisphere
receive input from each eye. However, each LGN only receives information
from one half of the visual field. This occurs due to axons of the
ganglion cells from the inner halves of the retina (the nasal sides)
decussating (crossing to the other side of the brain) through the optic
chiasma (khiasma means ``cross-shaped''). The axons of the ganglion
cells from the outer half of the retina (the temporal sides) remain on
the same side of the brain. Therefore, the right hemisphere receives
visual information from the left visual field, and the left hemisphere
receives visual information from the right visual field. Within one LGN,
the visual information is divided among the various layers as follows:

\begin{itemize}
\tightlist
\item
  the eye on the same side (the ipsilateral eye) sends information to
  layers 2, 3 and 5
\item
  the eye on the opposite side (the contralateral eye) sends information
  to layers 1, 4 and 6.
\end{itemize}

This description applies to the LGN of many primates, but not all.

The principal neurons in the LGN receive strong inputs from the retina.
However, the retina only accounts for a small percentage of LGN input.
As much as 95\% of input in the LGN comes from the visual cortex,
superior colliculus, pretectum, thalamic reticular nuclei, and local LGN
interneurons. Regions in the brainstem that are not involved in visual
perception also project to the LGN, such as the mesencephalic reticular
formation, dorsal raphe nucleus, periaqueductal grey matter, and the
locus coeruleus. These non-retinal inputs can be excitatory, inhibitory,
or modulatory.

Information leaving the LGN travels out on the optic radiations, which
form part of the retrolenticular portion of the internal capsule.

The axons that leave the LGN go to V1 visual cortex. Both the
magnocellular layers 1--2 and the parvocellular layers 3--6 send their
axons to layer 4 in V1. Within layer 4 of V1, layer 4cβ receives
parvocellular input, and layer 4cα receives magnocellular input.
However, the koniocellular layers, intercalated between LGN layers 1--6
send their axons primarily to the cytochrome-oxidase rich blobs of
layers 2 and 3 in V1. Axons from layer 6 of visual cortex send
information back to the LGN.

Studies involving blindsight have suggested that projections from the
LGN travel not only to the primary visual cortex but also to higher
cortical areas V2 and V3. Patients with blindsight are phenomenally
blind in certain areas of the visual field corresponding to a
contralateral lesion in the primary visual cortex; however, these
patients are able to perform certain motor tasks accurately in their
blind field, such as grasping. This suggests that neurons travel from
the LGN to both the primary visual cortex and higher cortex regions.

\hypertarget{the-visual-cortex}{%
\subsubsection{The Visual Cortex}\label{the-visual-cortex}}

The visual cortex of the brain is that part of the cerebral cortex which
processes visual information. It is located in the occipital lobe.

The visual cortex is the largest system in the human brain and is
responsible for processing the visual information. The region that
receives information directly from the LGN is called the primary visual
cortex, (also called V1 and striate cortex). The primary visual cortex
is the most studied visual area in the brain. Visual information then
flows through a cortical hierarchy. These areas include V2, V3, V4 and
area V5/MT (the exact connectivity depends on the species of the
animal).

As visual information passes forward through the visual hierarchy, the
complexity of the neural representations increases. Whereas a V1 neuron
may respond selectively to a line segment of a particular orientation in
a particular retinotopic location, neurons in the lateral occipital
complex respond selectively to complete object (e.g., a figure drawing),
and neurons in visual association cortex may respond selectively to
human faces, or to a particular object.

(ref:viscort)
\href{https://commons.wikimedia.org/wiki/File:Visual_field_maps.jpg}{A
visual field map} of the primary visual cortex and the numerous
extrastriate areas.

\texttt{\{r\ visualcortex,\ fig.cap=\textquotesingle{}(ref:)\textquotesingle{},\ echo=FALSE,\ message=FALSE,\ warning=FALSE\}\ knitr::include\_graphics("./figures/sensation/Visual\_field\_maps.jpg")}

Along with this increasing complexity of neural representation may come
a level of specialization of processing into two distinct pathways: the
dorsal stream and the ventral stream (the Two Streams hypothesis, first
proposed by Ungerleider and Mishkin in 1982). The dorsal stream,
commonly referred to as the ``where'' stream, is involved in spatial
attention (covert and overt), and communicates with regions that control
eye movements and hand movements. More recently, this area has been
called the ``how'' stream to emphasize its role in guiding behaviors to
spatial locations. The ventral stream, commonly referred as the ``what''
stream, is involved in the recognition, identification and
categorization of visual stimuli.

However, there is still much debate about the degree of specialization
within these two pathways, since they are in fact heavily
interconnected.

Visual information coming from the eye goes through the lateral
geniculate nucleus in the thalamus and then reaches the visual cortex.
The part of the visual cortex that receives the sensory inputs from the
thalamus is the primary visual cortex, also known as visual area 1 (V1,
Brodmann area 17), and the striate cortex. The extrastriate areas
consist of visual areas 2 (V2, Brodmann area 18), 3, 4, and 5 (V3, V4,
V5, all Brodmann area 19).

The primary visual cortex (V1) is located in and around the calcarine
fissure in the occipital lobe. Each hemisphere's V1 receives information
directly from its ipsilateral lateral geniculate nucleus that receives
signals from the contralateral visual hemifield.

Neurons in the visual cortex fire action potentials when visual stimuli
appear within their receptive field. By definition, the receptive field
is the region within the entire visual field that elicits an action
potential. But, for any given neuron, it may respond best to a subset of
stimuli within its receptive field. This property is called neuronal
tuning. In the earlier visual areas, neurons have simpler tuning. For
example, a neuron in V1 may fire to any vertical stimulus in its
receptive field. In the higher visual areas, neurons have complex
tuning. For example, in the inferior temporal cortex (IT), a neuron may
fire only when a certain face appears in its receptive field.

The visual cortex receives its blood supply primarily from the calcarine
branch of the posterior cerebral artery.

V1 transmits information to two primary pathways, called the ventral
stream and the dorsal stream. The ventral stream begins with V1, goes
through visual area V2, then through visual area V4, and to the inferior
temporal cortex (IT cortex). The ventral stream, sometimes called the
``What Pathway'', is associated with form recognition and object
representation. It is also associated with storage of long-term memory.
The dorsal stream begins with V1, goes through Visual area V2, then to
the dorsomedial area (DM/V6) and medial temporal area (MT/V5) and to the
posterior parietal cortex. The dorsal stream, sometimes called the
``Where Pathway'' or ``How Pathway'', is associated with motion,
representation of object locations, and control of the eyes and arms,
especially when visual information is used to guide saccades or
reaching.

\hypertarget{the-auditory-and-vestibular-systems}{%
\subsection{The Auditory And Vestibular
Systems}\label{the-auditory-and-vestibular-systems}}

The auditory system is the sensory system for the sense of hearing. It
includes both the sensory organs (the ears) and the auditory parts of
the sensory system. Hearing, or auditory perception, is the ability to
perceive sounds by detecting vibrations, changes in the pressure of the
surrounding medium over time, through the ear.

Providing balance, when moving or stationary, is also a central function
of the ear. The ear facilitates two types of balance: static balance,
which allows a person to feel the effects of gravity, and dynamic
balance, which allows a person to sense acceleration.

\hypertarget{the-ear}{%
\subsubsection{The Ear}\label{the-ear}}

In mammals, the ear is usually described as having three parts---the
outer ear, the middle ear and the inner ear. The outer ear consists of
the pinna and the ear canal. The folds of cartilage surrounding the ear
canal are called the pinna. Sound waves are reflected and attenuated
when they hit the pinna, and these changes provide additional
information that will help the brain determine the sound direction.
Since the outer ear is the only visible portion of the ear in most
animals, the word ``ear'' often refers to the external part alone. The
middle ear includes the tympanic cavity and the three ossicles. The
inner ear sits in the bony labyrinth, and contains structures which are
key to several senses: the semicircular canals, which enable balance and
eye tracking when moving; the utricle and saccule, which enable balance
when stationary; and the cochlea, which enables hearing. The ears of
vertebrates are placed somewhat symmetrically on either side of the
head, an arrangement that aids sound localisation.

The ear develops from the first pharyngeal pouch and six small swellings
that develop in the early embryo called otic placodes, which are derived
from ectoderm.

The ear canal of the outer ear is separated from the air-filled tympanic
cavity of the middle ear by the eardrum. The middle ear contains the
three small bones---the ossicles---involved in the transmission of
sound, and is connected to the throat at the nasopharynx, via the
pharyngeal opening of the Eustachian tube. The inner ear contains the
otolith organs---the utricle and saccule---and the semicircular canals
belonging to the vestibular system, as well as the cochlea of the
auditory system.

(ref:ear) Front view of the right outer, middle and inner human ear.
\href{https://archive.org/details/anatomydescripti00grayuoft/page/n6/mode/2up}{Gray,
Henry, 1825-1861. Anatomy, descriptive and surgical; ed.~by T. Pickering
Pick and Robert Howden. A revised American, from the fifteenth English
edition. Philadelphia, Lea, 1901}

\texttt{\{r\ rightear,\ fig.cap=\textquotesingle{}(ref:ear)\textquotesingle{},\ echo=FALSE,\ message=FALSE,\ warning=FALSE\}\ knitr::include\_graphics("./figures/sensation/GrayEar.jpg")}

Sound waves travel through the ear canal and hit the tympanic membrane,
or eardrum. This wave information travels across the air-filled middle
ear cavity via a series of delicate bones: the malleus (hammer), incus
(anvil) and stapes (stirrup). These ossicles act as a lever, converting
the lower-pressure eardrum sound vibrations into higher-pressure sound
vibrations at another, smaller membrane called the oval window or
vestibular window. The manubrium (handle) of the malleus articulates
with the tympanic membrane, while the footplate (base) of the stapes
articulates with the oval window. Higher pressure is necessary at the
oval window than at the typanic membrane because the inner ear beyond
the oval window contains liquid rather than air. The stapedius reflex of
the middle ear muscles helps protect the inner ear from damage by
reducing the transmission of sound energy when the stapedius muscle is
activated in response to sound. The middle ear still contains the sound
information in wave form; it is converted to nerve impulses in the
cochlea. The middle-ear ossicles further amplify the vibration pressure
roughly 20 times. The base of the stapes couples vibrations into the
cochlea via the oval window, which vibrates the perilymph liquid
(present throughout the inner ear) and causes the round window to bulb
out as the oval window bulges in.

The inner ear consists of the cochlea and several non-auditory
structures. The cochlea has three fluid-filled sections (i.e.~the scala
media, scala tympani and scala vestibuli), and supports a fluid wave
driven by pressure across the basilar membrane separating two of the
sections. Strikingly, one section, called the cochlear duct or scala
media, contains endolymph. Endolymph is a fluid similar in composition
to the intracellular fluid found inside cells. The organ of Corti is
located in this duct on the basilar membrane, and transforms mechanical
waves to electric signals in neurons. The other two sections are known
as the scala tympani and the scala vestibuli. These are located within
the bony labyrinth, which is filled with fluid called perilymph, similar
in composition to cerebrospinal fluid. The chemical difference between
the fluids endolymph and perilymph fluids is important for the function
of the inner ear.

\hypertarget{the-auditory-system}{%
\subsection{The Auditory System}\label{the-auditory-system}}

In humans and other vertebrates, hearing is performed primarily by the
auditory system: mechanical waves, known as vibrations, are detected by
the ear and transduced into nerve impulses that are perceived by the
brain (primarily in the temporal lobe). Like touch, audition requires
sensitivity to the movement of molecules in the world outside the
organism. Both hearing and touch are types of mechanosensation. Sound
may be heard through solid, liquid, or gaseous matter. It is one of the
traditional five senses; partial or total inability to hear is called
hearing loss.

\hypertarget{organ-of-corti}{%
\subsubsection{Organ Of Corti}\label{organ-of-corti}}

The organ of Corti, or spiral organ, is the receptor organ for hearing
and is located in the mammalian cochlea. This highly varied strip of
epithelial cells allows for transduction of auditory signals into nerve
impulses. Transduction occurs through vibrations of structures in the
inner ear causing displacement of cochlear fluid and movement of hair
cells at the organ of Corti to produce electrochemical signals.

Italian anatomist
\href{https://en.wikipedia.org/wiki/Alfonso_Giacomo_Gaspare_Corti}{Alfonso
Giacomo Gaspare Corti} (1822--1876) discovered the organ of corti in
1851.

The organ of corti is located in the scala media of the cochlea of the
inner ear between the vestibular duct and the tympanic duct and is
composed of mechanosensory cells, known as hair cells. strategically
positioned on the basilar membrane of the organ of corti are three rows
of outer hair cells (ohcs) and one row of inner hair cells (ihcs).
Separating these hair cells are supporting cells: Deiters cells, also
called phalangeal cells, which separate and support both the ohcs and
the ihcs.

(ref:corti)
\href{https://commons.wikimedia.org/wiki/File:Cochlea-crosssection.svg}{A
cross section of the cochlea illustrating the organ of Corti.}

\texttt{\{r\ organcorti,\ fig.cap=\textquotesingle{}(ref:corti)\textquotesingle{},\ echo=FALSE,\ message=FALSE,\ warning=FALSE\}\ knitr::include\_graphics("./figures/sensation/Cochlea-crosssection.svg")}

Projecting from the tips of the hair cells are tiny finger like
projections called stereocilia, which are arranged in a graduated
fashion with the shortest stereocilia on the outer rows and the longest
in the center.

If the cochlea were uncoiled it would roll out to be about 33 mm long in
women and 34 mm in men, with about 2.28 mm of standard deviation for the
population. The cochlea is also tonotopically organized, meaning that
different frequencies of sound waves interact with different locations
on the structure. The base of the cochlea, closest to the outer ear, is
the most stiff and narrow and is where the high frequency sounds are
transduced. The apex, or top, of the cochlea is wider and much more
flexible and loose and functions as the transduction site for low
frequency sounds.

\hypertarget{auditory-transduction}{%
\subsubsection{Auditory Transduction}\label{auditory-transduction}}

In normal hearing subjects, the majority of the auditory signals that
reach the organ of Corti in the first place come from the outer ear.
Sound waves enter through the auditory canal and vibrate the tympanic
membrane, also known as the eardrum, which vibrates three small bones
called the ossicles. As a result, the attached oval window moves and
causes movement of the round window, which leads to displacement of the
cochlear fluid. However, the stimulation can happen also via direct
vibration of the cochlea from the skull. The latter is referred to as
Bone Conduction (or BC) hearing, as complementary to the first one
described, which is instead called Air Conduction (or AC) hearing. Both
AC and BC stimulate the basilar membrane in the same way.

The basilar membrane on the tympanic duct presses against the hair cells
of the organ as perilymphatic pressure waves pass. The stereocilia atop
the IHCs move with this fluid displacement and in response their cation,
or positive ion selective, channels are pulled open by cadherin
structures called tip links that connect adjacent stereocilia. The organ
of Corti, surrounded in potassium rich fluid endolymph, lies on the
basilar membrane at the base of the scala media. Under the organ of
Corti is the scala tympani and above it, the scala vestibuli. Both
structures exist in a low potassium fluid called perilymph. Because
those stereocilia are in the midst of a high concentration of potassium,
once their cation channels are pulled open, potassium ions as well as
calcium ions flow into the top of the hair cell. With this influx of
positive ions the IHC becomes depolarized, opening voltage-gated calcium
channels at the basolateral region of the hair cells and triggering the
release of the neurotransmitter glutamate. An electrical signal is then
sent through the auditory nerve and into the auditory cortex of the
brain as a neural message.

The organ of Corti is also capable of modulating the auditory signal.
The outer hair cells (OHCs) can amplify the signal through a process
called electromotility where they increase movement of the basilar and
tectorial membranes and therefore increase deflection of stereocilia in
the IHCs.

A crucial piece to this cochlear amplification is the motor protein
prestin, which changes shape based on the voltage potential inside of
the hair cell. When the cell is depolarized, prestin shortens, and
because it is located on the membrane of OHCs it then pulls on the
basilar membrane and increasing how much the membrane is deflected,
creating a more intense effect on the inner hair cells (IHCs). When the
cell hyperpolarizes prestin lengthens and eases tension on the IHCs,
which decreases the neural impulses to the brain. In this way, the hair
cell itself is able to modify the auditory signal before it even reaches
the brain.

Hair cells are columnar cells, each with a bundle of 100--200
specialized cilia at the top, for which they are named. There are two
types of hair cells; inner and outer hair cells. Inner hair cells are
the mechanoreceptors for hearing: they transduce the vibration of sound
into electrical activity in nerve fibers, which is transmitted to the
brain. Outer hair cells are a motor structure. Sound energy causes
changes in the shape of these cells, which serves to amplify sound
vibrations in a frequency specific manner. Lightly resting atop the
longest cilia of the inner hair cells is the tectorial membrane, which
moves back and forth with each cycle of sound, tilting the cilia, which
is what elicits the hair cells' electrical responses.

Inner hair cells, like the photoreceptor cells of the eye, show a graded
response, instead of the spikes typical of other neurons.

\hypertarget{auditory-pathways}{%
\subsubsection{Auditory Pathways}\label{auditory-pathways}}

Afferent neurons innervate cochlear inner hair cells, at synapses where
the neurotransmitter glutamate communicates signals from the hair cells
to the dendrites of the primary auditory neurons.

There are far fewer inner hair cells in the cochlea than afferent nerve
fibers -- many auditory nerve fibers innervate each hair cell. The
neural dendrites belong to neurons of the auditory nerve, which in turn
joins the vestibular nerve to form the vestibulocochlear nerve, or
cranial nerve number VIII. The region of the basilar membrane supplying
the inputs to a particular afferent nerve fibre can be considered to be
its receptive field.

Efferent projections from the brain to the cochlea also play a role in
the perception of sound, although this is not well understood. Efferent
synapses occur on outer hair cells and on afferent (towards the brain)
dendrites under inner hair cells

\hypertarget{the-cochlear-nucleus}{%
\subsubsection{The Cochlear Nucleus}\label{the-cochlear-nucleus}}

The cochlear nucleus is the first site of the neuronal processing of the
newly converted ``digital'' data from the inner ear. In mammals, this
region is anatomically and physiologically split into two regions, the
dorsal cochlear nucleus (DCN), and ventral cochlear nucleus (VCN).

\hypertarget{the-trapezoid-body}{%
\subsubsection{The Trapezoid Body}\label{the-trapezoid-body}}

The trapezoid body is a bundle of decussating fibers in the ventral pons
that carry information used for binaural computations in the brainstem.
Some of these axons come from the cochlear nucleus and cross over to the
other side before traveling on to the superior olivary nucleus. This is
believed to help with localization of sound.

\hypertarget{the-superior-olivary-complex}{%
\subsubsection{The superior olivary
complex}\label{the-superior-olivary-complex}}

The superior olivary complex is located in the pons, and receives
projections predominantly from the ventral cochlear nucleus, although
the dorsal cochlear nucleus projects there as well, via the ventral
acoustic stria. Within the superior olivary complex lies the lateral
superior olive (LSO) and the medial superior olive (MSO). The former is
important in detecting interaural level differences while the latter is
important in distinguishing interaural time difference.

\hypertarget{the-lateral-lemniscus}{%
\subsubsection{The Lateral Lemniscus}\label{the-lateral-lemniscus}}

The lateral lemniscus is a tract of axons in the brainstem that carries
information about sound from the cochlear nucleus to various brainstem
nuclei and ultimately the contralateral inferior colliculus of the
midbrain.

\hypertarget{the-inferior-colliculi}{%
\subsubsection{The Inferior Colliculi}\label{the-inferior-colliculi}}

The inferior colliculi (IC) are located just below the visual processing
centers known as the superior colliculi. The central nucleus of the IC
is a nearly obligatory relay in the ascending auditory system, and most
likely acts to integrate information (specifically regarding sound
source localization from the superior olivary complex and dorsal
cochlear nucleus) before sending it to the thalamus and cortex.

\hypertarget{the-medial-geniculate-nucleus-mgn}{%
\subsubsection{The Medial Geniculate Nucleus
(MGN)}\label{the-medial-geniculate-nucleus-mgn}}

The medial geniculate nucleus is part of the thalamic relay system.

\hypertarget{the-primary-auditory-cortex}{%
\subsubsection{The Primary Auditory
Cortex}\label{the-primary-auditory-cortex}}

The primary auditory cortex is the first region of cerebral cortex to
receive auditory input.

Perception of sound is associated with the left posterior superior
temporal gyrus (STG). The superior temporal gyrus contains several
important structures of the brain, including Brodmann areas 41 and 42,
marking the location of the primary auditory cortex, the cortical region
responsible for the sensation of basic characteristics of sound such as
pitch and rhythm. We know from research in nonhuman primates that the
primary auditory cortex can probably be divided further into
functionally differentiable subregions. The neurons of the primary
auditory cortex can be considered to have receptive fields covering a
range of auditory frequencies and have selective responses to harmonic
pitches. Neurons integrating information from the two ears have
receptive fields covering a particular region of auditory space.

The primary auditory cortex is surrounded by secondary auditory cortex,
and interconnects with it. These secondary areas interconnect with
further processing areas in the superior temporal gyrus, in the dorsal
bank of the superior temporal sulcus, and in the frontal lobe. In
humans, connections of these regions with the middle temporal gyrus are
probably important for speech perception. The frontotemporal system
underlying auditory perception allows us to distinguish sounds as
speech, music, or noise.

\hypertarget{the-auditory-ventral-and-dorsal-streams}{%
\subsubsection{The Auditory Ventral And Dorsal
Streams}\label{the-auditory-ventral-and-dorsal-streams}}

From the primary auditory cortex emerge two separate pathways: the
auditory ventral stream and auditory dorsal stream. The auditory ventral
stream includes the anterior superior temporal gyrus, anterior superior
temporal sulcus, middle temporal gyrus and temporal pole. Neurons in
these areas are responsible for sound recognition, and extraction of
meaning from sentences. The auditory dorsal stream includes the
posterior superior temporal gyrus and sulcus, inferior parietal lobule
and intra-parietal sulcus. Both pathways project in humans to the
inferior frontal gyrus. The most established role of the auditory dorsal
stream in primates is sound localization. In humans, the auditory dorsal
stream in the left hemisphere is also responsible for speech repetition
and articulation, phonological long-term encoding of word names, and
verbal working memory.

\hypertarget{the-vestibular-system}{%
\subsection{The Vestibular System}\label{the-vestibular-system}}

The vestibular system, in vertebrates, is part of the inner ear. In most
mammals, the vestibular system is the sensory system that provides the
leading contribution to the sense of balance and spatial orientation for
the purpose of coordinating movement with balance. Together with the
cochlea, a part of the auditory system, it constitutes the labyrinth of
the inner ear in most mammals. As movements consist of rotations and
translations, the vestibular system comprises two components: the
semicircular canals which indicate rotational movements; and the
otoliths which indicate linear accelerations. The vestibular system
sends signals primarily to the neural structures that control eye
movements, and to the muscles that keep an animal upright and in general
control posture. The projections to the former provide the anatomical
basis of the vestibulo-ocular reflex, which is required for vision;
while the projections to the latter provide the anatomical means
required to enable an animal to maintain its position in space.

The brain uses information from the vestibular system in the head and
from proprioception throughout the body to enable the animal to
understand its body's dynamics and kinematics (including its position
and acceleration) from moment to moment. How these two perceptive
sources are integrated to provide the underlying structure of the
sensorium is unknown.

\hypertarget{the-semicircular-canals}{%
\subsubsection{The Semicircular Canals}\label{the-semicircular-canals}}

The semicircular canal system detects rotational movements.

The semicircular canals are a component of the bony labyrinth that are
at right angles to each other. At one end of each of the semicircular
canals is a dilated sac called an osseous ampulla which is more than
twice the diameter of the canal. Each ampulla contains an ampulla crest,
the crista ampullaris which consists of a thick gelatinous cap called a
cupula and many hair cells. The superior and posterior semicircular
canals are oriented vertically at right angles to each other. The
lateral semicircular canal is about a 30-degree angle from the
horizontal plane. The orientations of the canals cause a different canal
to be stimulated by movement of the head in different planes, and more
than one canal is stimulated at once if the movement is off those
planes. The horizontal canal detects angular acceleration of the head
when the head is turned and the superior and posterior canals detect
vertical head movements when the head is moved up or down. When the head
changes position, the endolymph in the canals lags behind due to inertia
and this acts on the cupula which bends the cilia of the hair cells. The
stimulation of the hair cells sends the message to the brain that
acceleration is taking place. The ampullae open into the vestibule by
five orifices, one of the apertures being common to two of the canals.

Since the world is three-dimensional, the vestibular system contains
three semicircular canals in each labyrinth. They are approximately
orthogonal (at right angles) to each other, and are the horizontal (or
lateral), the anterior semicircular canal (or superior), and the
posterior (or inferior) semicircular canal. Anterior and posterior
canals may collectively be called vertical semicircular canals.

The anterior and posterior semicircular canals detect rotations of the
head in the sagittal plane (as when nodding), and in the frontal plane,
as when cartwheeling. Both anterior and posterior canals are orientated
at approximately 45° between frontal and sagittal planes. The movement
of fluid pushes on a structure called the cupula which contains hair
cells that transduce the mechanical movement to electrical signals.

The canals are arranged in such a way that each canal on the left side
has an almost parallel counterpart on the right side. Each of these
three pairs works in a push-pull fashion: when one canal is stimulated,
its corresponding partner on the other side is inhibited, and vice
versa.

\hypertarget{the-otolithic-organs}{%
\subsubsection{The Otolithic Organs}\label{the-otolithic-organs}}

While the semicircular canals respond to rotations, the otolithic organs
sense linear accelerations. Humans have two otolithic organs on each
side, one called the utricle, the other called the saccule. The utricle
contains a patch of hair cells and supporting cells called a macula.
Similarly, the saccule contains a patch of hair cells and a macula. Each
hair cell of a macula has 40-70 stereocilia and one true cilium called a
kinocilium. The tips of these cilia are embedded in an otolithic
membrane. This membrane is weighted down with protein-calcium carbonate
granules called otoconia. These otoconia add to the weight and inertia
of the membrane and enhance the sense of gravity and motion. With the
head erect, the otolithic membrane bears directly down on the hair cells
and stimulation is minimal. When the head is tilted, however, the
otolithic membrane sags and bends the stereocilia, stimulating the hair
cells. Any orientation of the head causes a combination of stimulation
to the utricles and saccules of the two ears. The brain interprets head
orientation by comparing these inputs to each other and to other input
from the eyes and stretch receptors in the neck, thereby detecting
whether the head is tilted or the entire body is tipping. Essentially,
these otolithic organs sense how quickly you are accelerating forward or
backward, left or right, or up or down. Most of the utricular signals
elicit eye movements, while the majority of the saccular signals
projects to muscles that control our posture.

While the interpretation of the rotation signals from the semicircular
canals is straightforward, the interpretation of otolith signals is more
difficult: since gravity is equivalent to a constant linear
acceleration, one somehow has to distinguish otolith signals that are
caused by linear movements from those caused by gravity. Humans can do
that quite well, but the neural mechanisms underlying this separation
are not yet fully understood. Humans can sense head tilting and linear
acceleration even in dark environments because of the orientation of two
groups of hair cell bundles on either side of the striola. Hair cells on
opposite sides move with mirror symmetry, so when one side is moved, the
other is inhibited. The opposing effects caused by a tilt of the head
cause differential sensory inputs from the hair cell bundles allow
humans to tell which way the head is tilting, Sensory information is
then sent to the brain, which can respond with appropriate corrective
actions to the nervous and muscular systems to ensure that balance and
awareness are maintained.

Diseases of the vestibular system can take different forms, and usually
induce vertigo and instability or loss of balance, often accompanied by
nausea.

When the vestibular system and the visual system deliver incongruous
results, nausea often occurs. When the vestibular system reports no
movement but the visual system reports movement, the motion
disorientation is often called motion sickness (or seasickness, car
sickness, simulation sickness, or airsickness). In the opposite case,
such as when a person is in a zero-gravity environment or during a
virtual reality session, the disoriented sensation is often called space
sickness or space adaptation syndrome. Either of these ``sicknesses''
usually ceases once the congruity between the two systems is restored.

Each canal is filled with a fluid called endolymph and contains motion
sensors within the fluids. At the base of each canal, the bony region of
the canal is enlarged which opens into the utricle and has a dilated sac
at one end called the osseous ampullae. Within the ampulla is a mound of
hair cells and supporting cells called crista ampullaris. These hair
cells have many cytoplasmic projections on the apical surface called
stereocilia which are embedded in a gelatinous structure called the
cupula. As the head rotates the duct moves but the endolymph lags behind
owing to inertia. This deflects the cupula and bends the stereocilia
within. The bending of these stereocilia alters an electric signal that
is transmitted to the brain. Within approximately 10 seconds of
achieving constant motion, the endolymph catches up with the movement of
the duct and the cupula is no longer affected, stopping the sensation of
acceleration. The specific gravity of the cupula is comparable to that
of the surrounding endolymph. Consequently, the cupula is not displaced
by gravity, unlike the otolithic membranes of the utricle and saccule.
As with macular hair cells, hair cells of the crista ampullaris will
depolarise when the stereocilia deflect towards the kinocilium.
Deflection in the opposite direction results in hyperpolarisation and
inhibition. In the horizontal canal, ampullopetal flow is necessary for
hair-cell stimulation, whereas ampullofugal flow is necessary for the
anterior and posterior canals.

\hypertarget{vestibular-pathways}{%
\subsubsection{Vestibular Pathways}\label{vestibular-pathways}}

The vestibular nerve is one of the two branches of the vestibulocochlear
nerve (the cochlear nerve being the other). In humans the vestibular
nerve transmits sensory information from vestibular hair cells located
in the two otolith organs (the utricle and the saccule) and the three
semicircular canals via the vestibular ganglion.

Axons of the vestibular nerve synapse in the vestibular nucleus are
found on the lateral floor and wall of the fourth ventricle in the pons
and medulla.

It arises from bipolar cells in the vestibular ganglion, ganglion of
Scarpa, which is situated in the upper part of the outer end of the
internal auditory meatus.

The fibers of the vestibular nerve enter the medulla oblongata on the
medial side of those of the cochlear, and pass between the inferior
peduncle and the spinal tract of the trigeminal nerve.

They then divide into ascending and descending fibers. The latter end by
arborizing around the cells of the medial nucleus, which is situated in
the area acustica of the rhomboid fossa. The ascending fibers either end
in the same manner or in the lateral nucleus, which is situated lateral
to the area acustica and farther from the ventricular floor.

Some of the axons of the cells of the lateral nucleus, and possibly also
of the medial nucleus, are continued upward through the inferior
peduncle to the roof nuclei of the opposite side of the cerebellum, to
which also other fibers of the vestibular root are prolonged without
interruption in the nuclei of the medulla oblongata.

A second set of fibers from the medial and lateral nuclei end partly in
the tegmentum, while the remainder ascend in the medial longitudinal
fasciculus to arborize around the cells of the nuclei of the oculomotor
nerve.

Fibers from the lateral vestibular nucleus also pass via the
vestibulospinal tract, to anterior horn cells at many levels in the
spinal cord, in order to co-ordinate head and trunk movements.

The vestibular cortex is the portion of the cerebrum which responds to
input from the vestibular system. In humans, it has not been completely
delineated but is thought to encompass regions in the parietal and
temporal lobes.

\hypertarget{the-somatic-sensory-system}{%
\subsection{The Somatic Sensory
System}\label{the-somatic-sensory-system}}

The somatic sensory system is a part of the sensory nervous system. The
somatosensory system is a complex system of sensory neurons and neural
pathways that responds to changes at the surface or inside the body. The
axons (as afferent nerve fibers) of sensory neurons connect with, or
respond to, various receptor cells. These sensory receptor cells are
activated by different stimuli such as heat and nociception, giving a
functional name to the responding sensory neuron, such as a
thermoreceptor which carries information about temperature changes.
Other types include mechanoreceptors, chemoreceptors, and nociceptors
which send signals along a sensory nerve to the spinal cord where they
may be processed by other sensory neurons and then relayed to the brain
for further processing. Sensory receptors are found all over the body
including the skin, epithelial tissues, muscles, bones and joints,
internal organs, and the cardiovascular system.

Somatic senses are sometimes referred to as somesthetic senses, with the
understanding that somesthesis includes the sense of touch and
proprioception (sense of position and movement).

The mapping of the body surfaces in the brain is called somatotopy. In
the cortex, it is also referred to as the cortical homunculus. This
brain-surface (``cortical'') map is not immutable, however. Dramatic
shifts can occur in response to stroke or injury.

\hypertarget{touch}{%
\subsubsection{Touch}\label{touch}}

In contrast, the other sense, touch, is a somatic sense which does not
have a specialized organ but comes from all over the body, most
noticeably the skin but also the internal organs (viscera). Touch
includes mechanoreception (pressure, vibration and proprioception), pain
(nociception) and heat (thermoception), and such information is carried
in general somatic afferents and general visceral afferents.

Skin is the soft outer tissue covering of vertebrates with three main
functions: protection, regulation, and sensation.

(ref:skin)
\href{https://commons.wikimedia.org/wiki/File:Gray940.png}{Diagramatic
secion of hairless skin. Note tactile (Meissner) and Pacinian
corpuscles.}

\texttt{\{r\ skindiagram,\ fig.cap=\textquotesingle{}(ref:skin)\textquotesingle{},\ echo=FALSE,\ message=FALSE,\ warning=FALSE\}\ knitr::include\_graphics("./figures/sensation/Gray940.png")}

\hypertarget{cutaneous-mechanoreceptors}{%
\subsubsection{Cutaneous
Mechanoreceptors}\label{cutaneous-mechanoreceptors}}

Cutaneous mechanoreceptors respond to mechanical stimuli that result
from physical interaction, including pressure and vibration. They are
located in the skin. They are all innervated by Aβ fibers, except the
mechanorecepting free nerve endings, which are Aδ fibers. Cutaneous
mechanoreceptors can be categorized by morphology, by what kind of
sensation they perceive, and by the rate of adaptation. Furthermore,
each has a different receptive field.

(ref:mechano) A cartoon representation of the mechanosensitive Piezo1
channel in side and top view.
\href{https://www.rcsb.org/structure/5Z10}{PDB 5Z10}, rendered with open
source molecular visualization tool PyMol.

\texttt{\{r\ mechanosensitive,\ fig.cap=\textquotesingle{}(ref:mechano)\textquotesingle{},\ echo=FALSE,\ message=FALSE,\ warning=FALSE\}\ knitr::include\_graphics("./figures/sensation/mechano\_sensitive\_channel.png")}

In the somatosensory system, receptive fields are regions of the skin or
of internal organs. Some types of mechanoreceptors have large receptive
fields, while others have smaller ones. Large receptive fields allow the
cell to detect changes over a wider area, but lead to a less precise
perception. Thus, the fingers, which require the ability to detect fine
detail, have many, densely packed (up to 500 per cubic cm)
mechanoreceptors with small receptive fields (around 10 square mm),
while the back and legs, for example, have fewer receptors with large
receptive fields.

Tactile-sense-related cortical neurons have receptive fields on the skin
that can be modified by experience or by injury to sensory nerves
resulting in changes in the field's size and position. In general these
neurons have relatively large receptive fields (much larger than those
of dorsal root ganglion cells). However, the neurons are able to
discriminate fine detail due to patterns of excitation and inhibition
relative to the field which leads to spatial resolution.

The term receptive field was first used by Sherrington (1906) to
describe the area of skin from which a scratch reflex could be elicited
in a dog. According to Alonso and Chen (2008) it was Hartline (1938) who
applied the term to single neurons, in this case from the retina of a
frog.

A sensory space can also map into a particular region on an animal's
body. For example, it could be a hair in the cochlea or a piece of skin,
retina, or tongue or other part of an animal's body.

This concept of receptive fields can be extended further up the nervous
system; if many sensory receptors all form synapses with a single cell
further up, they collectively form the receptive field of that cell. For
example, the receptive field of a ganglion cell in the retina of the eye
is composed of input from all of the photoreceptors which synapse with
it, and a group of ganglion cells in turn forms the receptive field for
a cell in the brain. This process is called convergence.

Tactile corpuscles or Meissner's corpuscles are a type of
mechanoreceptor discovered by anatomist
\href{https://en.wikipedia.org/wiki/Georg_Meissner}{Georg Meissner}
(1829--1905) and
\href{https://en.wikipedia.org/wiki/Rudolf_Wagner}{Rudolf Wagner}. They
are a type of nerve ending in the skin that is responsible for
sensitivity to light touch. In particular, they have their highest
sensitivity (lowest threshold) when sensing vibrations between 10 and 50
hertz. They are rapidly adaptive receptors. They are most concentrated
in thick hairless skin, especially at the finger pads.

Tactile corpuscles are encapsulated myelinated nerve endings, which
consist of flattened supportive cells arranged as horizontal lamellae
surrounded by a connective tissue capsule. The corpuscle is 30--140 μm
in length and 40--60 μm in diameter. A single nerve fiber meanders
between the lamellae and throughout the corpuscle.

Pacinian corpuscles (or lamellar corpuscles; discovered by Italian
anatomist \href{https://en.wikipedia.org/wiki/Filippo_Pacini}{Filippo
Pacini}) are one of the four major types of mechanoreceptor cell in
glabrous (hairless) mammalian skin. They are nerve endings in the skin
responsible for sensitivity to vibration and pressure. They respond only
to sudden disturbances and are especially sensitive to vibration. The
vibrational role may be used to detect surface texture, e.g., rough
vs.~smooth. Pacinian corpuscles are also found in the pancreas, where
they detect vibration and possibly very low frequency sounds.{[}dubious
-- discuss{]} Pacinian corpuscles act as very rapidly adapting
mechanoreceptors. Groups of corpuscles respond to pressure changes,
e.g.~on grasping or releasing an object.

Pacinian corpuscles are larger and fewer in number than Meissner's
corpuscle, Merkel cells and Ruffini's corpuscles.

The Pacinian corpuscle is approximately oval-cylindrical-shaped and 1 mm
in length. The entire corpuscle is wrapped by a layer of connective
tissue. Its capsule consists of 20 to 60 concentric lamellae (hence the
alternative lamellar corpuscle) including fibroblasts and fibrous
connective tissue (mainly Type IV and Type II collagen network),
separated by gelatinous material, more than 92\% of which is water.

Pacinian corpuscles are rapidly adapting (phasic) receptors that detect
gross pressure changes and vibrations in the skin. Any deformation in
the corpuscle causes action potentials to be generated by opening
pressure-sensitive sodium ion channels in the axon membrane. This allows
sodium ions to influx, creating a receptor potential.

These corpuscles are especially susceptible to vibrations, which they
can sense even centimeters away. Their optimal sensitivity is 250 Hz,
and this is the frequency range generated upon fingertips by textures
made of features smaller than 1 µm. Pacinian corpuscles cause action
potentials when the skin is rapidly indented but not when the pressure
is steady, due to the layers of connective tissue that cover the nerve
ending. It is thought that they respond to high-velocity changes in
joint position. They have also been implicated in detecting the location
of touch sensations on handheld tools.

Pacinian corpuscles have a large receptive field on the skin's surface
with an especially sensitive center.

Merkel cells, also known as Merkel-Ranvier cells or tactile epithelial
cells, are oval-shaped mechanoreceptors essential for light touch
sensation and found in the skin of vertebrates. They are abundant in
highly sensitive skin like that of the fingertips in humans, and make
synaptic contacts with somatosensory afferent nerve fibers. Although
uncommon, these cells may become malignant and form a Merkel cell
carcinoma---an aggressive and difficult to treat skin cancer.

Though it has been reported that Merkel cells are derived from neural
crest cells, more recent experiments in mammals have indicated that they
are in fact epithelial in origin.

(ref:corp)
\href{https://commons.wikimedia.org/wiki/File:Gray940.png}{Tactile
corpuscles (from left to right): Meissner, Krause, Pacini}

\texttt{\{r\ tactilecorpuscle,\ fig.cap=\textquotesingle{}(ref:corp)\textquotesingle{},\ echo=FALSE,\ message=FALSE,\ warning=FALSE\}\ knitr::include\_graphics("./figures/sensation/TouchReceptors.jpg")}

Merkel cells are found in the skin and some parts of the mucosa of all
vertebrates. In mammalian skin, they are clear cells found in the
stratum basale (at the bottom of sweat duct ridges) of the epidermis
approximately 10 μm in diameter. They also occur in epidermal
invaginations of the plantar foot surface called rete ridges. Most
often, they are associated with sensory nerve endings, when they are
known as Merkel nerve endings (also called a Merkel cell-neurite
complex). They are associated with slowly adapting (SA1) somatosensory
nerve fibers. They react to low vibrations (5--15 Hz) and deep static
touch such as shapes and edges. Due to a small receptive field
(extremely detailed info) they are used in areas like fingertips the
most; they are not covered (shelled) and thus respond to pressures over
long periods.

The German anatomist
\href{https://en.wikipedia.org/wiki/Friedrich_Sigmund_Merkel}{Friedrich
Sigmund Merkel} referred to these cells as Tastzellen or ``touch cells''
but this proposed function has been controversial as it has been hard to
prove. Though, genetic knockout mice have recently shown that Merkel
cells are essential for the specialized coding by which afferent nerves
resolve fine spatial details. Merkel cells are sometimes considered APUD
cells (an older definition. More commonly classified as a part of
dispersed neuroendocrine system) because they contain dense core
granules, and thus may also have a neuroendocrine function.

The bulboid corpuscles (end-bulbs of Krause) are cutaneous receptors in
the human body. The end-bulbs of Krause were named after the German
anatomist \href{https://en.wikipedia.org/wiki/Wilhelm_Krause}{Wilhelm
Krause} (1833--1910). End-bulbs are found in the conjunctiva of the eye
(where they are spheroidal in shape in humans, but cylindrical in most
other animals), in the mucous membrane of the lips and tongue, and in
the epineurium of nerve trunks. They are also found in the penis and the
clitoris and have received the name of genital corpuscles. The end-bulbs
of Krause were thought to be thermoreceptors, sensing cold temperatures,
but their function is unknown.

The Bulbous corpuscle or Ruffini ending or Ruffini corpuscle is a slowly
adapting mechanoreceptor located in the cutaneous tissue between the
dermal papillae and the hypodermis. It is named after Angelo Ruffini.

Ruffini corpuscles are enlarged dendritic endings with elongated
capsules.

This spindle-shaped receptor is sensitive to skin stretch, and
contributes to the kinesthetic sense of and control of finger position
and movement. They are at the highest density around the fingernails
where they act in monitoring slippage of objects along the surface of
the skin, allowing modulation of grip on an object.

Ruffini corpuscles respond to sustained pressure and show very little
adaptation.

Ruffinian endings are located in the deep layers of the skin, and
register mechanical deformation within joints, more specifically angle
change, with a specificity of up to 2.75 degrees, as well as continuous
pressure states. They also act as thermoreceptors that respond for a
long time, so in case of deep burn there will be pain, as these
receptors will be burned off.

\hypertarget{nociception}{%
\subsubsection{Nociception}\label{nociception}}

\href{https://en.wikipedia.org/wiki/Nociception}{Nociception} (also
nocioception or nociperception, from Latin nocere `to harm or hurt') is
the sensory nervous system's response to certain harmful or potentially
harmful stimuli. The term ``nociception'' was coined by
\href{https://en.wikipedia.org/wiki/Charles_Scott_Sherrington}{Charles
Scott Sherrington} to distinguish the physiological process (nervous
activity) from pain (a subjective experience). In nociception, intense
chemical (e.g., cayenne powder), mechanical (e.g., cutting, crushing),
or thermal (heat and cold) stimulation of sensory nerve cells called
nociceptors produces a signal that travels along a chain of nerve fibers
via the spinal cord to the brain. Nociception triggers a variety of
physiological and behavioral responses and usually results in a
subjective experience of pain in sentient beings. Nociception can also
cause generalized autonomic responses before or without reaching
consciousness to cause pallor, sweating, tachycardia, hypertension,
lightheadedness, nausea and fainting.

Potentially damaging mechanical, thermal, and chemical stimuli are
detected by nerve endings called nociceptors, which are found in the
skin, on internal surfaces such as the periosteum, joint surfaces, and
in some internal organs. Some nociceptors are unspecialized free nerve
endings that have their cell bodies outside the spinal column in the
dorsal root ganglia. Nociceptors are categorized according to the axons
which travel from the receptors to the spinal cord or brain.

Nociceptors have a certain threshold; that is, they require a minimum
intensity of stimulation before they trigger a signal. Once this
threshold is reached a signal is passed along the axon of the neuron
into the spinal cord.

Nociceptive threshold testing deliberately applies a noxious stimulus to
a human or animal subject in order to study pain. In animals, the
technique is often used to study the efficacy of analgesic drugs and to
establish dosing levels and period of effect. After establishing a
baseline, the drug under test is given and the elevation in threshold
recorded at specified time points. When the drug wears off, the
threshold should return to the baseline (pre-treatment) value.

In some conditions, excitation of pain fibers becomes greater as the
pain stimulus continues, leading to a condition called hyperalgesia.

Thermoception refers to stimuli of moderate temperatures 24--28 °C
(75--82 °F), as anything beyond that range is considered pain and
moderated by nociceptors. TRP and potassium channels {[}TRPM (1-8), TRPV
(1-6), TRAAK, and TREK{]} each respond to different temperatures (among
other stimuli) which create action potentials in nerves which join the
mechano (touch) system in the posterolateral tract. Thermoception, like
proprioception, is then covered by the somatosensory system.

TRP channels that detect noxious stimuli (mechanical, thermal, and
chemical pain) relay that info to nociceptors that generate an action
potential. Mechanical TRP channels react to depression of their cells
(like touch), thermal TRP change shape in different temperatures, and
chemical TRP act like taste buds, signalling if their receptors bond to
certain elements/chemicals.

(ref:TRPV1) A cartoon representation of the transient receptor potential
cation channel subfamily V member 1 (TRPV1), also known as the capsaicin
receptor and the vanilloid receptor 1 viewd from the side and the top.
The function of TRPV1 is detection and regulation of body temperature.
In addition, TRPV1 provides a sensation of scalding heat and pain
(nociception). In primary afferent sensory neurons, it cooperates with
TRPA1 (a chemical irritant receptor) to mediate the detection of noxious
environmental stimuli. \href{https://www.rcsb.org/structure/5IS0}{PDB
5IS0}, rendered with open source molecular visualization tool PyMol.

\texttt{\{r\ TRPV1channel,\ fig.cap=\textquotesingle{}(ref:TRPV1)\textquotesingle{},\ echo=FALSE,\ message=FALSE,\ warning=FALSE\}\ knitr::include\_graphics("./figures/sensation/TRPV1.png")}

\hypertarget{proprioception}{%
\subsubsection{Proprioception}\label{proprioception}}

Proprioception is the sense of self-movement and body position. It is
sometimes described as the ``sixth sense''.

Proprioception is mediated by mechanically sensitive proprioceptor
neurons distributed throughout an animal's body. Most vertebrates
possess three basic types of proprioceptors: muscle spindles, which are
embedded in skeletal muscle fibers, Golgi tendon organs, which lie at
the interface of muscles and tendons, and joint receptors, which are
low-threshold mechanoreceptors embedded in joint capsules. Many
invertebrates, such as insects, also possess three basic proprioceptor
types with analogous functional properties: chordotonal neurons,
campaniform sensilla, and hair plates.

The initiation of proprioception is the activation of a proprioreceptor
in the periphery. The proprioceptive sense is believed to be composed of
information from sensory neurons located in the inner ear (motion and
orientation) and in the stretch receptors located in the muscles and the
joint-supporting ligaments (stance). There are specific nerve receptors
for this form of perception termed ``proprioreceptors''.

An important role for proprioception is to allow an animal to stabilize
itself against perturbations. For instance, for a person to walk or
stand upright, they must continuously monitor their posture and adjust
muscle activity as needed to provide balance. Similarly, when walking on
unfamiliar terrain or even tripping, the person must adjust the output
of their muscles quickly based on estimated limb position and velocity.
Proprioceptor reflex circuits are thought to play an important role to
allow fast and unconscious execution of these behaviors, To make control
of these behaviors efficient, proprioceptors are also thought to
regulate reciprocal inhibition in muscles, leading to agonist-antagonist
muscle pairs.

When planning complex movements such as reaching or grooming, animals
must consider the current position and velocity of their limb and use it
to adjust dynamics to target a final position. If the animal's estimate
of their limb's initial position is wrong, this can lead to a deficiency
in the movement. Furthermore, proprioception is crucial in refining the
movement if it deviates from the trajectory.

\hypertarget{muscle-spindles}{%
\subsubsection{Muscle Spindles}\label{muscle-spindles}}

Muscle spindles are stretch receptors within the body of a muscle that
primarily detect changes in the length of the muscle. They convey length
information to the central nervous system via afferent nerve fibers.
This information can be processed by the brain as proprioception. The
responses of muscle spindles to changes in length also play an important
role in regulating the contraction of muscles, for example, by
activating motor neurons via the stretch reflex to resist muscle
stretch.

(ref:spindle) Muscle spindle from the pectoral muscle of a frog. Bottom
part of figure: B) extrafusal muscle fibers with efferent motor nerve;
top part of figure: myelinated afferent nerve fiber.
\href{https://wellcomelibrary.org/item/b2129592x\#?c=0\&m=0\&s=0\&cv=14\&z=0\%2C-3.48\%2C1\%2C8.6591}{Histologie
du système nerveux de l'homme \& des vertébrés, Tome Premier} (1909) by
Santiago Ramón y Cajal translated from Spanish by Dr.~L. Azoulay.

\texttt{\{r\ musclespindle,\ fig.cap=\textquotesingle{}(ref:spindle)\textquotesingle{},\ echo=FALSE,\ message=FALSE,\ warning=FALSE\}\ knitr::include\_graphics("./figures/sensation/CajalMuscleSpindle.jpg")}

Muscle spindles are found within the belly of muscles, between
extrafusal muscle fibers.{[}b{]} The specialised fibers that constitute
the muscle spindle are known as intrafusal fibers (as they are present
within the spindle), to distinguish themselves from the fibres of the
muscle itself which are called extrafusal fibers. Muscle spindles have a
capsule of connective tissue, and run parallel to the extrafusal muscle
fibers.

\hypertarget{golgi-tendon-organ}{%
\subsubsection{Golgi Tendon Organ}\label{golgi-tendon-organ}}

The Golgi tendon organ (GTO) (also called Golgi organ, tendon organ,
neurotendinous organ or neurotendinous spindle) is a proprioceptive
sensory receptor organ that senses changes in muscle tension. It lies at
the origins and insertion of skeletal muscle fibers into the tendons of
skeletal muscle. It provides the sensory component of the Golgi tendon
reflex.

(ref:tendon) Golgi tendon organ from an adult cat. a) Termninal
arboriasation and c) fine ultimate brances of the afferent nerve; b)
myelinated afferent nerve fiber; e) muscle fibers.
\href{https://wellcomelibrary.org/item/b2129592x\#?c=0\&m=0\&s=0\&cv=14\&z=0\%2C-3.48\%2C1\%2C8.6591}{Histologie
du système nerveux de l'homme \& des vertébrés, Tome Premier} (1909) by
Santiago Ramón y Cajal translated from Spanish by Dr.~L. Azoulay.

\texttt{\{r\ tendonorgan,\ fig.cap=\textquotesingle{}(ref:tendon)\textquotesingle{},\ echo=FALSE,\ message=FALSE,\ warning=FALSE\}\ knitr::include\_graphics("./figures/sensation/CajalGolgiTendon.jpg")}

The body of the organ is made up of braided strands of collagen
(intrafusal fasciculi) that are less compact than elsewhere in the
tendon and are encapsulated. The capsule is connected in series with a
group of muscle fibers (10-20 fibers) at one end, and merge into the
tendon proper at the other. Each capsule is about 1 mm long, has a
diameter of about 0.1 mm, and is perforated by one or more afferent type
Ib sensory nerve fibers (Aɑ fiber), which are large (12-20 μm)
myelinated axons that can conduct nerve impulses very rapidly. Inside
the capsule, the afferent fibers lose their medullary sheaths, branch,
intertwine with the collagen fibers, and terminate as flattened
leaf-like endings between the collagen strands (see figure).

\hypertarget{the-somatosensory-pathways}{%
\subsubsection{The Somatosensory
Pathways}\label{the-somatosensory-pathways}}

All afferent touch/vibration info ascends the spinal cord via the
posterior (dorsal) column-medial lemniscus pathway via gracilis (T7 and
below) or cuneatus (T6 and above). Cuneatus sends signals to the
cochlear nucleus indirectly via spinal grey matter, this info is used in
determining if a perceived sound is just villi noise/irritation. All
fibers cross (left becomes right) in the medulla.

A somatosensory pathway will typically have three neurons: first-order,
second-order, and third-order.

\begin{enumerate}
\def\labelenumi{\arabic{enumi}.}
\tightlist
\item
  The first-order neuron is a type of pseudounipolar neuron and always
  has its cell body in the dorsal root ganglion of the spinal nerve with
  a peripheral axon innervating touch mechanoreceptors and a central
  axon synapsing on the second-order neuron. If the somatosensory
  pathway is in parts of the head or neck not covered by the cervical
  nerves, the first-order neuron will be the trigeminal nerve ganglia or
  the ganglia of other sensory cranial nerves).
\item
  The second-order neuron has its cell body either in the spinal cord or
  in the brainstem. This neuron's ascending axons will cross (decussate)
  to the opposite side either in the spinal cord or in the brainstem.
\item
  In the case of touch and certain types of pain, the third-order neuron
  has its cell body in the ventral posterior nucleus of the thalamus and
  ends in the postcentral gyrus of the parietal lobe in the primary
  somatosensory cortex (or S1).
\end{enumerate}

Fine touch (or discriminative touch) is a sensory modality that allows a
subject to sense and localize touch. The form of touch where
localization is not possible is known as crude touch. The posterior
column--medial lemniscus pathway is the pathway responsible for the
sending of fine touch information to the cerebral cortex of the brain.

Crude touch (or non-discriminative touch) is a sensory modality that
allows the subject to sense that something has touched them, without
being able to localize where they were touched (contrasting ``fine
touch''). Its fibres are carried in the spinothalamic tract, unlike the
fine touch, which is carried in the dorsal column. As fine touch
normally works in parallel to crude touch, a person will be able to
localize touch until fibres carrying fine touch (Posterior
column--medial lemniscus pathway) have been disrupted. Then the subject
will feel the touch, but be unable to identify where they were touched.

In humans, temperature sensation from thermoreceptors enters the spinal
cord along the axons of Lissauer's tract that synapse on second order
neurons in grey matter of the dorsal horn. The axons of these second
order neurons then decussate, joining the spinothalamic tract as they
ascend to neurons in the ventral posterolateral nucleus of the thalamus.

\hypertarget{the-primary-somatosensory-cortex}{%
\subsubsection{The Primary Somatosensory
Cortex}\label{the-primary-somatosensory-cortex}}

The primary somatosensory cortex is located in the postcentral gyrus,
and is part of the somatosensory system.

At the primary somatosensory cortex, tactile representation is orderly
arranged (in an inverted fashion) from the toe (at the top of the
cerebral hemisphere) to mouth (at the bottom). However, some body parts
may be mapped to partially overlapping regions of cortex. Each cerebral
hemisphere of the primary somatosensory cortex only contains a tactile
representation of the opposite (contralateral) side of the body. The
amount of primary somatosensory cortex devoted to a body part is not
proportional to the absolute size of the body surface, but, instead, to
the relative density of cutaneous tactile receptors on that body part.
The density of cutaneous tactile receptors on a body part is generally
indicative of the degree of sensitivity of tactile stimulation
experienced at said body part. For example, the human lips and hands
have a larger representation than other body parts.

Brodmann areas 3, 1, and 2 make up the primary somatosensory cortex of
the human brain (or S1). Brodmann area (BA) 3 is subdivided into areas
3a and 3b. Because Brodmann sliced the brain somewhat obliquely, he
encountered area 1 first; however, from anterior to posterior, the
Brodmann designations are 3, 1, and 2, respectively.

Areas 1 and 2 receive dense inputs from BA 3b. The projection from 3b to
1 primarily relays texture information; the projection to area 2
emphasizes size and shape. Lesions confined to these areas produce
predictable dysfunction in texture, size, and shape discrimination.

Somatosensory cortex, like other neocortex, is layered. Like other
sensory cortex (i.e., visual and auditory) the thalamic inputs project
into layer IV, which in turn project into other layers. As in other
sensory cortices, S1 neurons are grouped together with similar inputs
and responses into vertical columns that extend across cortical layers.

\hypertarget{the-olfactory-system}{%
\subsection{The Olfactory System}\label{the-olfactory-system}}

The olfactory system, or sense of smell, is the sensory system used for
smelling (olfaction). Olfaction is one of the special senses, that have
directly associated specific organs. Most mammals and reptiles have a
main olfactory system and an accessory olfactory system. The main
olfactory system detects airborne substances, while the accessory system
senses fluid-phase stimuli. Often, land organisms will have separate
olfaction systems for smell and taste (orthonasal smell and retronasal
smell), but water-dwelling organisms usually have only one system. The
senses of smell and taste (gustatory system) are often referred to
together as the chemosensory system, because they both give the brain
information about the chemical composition of objects.

Olfaction is a chemoreception that forms the sense of smell. Olfaction
has many purposes, such as the detection of hazards, pheromones, and
food. It integrates with other senses to form the sense of flavor.
Olfaction occurs when odorants bind to specific sites on olfactory
receptors located in the nasal cavity. Glomeruli aggregate signals from
these receptors and transmit them to the olfactory bulb, where the
sensory input will start to interact with parts of the brain responsible
for smell identification, memory, and emotion.

In insects, smells are sensed by olfactory sensory neurons in the
chemosensory sensilla, which are present in insect antenna, palps, and
tarsa, but also on other parts of the insect body. Odorants penetrate
into the cuticle pores of chemosensory sensilla and get in contact with
insect odorant-binding proteins (OBPs) or Chemosensory proteins (CSPs),
before activating the sensory neurons.

In vertebrates, smells are sensed by olfactory sensory neurons in the
olfactory epithelium. The olfactory epithelium is made up of at least
six morphologically and biochemically different cell types. The
proportion of olfactory epithelium compared to respiratory epithelium
(not innervated, or supplied with nerves) gives an indication of the
animal's olfactory sensitivity. Humans have about 10
cm\textsuperscript{2} of olfactory epithelium, whereas some dogs have
170 cm\textsuperscript{2}. A dog's olfactory epithelium is also
considerably more densely innervated, with a hundred times more
receptors per square centimeter.

Molecules of odorants passing through the superior nasal concha of the
nasal passages dissolve in the mucus that lines the superior portion of
the cavity and are detected by olfactory receptors on the dendrites of
the olfactory sensory neurons. This may occur by diffusion or by the
binding of the odorant to odorant-binding proteins. The mucus overlying
the epithelium contains mucopolysaccharides, salts, enzymes, and
antibodies (these are highly important, as the olfactory neurons provide
a direct passage for infection to pass to the brain). This mucus acts as
a solvent for odor molecules, flows constantly, and is replaced
approximately every ten minutes.

\hypertarget{the-nose}{%
\subsubsection{The Nose}\label{the-nose}}

The human nose is the most protruding part of the face. It bears the
nostrils and is the first organ of the respiratory system. It is also
the principal organ in the olfactory system. The shape of the nose is
determined by the nasal bones and the nasal cartilages, including the
nasal septum which separates the nostrils and divides the nasal cavity
into two.

The main function of the nose is respiration, and the nasal mucosa
lining the nasal cavity and the paranasal sinuses carries out the
necessary conditioning of inhaled air by warming and moistening it.
Nasal conchae, shell-like bones in the walls of the cavities, play a
major part in this process. Filtering of the air by nasal hair in the
nostrils prevents large particles from entering the lungs. Sneezing is a
reflex to expel unwanted particles from the nose that irritate the
mucosal lining. Sneezing can transmit infections, because aerosols are
created in which the droplets can harbour pathogens.

Another major function of the nose is olfaction, the sense of smell. The
area of olfactory epithelium, in the upper nasal cavity, contains
specialised olfactory cells responsible for this function.

The peripheral olfactory system consists mainly of the nostrils, ethmoid
bone, nasal cavity, and the olfactory epithelium (layers of thin tissue
covered in mucus that line the nasal cavity). The primary components of
the layers of epithelial tissue are the mucous membranes, olfactory
glands, olfactory neurons, and nerve fibers of the olfactory nerves.

(ref:septum) View of the right side of the nasal septum showing the
olfactory bulb and the filaments of the olfactory nerve.From
\href{https://wellcomelibrary.org/item/b21688692}{Gray, Henry: Anatomy
Descriptive And Surgical. 7\textsuperscript{th} Edition, Longmans,
Green, And Co., London, 1875}.

\texttt{\{r\ rightseptum,\ fig.cap=\textquotesingle{}(ref:septum)\textquotesingle{},\ echo=FALSE,\ message=FALSE,\ warning=FALSE\}\ knitr::include\_graphics("./figures/sensation/GrayP572.jpg")}

Odor molecules can enter the peripheral pathway and reach the nasal
cavity either through the nostrils when inhaling (olfaction) or through
the throat when the tongue pushes air to the back of the nasal cavity
while chewing or swallowing (retro-nasal olfaction). Inside the nasal
cavity, mucus lining the walls of the cavity dissolves odor molecules.
Mucus also covers the olfactory epithelium, which contains mucous
membranes that produce and store mucus and olfactory glands that secrete
metabolic enzymes found in the mucus.

\hypertarget{olfactory-sensory-neurons}{%
\subsubsection{Olfactory Sensory
Neurons}\label{olfactory-sensory-neurons}}

Humans have between 10 and 20 million olfactory receptor neurons (ORNs).
In vertebrates, ORNs are bipolar neurons with dendrites facing the
external surface of the cribriform plate with axons that pass through
the cribriform foramina with terminal end at olfactory bulbs. The ORNs
are located in the olfactory epithelium in the nasal cavity. The cell
bodies of the ORNs are distributed among all three of the stratified
layers of the olfactory epithelium.

Many tiny hair-like cilia protrude from the olfactory receptor cell's
dendrite into the mucus covering the surface of the olfactory
epithelium. The surface of these cilia is covered with olfactory
receptors, a type of G protein-coupled receptor. Each olfactory receptor
cell expresses only one type of olfactory receptor (OR), but many
separate olfactory receptor cells express ORs which bind the same set of
odors. The axons of olfactory receptor cells which express the same OR
converge to form glomeruli in the olfactory bulb.Olfactory sensory
neurons in the epithelium detect odor molecules dissolved in the mucus
and transmit information about the odor to the brain in a process called
sensory transduction. Olfactory neurons have cilia (tiny hairs)
containing olfactory receptors that bind to odor molecules, causing an
electrical response that spreads through the sensory neuron to the
olfactory nerve fibers at the back of the nasal cavity.

Olfactory receptors (ORs), also known as odorant receptors, are
expressed in the cell membranes of olfactory receptor neurons and are
responsible for the detection of odorants (i.e., compounds that have an
odor) which give rise to the sense of smell. Activated olfactory
receptors trigger nerve impulses which transmit information about odor
to the brain. These receptors are members of the class A rhodopsin-like
family of G protein-coupled receptors (GPCRs). The olfactory receptors
form a multigene family consisting of around 800 genes in humans and
1400 genes in mice

Rather than binding specific ligands, olfactory receptors display
affinity for a range of odor molecules, and conversely a single odorant
molecule may bind to a number of olfactory receptors with varying
affinities, which depend on physio-chemical properties of molecules like
their molecular volumes. Once the odorant has bound to the odor
receptor, the receptor undergoes structural changes and it binds and
activates the olfactory-type G protein on the inside of the olfactory
receptor neuron. The G protein (G\textsubscript{olf} and/or
G\textsubscript{s}) in turn activates adenylate cyclase - which converts
ATP into cyclic AMP (cAMP). The cAMP opens cyclic nucleotide-gated ion
channels which allow calcium and sodium ions to enter into the cell,
depolarizing the olfactory receptor neuron and triggering the firing of
action potentials which convey the information to the brain.

Olfactory nerves and fibers transmit information about odors from the
peripheral olfactory system to the central olfactory system of the
brain, which is separated from the epithelium by the cribriform plate of
the ethmoid bone. Olfactory nerve fibers, which originate in the
epithelium, pass through the cribriform plate, connecting the epithelium
to the brain's limbic system at the olfactory bulbs.

(ref:olf) Diagram of the structure of the olfactory bulb and olfactory
cortex. A) Olfacory mucosa; B) glomeruli in the olfactory bulb; C)
mitral cells; D) granule cells; E) olfactory nerve; F) pyramidal cells
in the olfactory cortex. Action potentials fired by the olfactory
receptor cells and subsequently by the cells in the olfactory bulb,
travel to the olfactory cortex (arrows).
\href{https://wellcomelibrary.org/item/b2129592x\#?c=0\&m=0\&s=0\&cv=14\&z=0\%2C-3.48\%2C1\%2C8.6591}{Histologie
du système nerveux de l'homme \& des vertébrés, Tome Premier} (1909) by
Santiago Ramón y Cajal translated from Spanish by Dr.~L. Azoulay.

\texttt{\{r\ olfactory,\ fig.cap=\textquotesingle{}(ref:olf)\textquotesingle{},\ echo=FALSE,\ message=FALSE,\ warning=FALSE\}\ knitr::include\_graphics("./figures/sensation/CajalOlfactoryPathway.jpg")}

\hypertarget{the-olfactory-bulb}{%
\subsubsection{The Olfactory Bulb}\label{the-olfactory-bulb}}

The main olfactory bulb transmits pulses to both mitral and tufted
cells, which help determine odor concentration based off the time
certain neuron clusters fire (called `timing code'). These cells also
note differences between highly similar odors and use that data to aid
in later recognition. The cells are different with mitral having low
firing-rates and being easily inhibited by neighboring cells, while
tufted have high rates of firing and are more difficult to inhibit.

\hypertarget{the-olfactory-cortex}{%
\subsubsection{The Olfactory Cortex}\label{the-olfactory-cortex}}

The uncus(an anterior extremity of the parahippocampal gyrus, a region
that surrounds the hippocampus and is part of the limbic system) houses
the olfactory cortex which includes the piriform cortex (posterior
orbitofrontal cortex), amygdala, olfactory tubercle, and parahippocampal
gyrus.

The olfactory tubercle connects to numerous areas of the amygdala,
thalamus, hypothalamus, hippocampus, brain stem, retina, auditory
cortex, and olfactory system.

The anterior olfactory nucleus distributes reciprocal signals between
the olfactory bulb and piriform cortex. The anterior olfactory nucleus
is the memory hub for smell.

\hypertarget{olfactory-pathways}{%
\subsubsection{Olfactory Pathways}\label{olfactory-pathways}}

Olfactory sensory neurons project axons to the brain within the
olfactory nerve, (cranial nerve I). These nerve fibers, lacking myelin
sheaths, pass to the olfactory bulb of the brain through perforations in
the cribriform plate, which in turn projects olfactory information to
the olfactory cortex and other areas. The axons from the olfactory
receptors converge in the outer layer of the olfactory bulb within small
(≈50 micrometers in diameter) structures called glomeruli. Mitral cells,
located in the inner layer of the olfactory bulb, form synapses with the
axons of the sensory neurons within glomeruli and send the information
about the odor to other parts of the olfactory system, where multiple
signals may be processed to form a synthesized olfactory perception. A
large degree of convergence occurs, with 25,000 axons synapsing on 25 or
so mitral cells, and with each of these mitral cells projecting to
multiple glomeruli. Mitral cells also project to periglomerular cells
and granular cells that inhibit the mitral cells surrounding it (lateral
inhibition). Granular cells also mediate inhibition and excitation of
mitral cells through pathways from centrifugal fibers and the anterior
olfactory nuclei. Neuromodulators like acetylcholine, serotonin and
norepinephrine all send axons to the olfactory bulb and have been
implicated in gain modulation, pattern separation, and memory functions,
respectively.

The mitral cells leave the olfactory bulb in the lateral olfactory
tract, which synapses on five major regions of the cerebrum: the
anterior olfactory nucleus, the olfactory tubercle, the amygdala, the
piriform cortex, and the entorhinal cortex. The anterior olfactory
nucleus projects, via the anterior commissure, to the contralateral
olfactory bulb, inhibiting it. The piriform cortex has two major
divisions with anatomically distinct organizations and functions. The
anterior piriform cortex (APC) appears to be better at determining the
chemical structure of the odorant molecules, and the posterior piriform
cortex (PPC) has a strong role in categorizing odors and assessing
similarities between odors (e.g.~minty, woody, and citrus are odors that
can, despite being highly variant chemicals, be distinguished via the
PPC in a concentration-independent manner). The piriform cortex projects
to the medial dorsal nucleus of the thalamus, which then projects to the
orbitofrontal cortex. The orbitofrontal cortex mediates conscious
perception of the odor (citation needed). The three-layered piriform
cortex projects to a number of thalamic and hypothalamic nuclei, the
hippocampus and amygdala and the orbitofrontal cortex, but its function
is largely unknown. The entorhinal cortex projects to the amygdala and
is involved in emotional and autonomic responses to odor. It also
projects to the hippocampus and is involved in motivation and memory.
Odor information is stored in long-term memory and has strong
connections to emotional memory. This is possibly due to the olfactory
system's close anatomical ties to the limbic system and hippocampus,
areas of the brain that have long been known to be involved in emotion
and place memory, respectively.

Since any one receptor is responsive to various odorants, and there is a
great deal of convergence at the level of the olfactory bulb, it may
seem strange that human beings are able to distinguish so many different
odors. It seems that a highly complex form of processing must be
occurring; however, as it can be shown that, while many neurons in the
olfactory bulb (and even the pyriform cortex and amygdala) are
responsive to many different odors, half the neurons in the
orbitofrontal cortex are responsive to only one odor, and the rest to
only a few. It has been shown through microelectrode studies that each
individual odor gives a particular spatial map of excitation in the
olfactory bulb. It is possible that the brain is able to distinguish
specific odors through spatial encoding, but temporal coding must also
be taken into account. Over time, the spatial maps change, even for one
particular odor, and the brain must be able to process these details as
well.

Inputs from the two nostrils have separate inputs to the brain, with the
result that, when each nostril takes up a different odorant, a person
may experience perceptual rivalry in the olfactory sense akin to that of
binocular rivalry.

In insects, smells are sensed by sensilla located on the antenna and
maxillary palp and first processed by the antennal lobe (analogous to
the olfactory bulb), and next by the mushroom bodies and lateral horn.

Many animals, including most mammals and reptiles, but not humans, have
two distinct and segregated olfactory systems: a main olfactory system,
which detects volatile stimuli, and an accessory olfactory system, which
detects fluid-phase stimuli. Behavioral evidence suggests that these
fluid-phase stimuli often function as pheromones, although pheromones
can also be detected by the main olfactory system. In the accessory
olfactory system, stimuli are detected by the vomeronasal organ, located
in the vomer, between the nose and the mouth. Snakes use it to smell
prey, sticking their tongue out and touching it to the organ.

The sensory receptors of the accessory olfactory system are located in
the vomeronasal organ. As in the main olfactory system, the axons of
these sensory neurons project from the vomeronasal organ to the
accessory olfactory bulb, which in the mouse is located on the
dorsal-posterior portion of the main olfactory bulb. Unlike in the main
olfactory system, the axons that leave the accessory olfactory bulb do
not project to the brain's cortex but rather to targets in the amygdala
and bed nucleus of the stria terminalis, and from there to the
hypothalamus, where they may influence aggression and mating behavior.

The process by which olfactory information is coded in the brain to
allow for proper perception is still being researched, and is not
completely understood. When an odorant is detected by receptors, they in
a sense break the odorant down, and then the brain puts the odorant back
together for identification and perception. The odorant binds to
receptors that recognize only a specific functional group, or feature,
of the odorant, which is why the chemical nature of the odorant is
important.

After binding the odorant, the receptor is activated and will send a
signal to the glomeruli. Each glomerulus receives signals from multiple
receptors that detect similar odorant features. Because several receptor
types are activated due to the different chemical features of the
odorant, several glomeruli are activated as well. All of the signals
from the glomeruli are then sent to the brain, where the combination of
glomeruli activation encodes the different chemical features of the
odorant. The brain then essentially puts the pieces of the activation
pattern back together in order to identify and perceive the odorant.
This distributed code allows the brain to detect specific odors in
mixtures of many background odors.

Different people smell different odors, and most of these differences
are caused by genetic differences. Although odorant receptor genes make
up one of the largest gene families in the human genome, only a handful
of genes have been linked conclusively to particular smells. For
instance, the odorant receptor OR5A1 and its genetic variants (alleles)
are responsible for our ability (or failure) to smell β-ionone, a key
aroma in foods and beverages. Similarly, the odorant receptor OR2J3 is
associated with the ability to detect the ``grassy'' odor,
cis-3-hexen-1-ol. The preference (or dislike) of cilantro (coriander)
has been linked to the olfactory receptor OR6A2.

\hypertarget{the-gustatory-system}{%
\subsection{The Gustatory System}\label{the-gustatory-system}}

Taste, gustatory perception, or gustation is one of the five traditional
senses that belongs to the gustatory system.

Chemicals that stimulate taste receptor cells are known as tastants. The
tongue is equipped with many taste buds on its dorsal surface, and each
taste bud is equipped with taste receptor cells that can sense
particular classes of tastes. Distinct types of taste receptor cells
respectively detect substances that are sweet, bitter, salty, sour,
spicy, or taste of umami.

Taste, along with smell (olfaction) and trigeminal nerve stimulation
(registering texture, pain, and temperature), determines flavors of food
and/or other substances. Humans have taste receptors on taste buds
(gustatory calyculi) and other areas including the upper surface of the
tongue and the epiglottis. The gustatory cortex is responsible for the
perception of taste.

\hypertarget{the-tongue}{%
\subsubsection{The Tongue}\label{the-tongue}}

The tongue is a muscular organ in the mouth of most vertebrates that
manipulates food for mastication and is used in the act of swallowing.
It has importance in the digestive system and is the primary organ of
taste in the gustatory system. The tongue's upper surface (dorsum) is
covered by taste buds housed in numerous lingual papillae. It is
sensitive and kept moist by saliva and is richly supplied with nerves
and blood vessels. The tongue also serves as a natural means of cleaning
the teeth. A major function of the tongue is the enabling of speech in
humans and vocalization in other animals.

(ref:ton) The human tongue and three kinds of papillae magnified. Bottom
left to right: filiform, fungiform and circumvallate papillae. From
\href{https://wellcomelibrary.org/item/b21688692}{Gray, Henry: Anatomy
Descriptive And Surgical. 7\textsuperscript{th} Edition, Longmans,
Green, And Co., London, 1875}

\texttt{\{r\ tongue,\ fig.cap=\textquotesingle{}(ref:ton)\textquotesingle{},\ echo=FALSE,\ message=FALSE,\ warning=FALSE\}\ knitr::include\_graphics("./figures/sensation/tongue.jpg")}

Innervation of the tongue consists of motor fibers, special sensory
fibers for taste, and general sensory fibers for sensation.

The tongue is covered with thousands of small bumps called papillae,
which are visible to the naked eye. Within each papilla are hundreds of
taste buds. The exception to this is the filiform papillae that do not
contain taste buds. Each taste bud contains 50 to 100 taste receptor
cells.

(ref:papilla) Section through a papilla vallata of the human tongue. 1)
Papilla. 2) Vallum. 3) Taste buds. From
\href{https://wellcomelibrary.org/item/b21271070}{Textbook of anatomy.
Section 2. The muscular system: the nervous system: the organs of sense
and integument edited by D. J. Cunningham}

\texttt{\{r\ vallatepapilla,\ fig.cap=\textquotesingle{}(ref:papilla)\textquotesingle{},\ echo=FALSE,\ message=FALSE,\ warning=FALSE\}\ knitr::include\_graphics("./figures/sensation/CunninghamSection2Page770.jpg")}

\hypertarget{the-five-basic-tastes}{%
\subsubsection{The Five Basic Tastes}\label{the-five-basic-tastes}}

The sensation of taste includes five established basic tastes:
sweetness, sourness, saltiness, bitterness, and savoriness (also known
as savory or umami). Taste buds are able to distinguish between
different tastes through detecting interaction with different molecules
or ions. Sweet, savoriness, and bitter tastes are triggered by the
binding of molecules to G protein-coupled receptors on the cell
membranes of taste buds. Saltiness and sourness are perceived when
alkali metal or hydrogen ions enter taste buds, respectively.

(ref:buds) A. Three quarter surface view of a taste bud from the papilla
foliata of a rabbit (highly magnified). B. Vertical section of taste bud
from the papilla foliata of a rabbit (highly magnified). From
\href{https://wellcomelibrary.org/item/b21271070}{Textbook of anatomy.
Section 2. The muscular system: the nervous system: the organs of sense
and integument edited by D. J. Cunningham}

\texttt{\{r\ tastebuds,\ fig.cap=\textquotesingle{}(ref:buds)\textquotesingle{},\ echo=FALSE,\ message=FALSE,\ warning=FALSE\}\ knitr::include\_graphics("./figures/sensation/CunninghamSection2Page771.jpg")}

(ref:cells) Isolated cells from taste bud of a rabbit. a, Supporting
cells. b, Gustatory cells. From
\href{https://wellcomelibrary.org/item/b21271070}{Textbook of anatomy.
Section 2. The muscular system: the nervous system: the organs of sense
and integument edited by D. J. Cunningham}

\texttt{\{r\ tastecells,\ fig.cap=\textquotesingle{}(ref:cells)\textquotesingle{},\ echo=FALSE,\ message=FALSE,\ warning=FALSE\}\ knitr::include\_graphics("./figures/sensation/CunninghamSection2Page771Cells.jpg")}

The basic tastes contribute only partially to the sensation and flavor
of food in the mouth---other factors include smell, detected by the
olfactory epithelium of the nose; texture, detected through a variety of
mechanoreceptors; temperature, detected by thermoreceptors; and
``coolness'' (such as of menthol) and ``hotness'' (pungency), through
\href{https://en.wikipedia.org/wiki/Chemesthesis}{chemesthesis}.

As taste senses both harmful and beneficial things, all basic tastes are
classified as either aversive or appetitive, depending upon the effect
the things they sense have on our bodies. Sweetness helps to identify
energy-rich foods, while bitterness serves as a warning sign of poisons.

Among humans, taste perception begins to fade around 50 years of age
because of loss of tongue papillae and a general decrease in saliva
production. Humans can also have distortion of tastes through dysgeusia.
Not all mammals share the same taste senses: some rodents can taste
starch (which humans cannot), cats cannot taste sweetness, and several
other carnivores including hyenas, dolphins, and sea lions, have lost
the ability to sense up to four of their ancestral five taste senses.

\hypertarget{sweetness}{%
\subsubsection{Sweetness}\label{sweetness}}

Sweetness, usually regarded as a pleasurable sensation, is produced by
the presence of sugars and substances that mimic sugar. Sweetness may be
connected to aldehydes and ketones, which contain a carbonyl group.
Sweetness is detected by a variety of G protein coupled receptors (GPCR)
coupled to the G protein gustducin found on the taste buds. At least two
different variants of the ``sweetness receptors'' must be activated for
the brain to register sweetness. Compounds the brain senses as sweet are
compounds that can bind with varying bond strength to two different
sweetness receptors. These receptors are T1R2+3 (heterodimer) and T1R3
(homodimer), which account for all sweet sensing in humans and animals.
Taste detection thresholds for sweet substances are rated relative to
sucrose, which has an index of 1. The average human detection threshold
for sucrose is 10 millimoles per liter. For lactose it is 30 millimoles
per liter, with a sweetness index of 0.3, and 5-nitro-2-propoxyaniline
0.002 millimoles per liter. ``Natural'' sweeteners such as saccharides
activate the GPCR, which releases gustducin. The gustducin then
activates the molecule adenylate cyclase, which catalyzes the production
of the molecule cAMP, or adenosine 3', 5'-cyclic monophosphate. This
molecule closes potassium ion channels, leading to depolarization and
neurotransmitter release. Synthetic sweeteners such as saccharin
activate different GPCRs and induce taste receptor cell depolarization
by an alternate pathway.

\hypertarget{sourness}{%
\subsubsection{Sourness}\label{sourness}}

Sourness is the taste that detects acidity. The sourness of substances
is rated relative to dilute hydrochloric acid, which has a sourness
index of 1. By comparison, tartaric acid has a sourness index of 0.7,
citric acid an index of 0.46, and carbonic acid an index of 0.06.

Sour taste is detected by a small subset of cells that are distributed
across all taste buds in the tongue. Sour taste cells can be identified
by expression of the protein PKD2L1, although this gene is not required
for sour responses. There is evidence that the protons that are abundant
in sour substances can directly enter the sour taste cells through
apically located ion channels. In 2018, the proton-elective ion channel
otopetrin 1 (Otop1) was implicated as the primary mediator of this
proton influx. This transfer of positive charge into the cell can itself
trigger an electrical response. It has also been proposed that weak
acids such as acetic acid, which is not fully dissociated at
physiological pH values, can penetrate taste cells and thereby elicit an
electrical response. According to this mechanism, intracellular hydrogen
ions inhibit potassium channels, which normally function to
hyperpolarize the cell. By a combination of direct intake of hydrogen
ions (which itself depolarizes the cell) and the inhibition of the
hyperpolarizing channel, sourness causes the taste cell to fire action
potentials and release neurotransmitter.

\hypertarget{saltiness}{%
\subsubsection{Saltiness}\label{saltiness}}

The simplest receptor found in the mouth is the sodium chloride (salt)
receptor. Saltiness is a taste produced primarily by the presence of
sodium ions. Other ions of the alkali metals group also taste salty, but
the further from sodium, the less salty the sensation is. A sodium
channel in the taste cell membrane allows sodium cations to enter the
cell. This on its own depolarizes the cell, and opens voltage-dependent
calcium channels, flooding the cell with positive calcium ions and
leading to neurotransmitter release. This sodium channel is known as an
epithelial sodium channel (ENaC) and is composed of three subunits. An
ENaC can be blocked by the drug amiloride in many mammals, especially
rats. The sensitivity of the salt taste to amiloride in humans, however,
is much less pronounced, leading to conjecture that there may be
additional receptor proteins besides ENaC to be discovered.

The size of lithium and potassium ions most closely resemble those of
sodium, and thus the saltiness is most similar. In contrast, rubidium
and caesium ions are far larger, so their salty taste differs
accordingly. The saltiness of substances is rated relative to sodium
chloride (NaCl), which has an index of 1. Potassium, as potassium
chloride (KCl), is the principal ingredient in salt substitutes and has
a saltiness index of 0.6.

Other monovalent cations, e.g.~ammonium
(NH\textsubscript{4}\textsuperscript{+}), and divalent cations of the
alkali earth metal group of the periodic table, e.g.~calcium
(Ca\textsuperscript{2+}), ions generally elicit a bitter rather than a
salty taste even though they, too, can pass directly through ion
channels in the tongue, generating an action potential.

\hypertarget{bitterness}{%
\subsubsection{Bitterness}\label{bitterness}}

Bitterness is one of the most sensitive of the tastes, and many perceive
it as unpleasant, sharp, or disagreeable, but it is sometimes desirable
and intentionally added via various bittering agents. Common bitter
foods and beverages include coffee, unsweetened cocoa, South American
mate, coca tea, bitter gourd, uncured olives, citrus peel, many plants
in the family \emph{Brassicaceae}, dandelion greens, horehound, wild
chicory, and escarole. The ethanol in alcoholic beverages tastes bitter,
as do the additional bitter ingredients found in some alcoholic
beverages including hops in beer and \emph{Gentiana} in bitters. Quinine
is also known for its bitter taste and is found in tonic water.

Bitterness is of interest to those who study evolution, as well as
various health researchers since a large number of natural bitter
compounds are known to be toxic. The ability to detect bitter-tasting,
toxic compounds at low thresholds is considered to provide an important
protective function. Plant leaves often contain toxic compounds, and
among leaf-eating primates there is a tendency to prefer immature
leaves, which tend to be higher in protein and lower in fiber and
poisons than mature leaves. Amongst humans, various food processing
techniques are used worldwide to detoxify otherwise inedible foods and
make them palatable. Furthermore, the use of fire, changes in diet, and
avoidance of toxins has led to neutral evolution in human bitter
sensitivity. This has allowed several loss of function mutations that
has led to a reduced sensory capacity towards bitterness in humans when
compared to other species.

The threshold for stimulation of bitter taste by quinine averages a
concentration of 8 μM (8 micromolar). The taste thresholds of other
bitter substances are rated relative to quinine, which is thus given a
reference index of 1. For example, brucine has an index of 11, is thus
perceived as intensely more bitter than quinine, and is detected at a
much lower solution threshold. The most bitter natural substance is
amarogentin a compound present in the roots of the plant \emph{Gentiana
lutea} and the most bitter substance known is the synthetic chemical
denatonium, which has an index of 1,000. It is used as an aversive agent
(a bitterant) that is added to toxic substances to prevent accidental
ingestion.

Research has shown that TAS2Rs (taste receptors, type 2, also known as
T2Rs) such as TAS2R38 coupled to the G protein gustducin are responsible
for the human ability to taste bitter substances. The TAS2R family in
humans is thought to comprise about 25 different taste receptors, some
of which can recognize a wide variety of bitter-tasting compounds. Over
670 bitter-tasting compounds have been identified, on a bitter database,
of which over 200 have been assigned to one or more specific receptors.
Researchers use two synthetic substances, phenylthiocarbamide (PTC) and
6-n-propylthiouracil (PROP) to study the genetics of bitter perception.
These two substances taste bitter to some people, but are virtually
tasteless to others. Among the tasters, some are so-called
``supertasters'' to whom PTC and PROP are extremely bitter. The
variation in sensitivity is determined by two common alleles at the
TAS2R38 locus.

\hypertarget{savoriness-umami}{%
\subsubsection{Savoriness (Umami)}\label{savoriness-umami}}

Savory, or savoriness is an appetitive taste and is occasionally
described by its Japanese name, umami or ``meaty''.

\hypertarget{the-taste-receptors}{%
\subsubsection{The Taste Receptors}\label{the-taste-receptors}}

There are four types taste receptors. When food or other substances
enter the mouth, molecules interact with saliva and are bound to taste
receptors in the oral cavity and other locations. Molecules which give a
sensation of taste are considered ``sapid''.

Taste receptors are divided into two families:

\begin{itemize}
\tightlist
\item
  Type 1, sweet: TAS1R2+TAS1R3; umami: TAS1R1+TAS1R3
\item
  Type 2, bitter: TAS2R
\end{itemize}

The standard bitter, sweet, or umami taste receptors are G
protein-coupled receptors with seven transmembrane domains. Ligand
binding at the taste receptors activate second messenger cascades to
depolarize the taste cell. Gustducin is the most common taste Gα
subunit, having a major role in TAS2R bitter taste reception. Gustducin
is a homologue for transducin, a G-protein involved in vision
transduction. Additionally, taste receptors share the use of the TRPM5
ion channel, as well as a phospholipase PLCβ2.

The TAS1R1+TAS1R3 heterodimer receptor functions as an umami receptor,
responding to L-amino acid binding, especially L-glutamate. The umami
taste is most frequently associated with the food additive monosodium
glutamate (MSG) and can be enhanced through the binding of inosine
monophosphate (IMP) and guanosine monophosphate (GMP) molecules.
TAS1R1+3 expressing cells are found mostly in the fungiform papillae at
the tip and edges of the tongue and palate taste receptor cells in the
roof of the mouth. These cells are shown to synapse upon the chorda
tympani nerves to send their signals to the brain, although some
activation of the glossopharyngeal nerve has been found.

Alternative candidate umami taste receptors include splice variants of
metabotropic glutamate receptors, mGluR4 and mGluR1, and the
N-methyl-D-aspartate type glutamate ion channel receptor.

The TAS1R2+TAS1R3 heterodimer receptor functions as the sweet receptor
by binding to a wide variety of sugars and sugar substitutes. TAS1R2+3
expressing cells are found in circumvallate papillae and foliate
papillae near the back of the tongue and palate taste receptor cells in
the roof of the mouth. These cells are shown to synapse upon the chorda
tympani and glossopharyngeal nerves to send their signals to the brain.
The TAS1R3 homodimer also functions as a sweet receptor in much the same
way as TAS1R2+3 but has decreased sensitivity to sweet substances.
Natural sugars are more easily detected by the TAS1R3 receptor than
sugar substitutes. This may help explain why sugar and artificial
sweeteners have different tastes. Genetic polymorphisms in TAS1R3 partly
explain the difference in sweet taste perception and sugar consumption
between people of African American ancestry and people of European and
Asian ancestries.

The TAS2R proteins function as bitter taste receptors. There are 43
human TAS2R genes, each of which (excluding the five pseudogenes) lacks
introns and codes for a GPCR protein. These proteins, as opposed to
TAS1R proteins, have short extracellular domains and are located in
circumvallate papillae, palate, foliate papillae, and epiglottis taste
buds, with reduced expression in fungiform papillae. Though it is
certain that multiple TAS2Rs are expressed in one taste receptor cell,
it is still debated whether mammals can distinguish between the tastes
of different bitter ligands. Some overlap must occur, however, as there
are far more bitter compounds than there are TAS2R genes. Common bitter
ligands include cycloheximide, denatonium, PROP
(6-n-propyl-2-thiouracil), PTC (phenylthiocarbamide), and
β-glucopyranosides.

Signal transduction of bitter stimuli is accomplished via the α-subunit
of gustducin. This G protein subunit activates a taste phosphodiesterase
and decreases cyclic nucleotide levels. Further steps in the
transduction pathway are still unknown. The βγ-subunit of gustducin also
mediates taste by activating IP3 (inositol triphosphate) and DAG
(diglyceride). These second messengers may open gated ion channels or
may cause release of internal calcium. Though all TAS2Rs are located in
gustducin-containing cells, knockout of gustducin does not completely
abolish sensitivity to bitter compounds, suggesting a redundant
mechanism for bitter tasting (unsurprising given that a bitter taste
generally signals the presence of a toxin). One proposed mechanism for
gustducin-independent bitter tasting is via ion channel interaction by
specific bitter ligands, similar to the ion channel interaction which
occurs in the tasting of sour and salty stimuli.

One of the best-researched TAS2R proteins is TAS2R38, which contributes
to the tasting of both PROP and PTC. It is the first taste receptor
whose polymorphisms are shown to be responsible for differences in taste
perception. Current studies are focused on determining other such taste
phenotype-determining polymorphisms. More recent studies show that
genetic polymorphisms in other bitter taste receptor genes influence
bitter taste perception of caffeine, quinine and denatonium benzoate.

Historically it was thought that the sour taste was produced solely when
free hydrogen ions (H\textsuperscript{+}) directly depolarised taste
receptors. However, specific receptors for sour taste with other methods
of action are now being proposed. HCN1 and HCN4 (HCN channels) were two
such proposals; both of these receptors are cyclic nucleotide-gated
channels. The two ion channels suggested to contribute to sour taste are
ACCN1 and TASK-1.

Various receptors have also been proposed for salty tastes, along with
the possible taste detection of lipids, complex carbohydrates, and
water. Evidence for these receptors is, however, shaky at best, and is
often unconvincing in mammal studies. For example, the proposed ENaC
receptor for sodium detection can only be shown to contribute to sodium
taste in Drosophilia.

Visual, olfactive, ``sapictive'' (the perception of tastes), trigeminal
(hot, cool), mechanical, all contribute to the perception of taste. Of
these, transient receptor potential cation channel subfamily V member 1
(TRPV1) vanilloid receptors are responsible for the perception of heat
from some molecules such as capsaicin, and a CMR1 receptor is
responsible for the perception of cold from molecules such as menthol,
eucalyptol, and icilin.

\hypertarget{the-gustatory-cortex}{%
\subsubsection{The Gustatory Cortex}\label{the-gustatory-cortex}}

The primary gustatory cortex is a brain structure responsible for the
perception of taste. It consists of two substructures: the anterior
insula on the insular lobe and the frontal operculum on the inferior
frontal gyrus of the frontal lobe. Because of its composition the
primary gustatory cortex is sometimes referred to in literature as the
AI/FO(Anterior Insula/Frontal Operculum). By using extracellular unit
recording techniques, scientists have elucidated that neurons in the
AI/FO respond to sweetness, saltiness, bitterness, and sourness, and
they code the intensity of the taste stimulus.

Like the olfactory system, the taste system is defined by its
specialized peripheral receptors and central pathways that relay and
process taste information. Peripheral taste receptors are found on the
upper surface of the tongue, soft palate, pharynx, and the upper part of
the esophagus. Taste cells synapse with primary sensory axons that run
in the chorda tympani and greater superficial petrosal branches of the
facial nerve (cranial nerve VII), the lingual branch of the
glossopharyngeal nerve (cranial nerve IX), and the superior laryngeal
branch of the vagus nerve (Cranial nerve X) to innervate the taste buds
in the tongue, palate, epiglottis, and esophagus respectively. The
central axons of these primary sensory neurons in the respective cranial
nerve ganglia project to rostral and lateral regions of the nucleus of
the solitary tract in the medulla, which is also known as the gustatory
nucleus of the solitary tract complex. Axons from the rostral
(gustatory) part of the solitary nucleus project to the ventral
posterior complex of the thalamus, where they terminate in the medial
half of the ventral posterior medial nucleus. This nucleus projects in
turn to several regions of the neocortex which includes the gustatory
cortex (the frontal operculum and the insula), which becomes activated
when the subject is consuming and experiencing taste.
